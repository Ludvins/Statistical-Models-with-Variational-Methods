

In Machine Learning and related fields, the distributions are not fully specified
and need to be learned from the data.

From now on, \(\mathcal{V}\) will denote the known data and \(\theta\) the set
of parameters of the data distributions. The main task is to determine this set
of parameters using the information given by the data.

\begin{definition}
\emph{Priors} and \emph{posteriors} typically refer to the parameter
distribution before and after seeing the data, respectively. Using Bayes' rule
\[
  P(\theta \mid  \mathcal{V}) = \frac{P(\mathcal{V}  \mid  \theta)P(\theta)}{P(\mathcal{V})}
\]
The factor \(P(\mathcal{V} \mid \theta)\) is called the \emph{likelihood}.
\end{definition}

Let us see an example of our goal, in it we will try to learn the bias of a coin,
given a set of tossing results.

\begin{exampleth}
  Let \(\{v_n\}_{n \in 0,\dots,N}\) be the results of tossing a coin \(N \in
  \mathbb{N}\) times, let \(1\) symbolize \emph{heads} and \(0\) \emph{tails}.

  Our objective is to estimate the probability \(\theta\) that the coin will be
  head \(P(v_n = 1  \mid  \theta)\), for this we have the random variables \(v_1,\dots,v_n\)
  and \(\theta\), and we require a model \(P(v_1,\dots,v_n,\theta)\). We are
  considering the variables \(v_i\) to be independent to each others, we have a
  Belief Network depicted in figure \ref{fig:learning_coin}
  \[
    P(v_1,\dots,v_n,\theta) = P(\theta)\prod_{n=1}^N P(v_n \mid \theta)
  \]

\begin{figure}[H]
\centering
\begin{tikzpicture}[
  node distance=1cm and 0.5cm,
  mynode/.style={draw,circle,text width=0.5cm,align=center}
]

\node[mynode] (a) {\(\theta\)};
\node[mynode, below=of a] (b) {\(v_i\)};
\plate{} {(b)} {\(N\)}; %
\path (a) edge[-latex] (b)
;

\end{tikzpicture}
\caption{Belief network for coin tossing}
\label{fig:learning_coin}
\end{figure}

We want to calculate
\[
  P(\theta \mid v_1,\dots,v_n) = \frac{P(v_1,\dots,v_n \mid \theta)P(\theta)}{P(v_1,\dots,v_n)}
\]
to do so, we need to specify the prior \(P(\theta)\), we are using a discrete
model where
\[
  P(\theta = 0.2) = 0.1 \hspace{2cm} P(\theta = 0.5) = 0.7 \hspace{2cm} P(\theta =
  0.8) = 0.2
\]
This means that we have a \(70\%\) belief that the coin is fair, a \(10\%\)
belied that is biased to tails and \(20\%\) that is biased to heads.
Notice that \(P(v_n \mid \theta) = \theta\) if \(v_n = 1\) and \(P(v_n \mid \theta) = 1 - \theta\) if \(v_n = 0\).

Let \(n_h\) be the number of heads in our observed data and \(n_t\)
the number of tails

\[
  P(\theta  \mid  v_1,\dots,v_n) = \frac{P(\theta)}{P(\mathcal{V})} \theta^{n_h}(1-\theta)^{n_t}
\]

Suppose now that \(n_h = 2\) and \(n_t = 0.8\), then
\begin{gather*}
  P(\theta = 0.2  \mid  \mathcal{V}) = \frac{1}{P(\mathcal{V})}\times 0.1 \times 0.2^{2}
  \times 0.8^{8} = \frac{1}{P(\mathcal{V})} \times 6.71\times10^{-4} \\
   P(\theta = 0.5  \mid  \mathcal{V}) = \frac{1}{P(\mathcal{V})}\times 0.7 \times 0.5^{2}
   \times 0.5^{8} = \frac{1}{P(\mathcal{V})} \times 6.83\times10^{-4}\\
    P(\theta = 0.8  \mid  \mathcal{V}) = \frac{1}{P(\mathcal{V})}\times 0.2 \times 0.2^{2}
  \times 0.8^{8} = \frac{1}{P(\mathcal{V})} \times 3.27\times10^{-7}
\end{gather*}

Now, we can compute
\[
   \frac{1}{P(\mathcal{V})} =  6.71\times10^{-4} +   6.83\times10^{-4} +
   3.27\times10^{-7} = 0.00135
 \]
 So,
\begin{gather*}
  P(\theta = 0.2  \mid  \mathcal{V}) = 0.4979\\
  P(\theta = 0.5  \mid  \mathcal{V}) = 0.5059\\
  P(\theta = 0.8  \mid  \mathcal{V}) = 0.00024
\end{gather*}

These are the posterior parameter beliefs of our experiment. Given this, it we
were to choose a single value for the posterior it would be \(\theta = 0.5\).
This result is intuitive since, we had a strong belief of the coin being fair
and even though the number of tails was quite bigger than heads, it
was not enough to make the difference. Obviously the posterior of the coin being
biased to tails is now bigger than the prior.

Let us use an uniform prior distribution so that \(P(\theta) = k \implies \int_0^1 P(\theta) d\theta
= k = 1\) due to normalization.

Using the previous calculations we have
\[
  P(\theta \mid  \mathcal{V}) = \frac{1}{P(\mathcal{V})} \theta^{n_h}(1-\theta)^{n_t}
\]
where
\[
  P(\mathcal{V}) = \int_0^1 \theta^{n_h}(1-\theta)^{n_t} d\theta
\]
this implies that
\[
  P(\theta \mid \mathcal{V}) = \frac{\theta^{n_h}(1-\theta)^{n_t} }{ \int_0^1 u^{n_h}(1-u)^{n_t}
    du}\equiv Beta(n_h + 1, n_t + 1)
\]
\end{exampleth}

\begin{definition}
If the posterior distribution is in the same probability distribution family as
the prior distribution, they are then called \emph{conjugate distributions}, and
the prior is called a \emph{conjugate prior} of the likelihood distribution.
\end{definition}

Let's use a Beta distribution as the prior in the last example

\[
  P(\theta) = \frac{1}{B(\alpha, \beta)}\theta^{\alpha - 1}(1 - \theta)^{\beta -
    1} \equiv Beta(\alpha, \beta)
\]
then, repeating the same as before we get that
\[
  P(\theta, \mathcal{V}) = \frac{1}{B(\alpha + n_h, \beta + n_t)}\theta^{\alpha
    + n_h - 1}(1 - \theta)^{\beta + n_t - 1} \equiv Beta(\alpha + n_h, \beta + n_t)
\]

So both the prior and posterior are Beta distributions, then the Beta
distribution is called ``conjugate'' of the Binomial distribution.


\section{Utility}

The Bayesian posterior says nothing about how to summarize the beliefs it
represents, in order to do this we need to specify the utility of each decision.

With this idea we define an utility function over the parameters

\[
  U(\theta, \theta_{true}) = \alpha \mathbb{I}[\theta = \theta_{true}] - \beta
  \mathbb{I}[\theta \neq \theta_{true}]
\]
where \(\theta_{true}\) symbolizes the true value of the parameter, and \(\alpha, \beta \in
\R\).

Then the expected utility of a parameter \(\theta_0\) is calculated as
\[
  U(\theta = \theta_0) = \sum_{\theta_{true}}U(\theta = \theta_0,
  \theta_{true})P(\theta = \theta_{true}  \mid  \mathcal{V})
\]

Using the last example, we may define out utility function as
\[
  U(\theta, \theta_{true}) = 10 \mathbb{I}[\theta = \theta_{true}] - 20
  \mathbb{I}[\theta \neq \theta_{true}]
\]

so the expected utility of the decision that the parameter is \(\theta = 0.2\)
is
\[
  \begin{aligned}
  U(\theta = 0.2) &= U(\theta = 0.2, \theta_{true} = 0.2)P(\theta_{true} = 0.2  \mid
  \mathcal{V})\\
  &+ U(\theta = 0.2, \theta_{true} = 0.5)P(\theta_{true} = 0.5  \mid
  \mathcal{V}) \\
  & +  U(\theta = 0.2, \theta_{true} = 0.8)P(\theta_{true} = 0.8  \mid  \mathcal{V})
\end{aligned}
\]


\section{Maximum A Posteriori and Maximum Likelihood}

The posterior reflects our beliefs about the full range of probabilities, but we
may want to summarize all this information, even though, we may lose loads of it.

\begin{definition}
  Maximum Likelihood is calculated as
  \[
    \theta^{ML} = \argmax_\theta p(\mathcal{V} \mid \theta)
  \]
   it refers to the value of the parameter
\(\theta\) for which the observed data better fits the model.
\end{definition}

\begin{definition}
  Maximum A Posteriori is calculated as
  \[
    \theta^{MAP} = \argmax_\theta p(\mathcal{V} \mid \theta)P(\theta)
  \]
\end{definition}

The decision of taking the Maximum A Posteriori can be motivated using an
utility that equals zero for all but the correct parameter
\[
  U(\theta, \theta_{true}) = \mathbb{I}[\theta = \theta_{true}]
\]

using this, the expected utility of a parameter \(\theta = \theta_0\) is

\[
  U(\theta = \theta_0) = \sum_{\theta_{true}}\mathbb{I}[\theta_{true} = \theta_0]P(\theta = \theta_{true}  \mid  \mathcal{V}) = P(\theta_0  \mid  \mathcal{V})
\]

This means that the maximum utility decision is to take the value \(\theta_0\)
with the highest posterior value.


It is worth mentioning that when using a flat prior \(\theta^{ML}
= \theta ^{MAP}\).


Let \(\{x_1, \dots, x_m\}\) be a set of discrete variables, we can define the
empirical distribution as a distribution of the variables whose mass probability
function is
\[
  Q(x) = \frac{1}{m}\sum_{i = 1}^m \mathbb{I}[x = x_i]
\]

Now, we are going to show the relation between the Maximum Likelihood and the
Kullback-Leibler divergence of the empirical distribution and our model. We may calculate this Kullback-Leibler divergence and study their functional independence.

\[
  KL(Q \mid P) = \E[\log(Q(x))]_{Q} - \E[\log(P(x))]_Q
\]

Notice the term \(\E[\log(Q(x))]_{Q} \) is a constant and assume the data is i.i.d, so that
\[
   \E[\log(P(x))]_Q = \frac{1}{m}\sum_{i = 1}^mlogP(x_i)
 \]
 where the right side is the log likelihood under \(P(x)\). As the logarithm is
 a strictly increasing function, maximizing the log likelihood equals to
 maximize the likelihood itself, and we can see here how it is equivalent to
 minimize the Kullback-Leibler divergence between the empirical distribution and
 our distribution.

 In case \(P(x)\) is unconstrained, the optimal choice is \(P(x) = Q(x)\), that
 is, the maximum likelihood distribution corresponds to the empirical distribution.

 For a Belief Network \(P(x)\) presents the following constraint
 \[
   P(x) = \prod_{i = 1}^K P(x_i  \mid  pa(x_i))
 \]
 We now want to minimize the Kullback-Leibler divergence between the empirical
 distribution \(Q(x)\) and \(P(x)\) in order to get the Maximum Likelihood.

 \[
   \begin{aligned}
   KL(Q \mid P) &= - \E\big[\sum_{i = 1}^K\log{P(x_i \mid pa(x_i))}\big]_Q +
   \E\big[\sum_{i = 1}^K\log{P(x_i \mid pa(x_i))}\big]_P
   \\ &= - \sum_{i =
     1}^K \E\big[\log{P(x_i \mid pa(x_i))}\big]_Q + \sum_{i =
     1}^K \E\big[\log{P(x_i \mid pa(x_i))}\big]_P
   \end{aligned}
 \]
 We can now use that \(\log{P(x_i \mid pa(x_i))}\) only depends on
 \(Q(x_i  \mid  pa(x_i))\) to rewrite it as

 \[
   \begin{aligned}
     KL(Q \mid P) &= \sum_{i = 1}^K \E\Big[ \log{Q(x_i \mid pa(x_i))}\Big]_{Q(x_i,pa(x_i))} - \E\Big[
     \log{P(x_i \mid pa(x_i))}\Big]_{Q(x_i,pa(x_i))} \\
     &= \sum_{i = 1}^K \E \Big[ KL\Big(Q(x_i \mid pa(x_i)) \mid P(x_i \mid pa(x_i))\Big) \Big]_{Q(x_i,pa(x_i))}
   \end{aligned}
 \]

 The minimal setting is then
 \[
   P(x_i \mid pa(x_i)) = Q(x_i \mid pa(x_i))
 \]
 in terms of the initial data it is to set \(P(x_i \mid pa(x_i))\) to the number of
 times the state appears in it.

 \section{Bayesian Belief Network Training}

A Bayesian approach where we set a distribution over the parameters is an
alternative to Maximum Likelihood training of a Bayesian Network.

To illustrate this, we are using the following scenario, consider a disease
\(D\) and two habits \(A\) and \(B\), following the Bayesian Network

\begin{figure}[!ht]
  \centering
  \begin{tikzpicture}[
    node distance=1cm and 0.5cm,
    mynode/.style={draw,circle,text width=0.5cm,align=center}
    ]

    \node[mynode] (d) {D};
    \node[mynode, left=of d] (a) {A};
    \node[mynode, right=of d] (b) {B};

    \path (a) edge[-latex] (d)
    (b) edge[-latex] (d)
    ;

  \end{tikzpicture}
  \qquad
  \begin{tabular}{|l|l|l|}
    \hline
    A & B & D \\ \hline
    1 & 1 & 1 \\ \hline
    1 & 0 & 0 \\ \hline
    0 & 1 & 1 \\ \hline
    0 & 1 & 0 \\ \hline
    1 & 1 & 1 \\ \hline
    0 & 0 & 0 \\ \hline
    1 & 0 & 1 \\ \hline
  \end{tabular}
  \caption{Model for the relationship between \(D,A\) and \(B\), and
    observations}
 \label{fig:bayesian_example}
\end{figure}



\[
P(a,b,d) = P(d|a,b)P(a)P(b)
\]

Consider a set of observations
\(\mathcal{V} = \{(a_{n}, b_{n}, d_{n}), n = 1,\dots , N\}\)

We need a notation for the parameters, as all the variables are binary we are
going to use
\[
  P(A = 1 \mid \theta_{a}) = \theta_{a}, \hspace{1cm} P(B = 1 \mid \theta_{b} = \theta_{b}), \hspace{1cm} P(D = 1 \mid A = 0, B = 1, \theta_{d}) = \theta_{1}
\]
\[\theta_{d} = (\theta_{0}, \theta_{1}, \theta_{2}, \theta_{3})\]
Using the binary number \(A\) and \(B\) create \((01)\) and its correspondent
decimal form in the sub-index of \(\theta\).

We need to specify a prior and since dealing with multi-dimensional continuous
distributions is computationally problematic it is normal to use uni-variate
distributions.

A convenient assumption is that the prior factorizes, this is usually called
\emph{global parameter independence}. We assume then
\[
  P(\theta_{a}, \theta_{b}, \theta_{d}) = P(\theta_{a})P(\theta_{b})P(\theta_{d})
\]
Assuming our data is i.i.d, we have
\[
  P(\theta_{a}, \theta_{b}, \theta_{d}, \mathcal{V}) = P(\theta_{a})P(\theta_{b})P(\theta_{d})\prod_{n}P(a_{n}\mid \theta_{a})P(b_{n} \mid \theta_{b})P(d_{n}\mid a_{n}, b_{n}, \theta_{d})
\]

Learning then corresponds to inference

\[
  \begin{aligned}
    P(\theta_{a}, \theta_{b}, \theta_{d}\mid \mathcal{V}) &= \frac{P(\mathcal{V} \mid \theta_{a}, \theta_{b}, \theta_{d})P(\theta_{a}, \theta_{b}, \theta_{d})}{P(\mathcal{V})} =\frac{P(\mathcal{V} \mid \theta_{a}, \theta_{b}, \theta_{d})P(\theta_{a}) P(\theta_{b})P( \theta_{d})}{P(\mathcal{V})}\\
    &= \frac{1}{P(\mathcal{V})}P(\theta_{a})\prod_{n}P(a_{n}\mid \theta_{a})P(\theta_{b})\prod_{n}P(b_{n}\mid \theta_{b})P(\theta_{d})\prod_{n}P(d_{n}\mid a_{n}, b_{n},\theta_{d})\\
    &= P(\theta_{a} \mid \V_{a} )P(\theta_{b}\mid \V_{b})P(\theta_{d} \mid \V)
  \end{aligned}
\]

Where \(V_{i}\) is the subset of the data restricted to the variable \(i\). If
we further assume that \(P(\theta_{d})\) factorizes as
\(P(\theta_{d}) = P(\theta_{0})P(\theta_{1})P(\theta_{2})P(\theta_{3})\),
this is called \emph{local parameter independence}, then it follows that
\[
  P(\theta_{d}\mid \V) = P(\theta_{0} \mid \V )P(\theta_{1} \mid \V )P(\theta_{2} \mid \V )P(\theta_{3} \mid \V )
\]

The simplest cases to continue are \(P(a\mid \theta_{a})\) and
\(P(b \mid \theta_{b})\) since they require only a univariate prior distribution
\(P(\theta_{a})\) or \(P(\theta_{b})\). Lets illustrate it using
\(P(\theta_{a})\):

The posterior is
\[
  P(\theta_{a} \mid \V_{a}) = \frac{1}{P(\V_{a})}P(\theta_{a})\theta_{a}^{\#(a=1)}(1-\theta_{a})^{\#(a=0)}
\]

The most convenient choice for the prior is a Beta distribution as conjugacy
will hold.

\[
  P(\theta_{a}) = B(\theta_{a} \mid \alpha_{a}, \beta_{a}) = \frac{1}{B(\alpha_{a}, \beta_{a})}\theta_{a}^{\alpha_{a}-1}(1-\theta_{a})^{\beta_{a} - 1}
\]
\[
  P(\theta_{a} \mid \V_{a}) = B(\theta_{a} \mid \alpha_{a} + \#(a=1), \beta_{a} + \#(a = 0))
\]

The marginal is then
\[
  \begin{aligned}
    P(a = 1 \mid \V_{a})
    &= \frac{P(a = 1, \V_{a})}{P(\V_{a})} = \int_{\theta_{a}}  \frac{P(a = 1, \V_{a}, \theta_{a})}{P(\V_{a})} =  \int_{\theta_{a}}  \frac{P(a = 1 \mid \V_{a}, \theta_{a}) P(\V_{a}, \theta_{a})}{P(\V_{a})} \\
    &=  \int_{\theta_{a}}  \frac{P(a = 1 \mid \V_{a}, \theta_{a}) P(\theta_{a} \mid \V_{a})P(\V_{a})}{P(\V_{a})} = \int_{\theta_{a}}P(\theta_{a}\mid \V_{a})\theta_{a} = \E[\theta_{a} \mid \V_{a}] \\
    &= \frac{\alpha_{a} + \#(a= 1)}{\alpha_{a} + \#(a=1) + \beta_{a} + \#(a=0)}
  \end{aligned}
\]

For \(P(d \mid a ,b)\) the situation is more complex, the most convenient way is
to specify a Beta prior for each one of the four components of \(\theta_{d}\).
Lets focus on \(P(d = 1 \mid a = 1, b = 0)\), notice the parameters \(\alpha\)
and \(\beta\) we used before now depend on \(a\) and \(b\), for this reason we
are using \(\alpha_{d}(a,b)\) and \(\beta_{d}(a,b)\) as prior parameters, these
are called \emph{hyperparameters}.
\[
  P(\theta_{2}) = B(\theta_{2} \mid \alpha_{d}(1,0) + \#(d = 1, a = 1, b = 0), \beta_{d}(1,0) + \#(d = 0, a = 1, b = 0))
\]

As before we got that

\[
  P(d = 1 \mid a = 1, b = 0, \V) = \frac{\alpha_{d}(1,0) + \#(d = 1, a = 1, b = 0)}{\alpha_{d}(1,0) + \beta_{d}(1,0) + \#(a=1, b = 0)}
\]

In case we had no preference, we could set all hyperparameters to the same
value, and, a complete ignorance prior would correspond to set them to 1.

Let now consider two limit possibilities, the one where we have no data at all,
and the one where we have infinite data.

In case we have no data, the marginal probability corresponds to the prior which
in the last case is
\[
   P(d = 1 \mid a = 1, b = 0, \V) = \frac{\alpha_{d}(1,0)}{\alpha_{d}(1,0) + \beta_{d}(1,0)}
 \]
 Note that equal hyperparameters would give a result of \(0.5\).

 When infinite data is available, the marginal is generally dominated by it,
 this corresponds to the Maximum Likelihood solution.
 \[
   P(d = 1 \mid a = 1, b = 0, \V) = \frac{\#(d = 1, a = 1 , b = 0)}{\#(a = 1, b = 0)}
 \]
 This happens unless the prior has a pathologically strong effect.

 Consider the data given in the table in figure \ref{fig:bayesian_example}, and
 equal parameters and hyperparameters \(1\). Then we can compute the differences
 between this and using the Maximum Likelihood technique.
 \[
   P(a = 1 \mid \V) = \frac{1 + \#(a = 1)}{2 + N} = \frac{5}{9} \approx 0.556
 \]
 By comparison, the Maximum Likelihood result is \(4/7 = 0.571\), the Bayesian
 result is more prudent than this one, which fits in with our belief that any
 setting is equally probable i.e \(0.5\).


 The natural generalization to more than two states is using a Dirichlet
 distribution as prior, assuming i.i.d data and local and global prior independence.

 Consider a variable \(X\) with \(Dom(X) = \{1, \dots, I\}\), then
 \[
   P(x \mid \theta) = \prod_{i = 1}^{I}\theta_{i}^{\mathbb{I}[x = i]} \text{
   with  } \sum_{i=1}^{I}\theta_{i} = 1
\]
So that the posterior (considering \(N\) observations of the variable
\((x_{1}, \dots, x_{N}) = \V\)) is
\[
  P(\theta \mid x_{1},\dots,x_{N}) = \frac{1}{P(\V)} P(\theta) \prod_{n = 1}^{N}\prod_{i =1 }^{I}\theta_{i}^{\mathbb{I}[x_{n} = i]} =  \frac{1}{P(\V)} P(\theta) \prod_{i = 1}^{I} \theta_{i}^{\sum_{n} \mathbb{I}[x_{n}=i]}
\]
Then assuming a Dirichlet prior with hyperparameters \(\bm{u} = (u_{1}, \dots, u_{N})\)
\[
  P(\theta) = \frac{1}{B(\bm{u})}\prod_{i =1}^{I}\theta_{i}^{u_{i}-1} \implies P(\theta \mid \V) = \frac{1}{B(\bm{u})P(\V)}\prod_{i=1}^{I}\theta_{i}^{u_{i}-1 + \sum_{n}\mathbb{I}[x_{n} = i]}
\]

Which means that, defining \(\bm{c} = ( \sum_{n=1}^{N}\mathbb{I}[x_{n} = i])_{i \in \I}\)
\[
  P(\theta \mid \V) = \text{Dirichlet}(\theta \mid \bm{u} + \bm{c})
\]

The marginal is then given by
\[
  \begin{aligned}
    P(x=i \mid \V) &= \int_{\theta}P(x=i \mid \theta)P(\theta \mid \V) =  \int_{\theta}\theta_{i}P(\theta \mid \V)\\
    &=  \int_{\theta_{i}}\theta_{i}P(\theta_{i} \mid \V)
\end{aligned}
\]
Where we used that
\[\int_{\theta_{j \neq i}}\theta_{i} P(\theta \mid \V) = \theta_{i}\prod_{k\neq j}P(\theta_{k} \mid V) \int_{\theta_{j}}P(\theta_{j}\mid \V) = \theta_{i} \prod_{k \neq j}P(\theta_{k} \mid \V)\]
