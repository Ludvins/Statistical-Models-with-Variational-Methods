

\begin{definition}
A \emph{potential} \(\phi\) is a non-negative function. It is worth to mention
that a probability distribution is a special case of a potential.
\end{definition}

\begin{definition}
Let \(\bm{X}\) be a set of random variables, \(G\) an undirected graph,
\(\bm{X}_c, c \in \{1,\dots,C\}\) be the maximal cliques of \(G\) and \(P\) a
probability distribution over \(\bm{X}\). The pair \((G,
P)\) is said to be a \emph{Markov network or Markov random field} if, and only if
\[
P(x_1,\dots,x_n) = \frac{1}{Z}\prod_{ c = 1 }^{C}\phi_c(\bm{X}_c)
\]
where \(Z\) is a constant that ensures normalization.
\end{definition}

\begin{figure}[h]
\centering
\begin{tikzpicture}[
  node distance=1cm and 1.5cm,
  mynode/.style={draw,circle,text width=0.5cm,align=center}
]

\node[mynode] (1) {\(X_1\)};
\node[mynode,right=of 1] (2) {\(X_2\)};
\node[mynode,below=of 1] (3) {\(X_3\)};
\node[mynode,right=of 3] (4) {\(X_4\)};
\path (1) edge (2)
(1) edge (3)
(2) edge (3)
(2) edge (4)
(1) edge (4)
;
\end{tikzpicture}
\captionof{figure}{Markov Network \(P(x_1, x_2, x_3, x_4) = \phi(x_1, x_2,
  x_3)\phi(x_2, x_3, x_4)/Z\)}
\label{fig:mn_example}
\end{figure}


In figure \ref{fig:mn_example} we can see an example of the factorization, without
  giving any reference of the potentials.

  Let \((G,P)\) be a Markov network, then it satisfies the following properties
  known as Markov properties:
  \begin{itemize}
  \item Pairwise Markov property. Any two non-adjacent variables are
    conditionally independent given all other variables.
  \item Local Markov property. A variable is conditionally independent over all
    other variables given it's neighbors. That is,
        \[
      P(x_i | x_{\backslash i}) = p(x_i | ne(x_i)) \footnote{\(X_{\backslash i} =
        \{X_j \ | \ j \neq i\} \subset \bm{X} \) }
    \]
  \item Global Markov property. Any two subsets of variables are conditionally
    independent given a separating subset (any path from one set to the other
    passes trough this one).
  \end{itemize}

  \begin{remark}
    A Markov network can also be defined as a pair \((G, P)\) such as all Markov
    properties are satisfied. The clique factorization definition is a special
    case of these properties.
  \end{remark}


  \begin{definition}
    Let \(G\) be an undirected graph and \(P\) a probability distribution over a
    set of random variables \(\bm{X}\). The pair \((G, P)\) is called a Markov
    Random Field if and only if it follows the local Markov Property.


  \end{definition}
