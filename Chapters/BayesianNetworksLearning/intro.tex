
In Machine Learning and related fields, the distributions are not fully specified
and need to be learned from the data.

From now on, \(\mathcal{V}\) will denote the known data and \(\theta\) the set
of parameters of the data distributions. The main task is to determine this set
of parameters using the information given by the data. No latent or hidden variables are being considered right now.

The goal now is to determine the posterior \(P(\theta \mid \V)\) using a prior belief \(P(\theta)\) and the likelihood function \(P(\V \mid \theta)\).  
\[
  P(\theta \mid  \mathcal{V}) = \frac{P(\mathcal{V}  \mid  \theta)P(\theta)}{P(\mathcal{V})}
\]

We are using the following example to illustrate the situation, in it we will try to learn the bias of a coin, given a set of tossing results.

\begin{exampleth}
  Let \(\V = \{v_n\}_{n \in 0,\dots,N}\) be the results of tossing a coin \(N \in
  \mathbb{N}\) times, let \(1\) symbolize \emph{heads} and \(0\) \emph{tails}.

  Our objective is to estimate the probability \(\theta\) that the coin will be
  head \(P(v_n = 1  \mid  \theta)\), for this we have the i.i.d random variables \(v_1,\dots,v_n\)
  and \(\theta\), and we require a model \(P(v_1,\dots,v_n,\theta)\). We have a
  Belief Network shown in figure \ref{fig:learning_coin}
  \[
    P(\V,\theta) = P(\theta)\prod_{n=1}^N P(v_n \mid \theta)
  \]

\begin{figure}[H]
\centering
\begin{tikzpicture}[
  node distance=1cm and 0.5cm,
  mynode/.style={draw,circle,text width=0.5cm,align=center}
]

\node[mynode] (a) {\(\theta\)};
\node[mynode, below=of a] (b) {\(v_i\)};
\plate{} {(b)} {\(N\)}; %
\path (a) edge[-latex] (b)
;

\end{tikzpicture}
\caption{Belief network for coin tossing}
\label{fig:learning_coin}
\end{figure}

We want to calculate
\[
  P(\theta \mid \V) = \frac{P(\V \mid \theta)P(\theta)}{P(\V)}
\]
to do so, we need to specify the prior \(P(\theta)\), we are using a discrete
model where
\[
  P(\theta = 0.2) = 0.1 \hspace{2cm} P(\theta = 0.5) = 0.7 \hspace{2cm} P(\theta =
  0.8) = 0.2
\]
This means that we have a \(70\%\) belief that the coin is fair, a \(10\%\)
belied that is biased to tails and \(20\%\) that is biased to heads.
Notice that \(P(v_n = 1 \mid \theta) = \theta\) and \(P(v_n = 0 \mid \theta) = 1 - \theta\).

Let \(n_h\) be the number of heads in our observed data and \(n_t\)
the number of tails
\[
  n_{h} = \#\{v= 1\} \hspace{2cm} n_{t} = \#\{v = 0\}
\]

then the posterior has the form

\[
  P(\theta  \mid \V) = \frac{P(\theta)}{P(\mathcal{V})} \theta^{n_h}(1-\theta)^{n_t}
\]

Suppose now that \(n_h = 2\) and \(n_t = 8\), then
\begin{gather*}
  P(\theta = 0.2  \mid  \mathcal{V}) = \frac{1}{P(\mathcal{V})}\times 0.1 \times 0.2^{2}
  \times 0.8^{8} = \frac{1}{P(\mathcal{V})} \times 6.71\times10^{-4} \\
   P(\theta = 0.5  \mid  \mathcal{V}) = \frac{1}{P(\mathcal{V})}\times 0.7 \times 0.5^{2}
   \times 0.5^{8} = \frac{1}{P(\mathcal{V})} \times 6.83\times10^{-4}\\
    P(\theta = 0.8  \mid  \mathcal{V}) = \frac{1}{P(\mathcal{V})}\times 0.2 \times 0.2^{2}
  \times 0.8^{8} = \frac{1}{P(\mathcal{V})} \times 3.27\times10^{-7}
\end{gather*}

Now, we can compute
\[
   \frac{1}{P(\mathcal{V})} =  6.71\times10^{-4} +   6.83\times10^{-4} +
   3.27\times10^{-7} = 0.00135
 \]
 So,
\begin{gather*}
  P(\theta = 0.2  \mid  \mathcal{V}) = 0.4979\\
  P(\theta = 0.5  \mid  \mathcal{V}) = 0.5059\\
  P(\theta = 0.8  \mid  \mathcal{V}) = 0.00024
\end{gather*}

These are the posterior parameter beliefs of our experiment. Given this, it we
were to choose a single value for the posterior it would be \(\theta = 0.5\).
This result is intuitive, we had a strong belief of the coin being fair
and even though the number of tails was quite bigger than heads, it
was not enough to make the difference. Obviously the posterior of the coin being
biased to tails is now bigger than the prior.

Suppose an uniform prior distribution so that \(P(\theta) = k \implies \int_0^1 P(\theta) d\theta
= k = 1\) due to normalization.

Using the previous calculations we have
\[
  P(\theta \mid  \mathcal{V}) = \frac{1}{P(\mathcal{V})} \theta^{n_h}(1-\theta)^{n_t}
\]
where
\[
  P(\mathcal{V}) = \int_0^1 \theta^{n_h}(1-\theta)^{n_t} d\theta
\]
this implies that
\[
  P(\theta \mid \mathcal{V}) = \frac{\theta^{n_h}(1-\theta)^{n_t} }{ \int_0^1 u^{n_h}(1-u)^{n_t}
    du} \implies \theta \mid \V \sim Beta(n_h + 1, n_t + 1)
\]
\end{exampleth}

Let's use a Beta distribution as the prior in the last example

\[
  \theta \sim \text{Beta}(\alpha, \beta) \implies P(\theta) = \frac{1}{B(\alpha, \beta)}\theta^{\alpha - 1}(1 - \theta)^{\beta -
    1}
\]
then, repeating the same as before we get that
\[
  P(\theta, \mathcal{V}) = \frac{1}{B(\alpha + n_h, \beta + n_t)}\theta^{\alpha
    + n_h - 1}(1 - \theta)^{\beta + n_t - 1} \implies (\theta, \V) \sim Beta(\alpha + n_h, \beta + n_t)
\]

So both the prior and posterior are Beta distributions, then the Beta
distribution is called ``conjugate'' of the Binomial distribution.

\section{Utility}

The Bayesian posterior says nothing about how to benefit from the beliefs it
represents, in order to do this we need to specify the utility of each decision.

With this idea we define an utility function over the parameters

\[
  U(\theta, \theta_{true}) = \alpha \mathbb{I}[\theta = \theta_{true}] - \beta
  \mathbb{I}[\theta \neq \theta_{true}]
\]
where \(\alpha, \beta \in \R\). This symbolizes the gains or looses of choosing
the parameter \(\theta\), when the true value of the parameter is supposed to be
\(\theta_{true}\). Then the expected utility of a parameter \(\theta_0\) is
calculated as
\[
  U(\theta = \theta_0) = \sum_{\theta_{true}}U(\theta = \theta_0,
  \theta_{true})P(\theta = \theta_{true}  \mid  \mathcal{V})
\]

Using the last example, we may define out utility function as
\[
  U(\theta, \theta_{true}) = 10\mathbb{I}[\theta = \theta_{true}] - 20
  \mathbb{I}[\theta \neq \theta_{true}]
\]
Where we interpret that the loss of choosing the wrong parameter is twice as
important as the gains from doing it right.

The expected utility of the decision that the parameter is \(\theta = 0.2\)
in our discrete example would be
\[
  \begin{aligned}
  U(\theta = 0.2) &= U(\theta = 0.2, \theta_{true} = 0.2)P(\theta_{true} = 0.2  \mid
  \mathcal{V})\\
  &+ U(\theta = 0.2, \theta_{true} = 0.5)P(\theta_{true} = 0.5  \mid
  \mathcal{V}) \\
  & +  U(\theta = 0.2, \theta_{true} = 0.8)P(\theta_{true} = 0.8  \mid  \mathcal{V})\\
  &= 10 \times 0.4979 - 20\times 0.5059 -20 \times 0.00024 \\
  &= -5.1438\\
  U(\theta = 0.5) &= -4.9038 \\
  U(\theta = 0.8) &= -20.0736
\end{aligned}
\]


This illustrate how an utility function can affect the results of the inference.
The most probable value for \(\theta\) was \(0.2\), but, using this utility
function, \(0.5\) is the one with which we expect minor losses.
