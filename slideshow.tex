\documentclass{beamer}
\usetheme{metropolis}
\usepackage{bm}
\usepackage{xcolor}
\colorlet{shadecolor}{blue!15}
\usepackage{framed}
\usepackage{amsthm}
\usepackage[utf8]{inputenc}

\newcommand{\bx}{\bm{x}}
\newcommand{\bX}{\bm{X}}
\newcommand{\by}{\bm{y}}
\newcommand{\bY}{\bm{Y}}
\newcommand{\bw}{\bm{w}}
\newcommand{\bW}{\bm{W}}
\newcommand{\bz}{\bm{z}}
\newcommand{\bZ}{\bm{Z}}
\newcommand{\bv}{\bm{v}}
\newcommand{\bh}{\bm{h}}
\newcommand{\bSigma}{\bm{\Sigma}}
\newcommand{\bpi}{\bm{\pi}}
\newcommand{\bLambda}{\bm{\Lambda}}
\newcommand{\btheta}{\bm{\theta}}
\newcommand{\bnu}{\bm{\nu}}
\newcommand{\bV}{\bm{V}}
\DeclareMathOperator*{\argmax}{arg\,max}
\DeclareMathOperator*{\argmin}{arg\,min}

\newcommand{\V}{\mathcal{V}}
\newcommand{\D}{\mathcal{D}}
\newcommand{\X}{\mathcal{X}}
\newcommand{\I}{\mathcal{I}}

\newcommand\ddfrac[2]{\frac{\displaystyle #1}{\displaystyle #2}}

\newcommand\E[2]{\mathbb{E}_{#1}\Big[#2\Big]}
\newcommand\KL[2]{KL\Big(#1 \bigm| #2\Big)}
\newcommand{\bigCI}{\mathrel{\text{\scalebox{1.07}{$\perp\mkern-10mu\perp$}}}}
\newcommand{\bigCD}{\centernot{\bigCI}}

\newtheorem*{prop}{Proposition}
\newenvironment{proposition}
  {\begin{shaded}\begin{prop}}
  {\end{prop}\end{shaded}}




\title{Statistical models with variational methods}
\date{\today}
\author{Luis Antonio Ortega Andrés}
\institute{Here we go again...}
\begin{document}
  \maketitle
  \section{First Section}

  \begin{frame}{Notation}
    \begin{itemize}
      \item \(X\) denotes a generic observed random variable.
      \item \(Z\) denotes a generic hidden random variable.
      \item \(P(x)\) denotes \(X\)'s mass function of density function evaluated on \(x\).
      \item Bold symbols as \(\bm{X}\) refer to sets of variables or multidimensional variables.
      \item \(Q\) symbolizes a variational distribution.
      \item \(\bx = (x_{1},\dots,x_{N})\) symbolizes a set of observations and \(\btheta\) a set of parameters.
    \end{itemize}
  \end{frame}

  \begin{frame}{Kullback-Leibler divergence}
    Let \(P\) and \(Q\) be two probability distributions over the same probability space \( \Omega \), the \emph{Kullback-Leibler divergence}
    \(\KL{Q}{P}\) measures the ``difference'' between both distributions as
    \[
      \KL{Q}{P} = \E{Q}{\log Q(x) - \log P(x)}.
    \]
    \begin{shaded}
      The Kullback-Leibler divergence is always non-negative.
    \end{shaded}
  \end{frame}

  \begin{frame}{Statistical Inference}
    The process of deducing properties of an underlying distribution.

    \begin{itemize}
      \item \textbf{Likelihoodist}: performs parameter estimation maximizing the \emph{likelihood function} \(P(\bx \mid \btheta)\).
      \item \textbf{Bayesian}: derives a \emph{posterior distribution} using Bayes' theorem
        \[
        \underbrace{P(\btheta \mid \bx)}_{posterior} = \frac{\overbrace{P(\bx \mid \btheta)}^{likelihood}\overbrace{P(\btheta)}^{prior}}{P(\bx)}.
        \]
    \end{itemize}
  \end{frame}


  \begin{frame}{Maximum likelihood estimation}
    Attempts to compute the value of the parameter to which the data is most probable to be generated with, that is,
    \[
      \btheta^{ML} = \argmax_{\btheta} P(\bx \mid \btheta).
    \]
    This value minimizes the Kullback-Leibler divergence between the distribution \(P\) and the empirical distribution \(Q\):
    \[
      \btheta^{ML} = \argmin_{\btheta} \KL{Q}{P}\quad \text{where}\quad Q(x) = \frac{1}{N}\sum_{n=1}^{N}\mathbb{I}[x = x_{n}].
    \]
  \end{frame}

  \begin{frame}{Variational inference}
    Solves the inference problem by creating an equivalent optimization problem:

    \begin{shaded}
      Given a fixed family of distributions \(\mathcal{Q}\) over the set of latent variables \(\bZ\), find
      \[
        Q^{opt} = \argmin_{Q \in \mathcal{Q}} \KL{Q(\bz)}{P(\bz \mid \bx)}.
      \]
    \end{shaded}

    This problem is then approached through machine learning techniques, such as \emph{gradient or coordinate descent}.
  \end{frame}

  \begin{frame}{Evidence Lower Bound}
    Using the positiveness of the Kullback-Leibler divergence, the evidence lower bound (ELBO) is established.
    \[
      \KL{Q(\bz)}{P(\bz \mid \bx)} \geq 0
    \]
    \[
      \Updownarrow
    \]
    \[
      \log P(\bx) \geq \underbrace{- \E{Q(\bz)}{\log Q(\bz)}}_{Entropy} + \underbrace{\E{Q(\bz)}{\log P(\bx, \bz)}}_{Energy}.
    \]
  \end{frame}

  \begin{frame}{Algorithms}
    Two main algorithms arise to study models with latent variables using this lower bound:
    \begin{itemize}
      \item \textbf{Expectation-Maximization}: likelihoodist iterative method to estimate model parameters.
      \item \textbf{Coordinate Ascent Variational Inference}: Bayesian coordinate ascent approach.
    \end{itemize}
  \end{frame}

  \begin{frame}{Expectation-Maximization algorithm}
    Attempts to maximize the ELBO given a parametric model.
    \[
      ELBO(Q, \btheta) = \underbrace{- \E{Q(\bz)}{\log Q(\bz)}}_{Entropy} + \underbrace{\E{Q(\bz)}{\log P(\bx, \bz \mid \btheta)}}_{Energy}.
    \]
    Two alternating maximization steps:
    \begin{itemize}
      \item \textbf{E-step}: Given a fixed parameter \(\btheta\),
        \[
        Q^{new}(\bz) = \argmax_{Q} ELBO(Q, \btheta) = P(\bz \mid \bx, \btheta).
        \]
      \item \textbf{M-step}: Given a fixed distribution \(Q\),
        \[
        \btheta^{new} = \argmax_{\btheta} ELBO(Q, \btheta) = \argmax_{\btheta} \E{Q(\bz)}{\log P(\bx, \bz \mid \btheta)}.
        \]
    \end{itemize}
  \end{frame}

  \begin{frame}{Coordinate ascent variational inference algorithm}
    hola
  \end{frame}
\end{document}
