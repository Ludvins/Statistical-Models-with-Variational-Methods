
% Plantilla para un Trabajo Fin de Grado de la Universidad de Granada,
% adaptada para el Doble Grado en Ingeniería Informática y Matemáticas.
%
%  Autor: Mario Román.
%  Licencia: GNU GPLv2.
%
% Esta plantilla es una adaptación al castellano de la plantilla
% classicthesis de André Miede, que puede obtenerse en:
%  https://ctan.org/tex-archive/macros/latex/contrib/classicthesis?lang=en
% La plantilla original se licencia en GNU GPLv2.
%
% Esta plantilla usa símbolos de la Universidad de Granada sujetos a la normativa
% de identidad visual corporativa, que puede encontrarse en:
% http://secretariageneral.ugr.es/pages/ivc/normativa
%
% La compilación se realiza con las siguientes instrucciones:
%   pdflatex --shell-escape main.tex
%   bibtex main
%   pdflatex --shell-escape main.tex
%   pdflatex --shell-escape main.tex

% Opciones del tipo de documento
\documentclass[oneside,openright,titlepage,numbers=noenddot,openany,headinclude,footinclude=true,
  cleardoublepage=empty,abstractoff,BCOR=5mm,paper=a4,fontsize=12pt]{scrreprt}

% Paquetes de latex que se cargan al inicio. Cubren la entrada de
% texto, gráficos, código fuente y símbolos.
\usepackage[utf8]{inputenc}
\usepackage[T1]{fontenc}
\usepackage{fixltx2e}
\usepackage{graphicx} % Inclusión de imágenes.
\usepackage{grffile}  % Distintos formatos para imágenes.
\usepackage{longtable} % Tablas multipágina.
\usepackage{wrapfig} % Coloca texto alrededor de una figura.
\usepackage{rotating}
\usepackage[normalem]{ulem}
\usepackage{amsmath}
\usepackage{textcomp}
\usepackage{amssymb}
\usepackage{capt-of}
\usepackage[colorlinks=true]{hyperref}
\usepackage{tikz} % Diagramas conmutativos.
\usepackage{minted} % Código fuente.
\usepackage{natbib}
\usepackage{bm}
\usepackage{todonotes}
\usepackage{centernot}
\usetikzlibrary{positioning}
% Plantilla classicthesis
\usepackage[beramono,eulerchapternumbers,linedheaders,parts,a5paper,dottedtoc,
manychapters,pdfspacing]{classicthesis}

% Geometría y espaciado de párrafos.
\setcounter{secnumdepth}{0}
\usepackage{enumitem}
\setitemize{noitemsep,topsep=0pt,parsep=0pt,partopsep=0pt}
\setlist[enumerate]{topsep=0pt,itemsep=-1ex,partopsep=1ex,parsep=1ex}
\usepackage[top=1in, bottom=1.5in, left=1in, right=1in]{geometry}
\setlength\itemsep{0em}
\setlength{\parindent}{0pt}
\usepackage{parskip}

% Profundidad de la tabla de contenidos.
\setcounter{secnumdepth}{3}

% Usa el paquete minted para mostrar trozos de código.
% Pueden seleccionarse el lenguaje apropiado y el estilo del código.
\usepackage{minted}
\usemintedstyle{colorful}
\setminted{fontsize=\small}
\renewcommand{\theFancyVerbLine}{\sffamily\textcolor[rgb]{0.5,0.5,1.0}{\oldstylenums{\arabic{FancyVerbLine}}}}
\newcommand{\bigCI}{\mathrel{\text{\scalebox{1.07}{$\perp\mkern-10mu\perp$}}}}
\newcommand{\bigCD}{\centernot{\bigCI}}
% Archivos de configuración.
%------------------------
% Bibliotecas para matemáticas de latex
%------------------------
\usepackage{amsthm}
\usepackage{amsmath}
\usepackage{tikz}
\usepackage{tikz-cd}
\usetikzlibrary{shapes,fit}
\usepackage{bussproofs}
\EnableBpAbbreviations{}
\usepackage{mathtools}
\usepackage{scalerel}
\usepackage{stmaryrd}

%------------------------
% Estilos para los teoremas
%------------------------
\theoremstyle{plain}
\newtheorem{theorem}{Theorem}
\newtheorem{proposition}{Proposition}
\newtheorem{lemma}{Lemma}
\newtheorem{corollary}{Corollary}
\theoremstyle{definition}
\newtheorem{definition}{Definition}
\newtheorem{proofs}{Proof}
\theoremstyle{remark}
\newtheorem{remark}{Remark}
\newtheorem{exampleth}{Example}

\begingroup\makeatletter\@for\theoremstyle:=definition,remark,plain\do{\expandafter\g@addto@macro\csname th@\theoremstyle\endcsname{\addtolength\thm@preskip\parskip}}\endgroup

%------------------------
% Macros
% ------------------------

% Aquí pueden añadirse abreviaturas para comandos de latex
% frequentemente usados.
\newcommand*\diff{\mathop{}\!\mathrm{d}}
\newcommand{\R}{\mathbb{R}}
% \newcommand{\E}{\mathbb{E}}
\newcommand{\bmu}{\bm{\mu}}
\newcommand{\bx}{\bm{x}}
\newcommand{\bX}{\bm{X}}
\newcommand{\bz}{\bm{z}}
\newcommand{\bZ}{\bm{Z}}
\newcommand{\bv}{\bm{v}}
\newcommand{\bh}{\bm{h}}
\newcommand{\bSigma}{\bm{\Sigma}}
\newcommand{\bpi}{\bm{\pi}}
\newcommand{\bLambda}{\bm{\Lambda}}
\newcommand{\btheta}{\bm{\theta}}

\newcommand{\V}{\mathcal{V}}
\newcommand{\D}{\mathcal{D}}
\newcommand{\X}{\mathcal{X}}
\newcommand{\I}{\mathcal{I}}

\newcommand\ddfrac[2]{\frac{\displaystyle #1}{\displaystyle #2}}

\newcommand\E[2]{\mathbb{E}_{#1}\Big[#2\Big]}
\newcommand\KL[2]{KL\Big(#1 \bigm| #2\Big)}
\newcommand{\bigCI}{\mathrel{\text{\scalebox{1.07}{$\perp\mkern-10mu\perp$}}}}
\newcommand{\bigCD}{\centernot{\bigCI}}

\DeclareMathOperator*{\argmax}{arg\,max}
\DeclareMathOperator*{\argmin}{arg\,min}
  % En macros.tex se almacenan las opciones y comandos para escribir matemáticas.
% ****************************************************************************************************
% classicthesis-config.tex 
% formerly known as loadpackages.sty, classicthesis-ldpkg.sty, and classicthesis-preamble.sty 
% Use it at the beginning of your ClassicThesis.tex, or as a LaTeX Preamble 
% in your ClassicThesis.{tex,lyx} with % ****************************************************************************************************
% classicthesis-config.tex 
% formerly known as loadpackages.sty, classicthesis-ldpkg.sty, and classicthesis-preamble.sty 
% Use it at the beginning of your ClassicThesis.tex, or as a LaTeX Preamble 
% in your ClassicThesis.{tex,lyx} with % ****************************************************************************************************
% classicthesis-config.tex 
% formerly known as loadpackages.sty, classicthesis-ldpkg.sty, and classicthesis-preamble.sty 
% Use it at the beginning of your ClassicThesis.tex, or as a LaTeX Preamble 
% in your ClassicThesis.{tex,lyx} with \input{classicthesis-config}
% ****************************************************************************************************  
% If you like the classicthesis, then I would appreciate a postcard. 
% My address can be found in the file ClassicThesis.pdf. A collection 
% of the postcards I received so far is available online at 
% http://postcards.miede.de
% ****************************************************************************************************


% ****************************************************************************************************
% 0. Set the encoding of your files. UTF-8 is the only sensible encoding nowadays. If you can't read
% äöüßáéçèê∂åëæƒÏ€ then change the encoding setting in your editor, not the line below. If your editor
% does not support utf8 use another editor!
% ****************************************************************************************************
\PassOptionsToPackage{utf8x}{inputenc}
	\usepackage{inputenc}

% ****************************************************************************************************
% 1. Configure classicthesis for your needs here, e.g., remove "drafting" below 
% in order to deactivate the time-stamp on the pages
% ****************************************************************************************************
\PassOptionsToPackage{eulerchapternumbers,listings,drafting,%
		pdfspacing,%floatperchapter,%linedheaders,%
                subfig,beramono,eulermath,parts,dottedtoc}{classicthesis}                                        
% ********************************************************************
% Available options for classicthesis.sty 
% (see ClassicThesis.pdf for more information):
% drafting
% parts nochapters linedheaders
% eulerchapternumbers beramono eulermath pdfspacing minionprospacing
% tocaligned dottedtoc manychapters
% listings floatperchapter subfig
% ********************************************************************

% ****************************************************************************************************
% 2. Personal data and user ad-hoc commands
% ****************************************************************************************************
\newcommand{\myTitle}{A Classic Thesis Style\xspace}
\newcommand{\mySubtitle}{An Homage to The Elements of Typographic Style\xspace}
\newcommand{\myDegree}{Doktor-Ingenieur (Dr.-Ing.)\xspace}
\newcommand{\myName}{André Miede\xspace}
\newcommand{\myProf}{Put name here\xspace}
\newcommand{\myOtherProf}{Put name here\xspace}
\newcommand{\mySupervisor}{Put name here\xspace}
\newcommand{\myFaculty}{Put data here\xspace}
\newcommand{\myDepartment}{Put data here\xspace}
\newcommand{\myUni}{Put data here\xspace}
\newcommand{\myLocation}{Saarbrücken\xspace}
\newcommand{\myTime}{September 2015\xspace}
%\newcommand{\myVersion}{version 4.2\xspace}

% ********************************************************************
% Setup, finetuning, and useful commands
% ********************************************************************
\newcounter{dummy} % necessary for correct hyperlinks (to index, bib, etc.)
\newlength{\abcd} % for ab..z string length calculation
\providecommand{\mLyX}{L\kern-.1667em\lower.25em\hbox{Y}\kern-.125emX\@}
\newcommand{\ie}{i.\,e.}
\newcommand{\Ie}{I.\,e.}
\newcommand{\eg}{e.\,g.}
\newcommand{\Eg}{E.\,g.} 
% ****************************************************************************************************


% ****************************************************************************************************
% 3. Loading some handy packages
% ****************************************************************************************************
% ******************************************************************** 
% Packages with options that might require adjustments
% ******************************************************************** 
%\PassOptionsToPackage{ngerman,american}{babel}   % change this to your language(s)
% Spanish languages need extra options in order to work with this template
% \PassOptionsToPackage{es-lcroman,spanish}{babel}
\usepackage[main=english]{babel}

%\usepackage{csquotes}
% \PassOptionsToPackage{%
%     %backend=biber, %instead of bibtex
% 	backend=bibtex8,bibencoding=ascii,%
% 	language=auto,%
% 	style=alpha,%
%     %style=authoryear-comp, % Author 1999, 2010
%     %bibstyle=authoryear,dashed=false, % dashed: substitute rep. author with ---
%     sorting=nyt, % name, year, title
%     maxbibnames=10, % default: 3, et al.
%     %backref=true,%
%     natbib=true % natbib compatibility mode (\citep and \citet still work)
% }{biblatex}
%     \usepackage{biblatex}

% \PassOptionsToPackage{fleqn}{amsmath}       % math environments and more by the AMS 
%     \usepackage{amsmath}

% ******************************************************************** 
% General useful packages
% ******************************************************************** 
\PassOptionsToPackage{T1}{fontenc} % T2A for cyrillics
    \usepackage{fontenc}     
\usepackage{textcomp} % fix warning with missing font shapes
\usepackage{scrhack} % fix warnings when using KOMA with listings package          
\usepackage{xspace} % to get the spacing after macros right  
\usepackage{mparhack} % get marginpar right
\usepackage{fixltx2e} % fixes some LaTeX stuff --> since 2015 in the LaTeX kernel (see below)
%\usepackage[latest]{latexrelease} % will be used once available in more distributions (ISSUE #107)
\PassOptionsToPackage{printonlyused,smaller}{acronym} 
    \usepackage{acronym} % nice macros for handling all acronyms in the thesis
    %\renewcommand{\bflabel}[1]{{#1}\hfill} % fix the list of acronyms --> no longer working
    %\renewcommand*{\acsfont}[1]{\textsc{#1}} 
    \renewcommand*{\aclabelfont}[1]{\acsfont{#1}}
% ****************************************************************************************************


% ****************************************************************************************************
% 4. Setup floats: tables, (sub)figures, and captions
% ****************************************************************************************************
\usepackage{tabularx} % better tables
    \setlength{\extrarowheight}{3pt} % increase table row height
\newcommand{\tableheadline}[1]{\multicolumn{1}{c}{\spacedlowsmallcaps{#1}}}
\newcommand{\myfloatalign}{\centering} % to be used with each float for alignment
\usepackage{caption}
% Thanks to cgnieder and Claus Lahiri
% http://tex.stackexchange.com/questions/69349/spacedlowsmallcaps-in-caption-label
% [REMOVED DUE TO OTHER PROBLEMS, SEE ISSUE #82]    
%\DeclareCaptionLabelFormat{smallcaps}{\bothIfFirst{#1}{~}\MakeTextLowercase{\textsc{#2}}}
%\captionsetup{font=small,labelformat=smallcaps} % format=hang,
\captionsetup{font=small} % format=hang,
\usepackage{subfig}  
% ****************************************************************************************************


% ****************************************************************************************************
% 5. Setup code listings
% ****************************************************************************************************
% \usepackage{listings} 
% %\lstset{emph={trueIndex,root},emphstyle=\color{BlueViolet}}%\underbar} % for special keywords
% \lstset{language={Haskell},morekeywords={PassOptionsToPackage,selectlanguage},keywordstyle=\color{RoyalBlue},basicstyle=\small\ttfamily,commentstyle=\color{Green}\ttfamily,stringstyle=\rmfamily,numbers=none,numberstyle=\scriptsize,stepnumber=5,numbersep=8pt,showstringspaces=false,breaklines=true,belowcaptionskip=.75\baselineskip} 
% ****************************************************************************************************             


% ****************************************************************************************************
% 6. PDFLaTeX, hyperreferences and citation backreferences
% ****************************************************************************************************
% ********************************************************************
% Using PDFLaTeX
% ********************************************************************
\PassOptionsToPackage{pdftex,hyperfootnotes=false,pdfpagelabels}{hyperref}
    \usepackage{hyperref}  % backref linktocpage pagebackref
\pdfcompresslevel=9
\pdfadjustspacing=1 
\PassOptionsToPackage{pdftex}{graphicx}
    \usepackage{graphicx} 
 

% ********************************************************************
% Hyperreferences
% ********************************************************************
\hypersetup{%
    %draft, % = no hyperlinking at all (useful in b/w printouts)
    colorlinks=true, linktocpage=true, pdfstartpage=3, pdfstartview=FitV,%
    % uncomment the following line if you want to have black links (e.g., for printing)
    %colorlinks=false, linktocpage=false, pdfstartpage=3, pdfstartview=FitV, pdfborder={0 0 0},%
    breaklinks=true, pdfpagemode=UseNone, pageanchor=true, pdfpagemode=UseOutlines,%
    plainpages=false, bookmarksnumbered, bookmarksopen=true, bookmarksopenlevel=1,%
    hypertexnames=true, pdfhighlight=/O,%nesting=true,%frenchlinks,%
    urlcolor=webbrown, linkcolor=RoyalBlue, citecolor=webgreen, %pagecolor=RoyalBlue,%
    %urlcolor=Black, linkcolor=Black, citecolor=Black, %pagecolor=Black,%
    pdftitle={\myTitle},%
    pdfauthor={\textcopyright\ \myName, \myUni, \myFaculty},%
    pdfsubject={},%
    pdfkeywords={},%
    pdfcreator={pdfLaTeX},%
    pdfproducer={LaTeX with hyperref and classicthesis}%
}   

% ********************************************************************
% Setup autoreferences
% ********************************************************************
% There are some issues regarding autorefnames
% http://www.ureader.de/msg/136221647.aspx
% http://www.tex.ac.uk/cgi-bin/texfaq2html?label=latexwords
% you have to redefine the makros for the 
% language you use, e.g., american, ngerman
% (as chosen when loading babel/AtBeginDocument)
% ********************************************************************
\makeatletter
\@ifpackageloaded{babel}%
    {%
       \addto\extrasamerican{%
			\renewcommand*{\figureautorefname}{Figure}%
			\renewcommand*{\tableautorefname}{Table}%
			\renewcommand*{\partautorefname}{Part}%
			\renewcommand*{\chapterautorefname}{Chapter}%
			\renewcommand*{\sectionautorefname}{Section}%
			\renewcommand*{\subsectionautorefname}{Section}%
			\renewcommand*{\subsubsectionautorefname}{Section}%     
                }%
       \addto\extrasngerman{% 
			\renewcommand*{\paragraphautorefname}{Absatz}%
			\renewcommand*{\subparagraphautorefname}{Unterabsatz}%
			\renewcommand*{\footnoteautorefname}{Fu\"snote}%
			\renewcommand*{\FancyVerbLineautorefname}{Zeile}%
			\renewcommand*{\theoremautorefname}{Theorem}%
			\renewcommand*{\appendixautorefname}{Anhang}%
			\renewcommand*{\equationautorefname}{Gleichung}%        
			\renewcommand*{\itemautorefname}{Punkt}%
                }%  
            % Fix to getting autorefs for subfigures right (thanks to Belinda Vogt for changing the definition)
            \providecommand{\subfigureautorefname}{\figureautorefname}%             
    }{\relax}
\makeatother


% ****************************************************************************************************
% 7. Last calls before the bar closes
% ****************************************************************************************************
% ********************************************************************
% Development Stuff
% ********************************************************************
\listfiles
%\PassOptionsToPackage{l2tabu,orthodox,abort}{nag}
%   \usepackage{nag}
%\PassOptionsToPackage{warning, all}{onlyamsmath}
%   \usepackage{onlyamsmath}

% ********************************************************************
% Last, but not least...
% ********************************************************************
\usepackage{classicthesis} 
% ****************************************************************************************************


% ****************************************************************************************************
% 8. Further adjustments (experimental)
% ****************************************************************************************************
% ********************************************************************
% Changing the text area
% ********************************************************************
\linespread{1.05} % a bit more for Palatino
% \areaset[current]{325pt}{680pt} % 686 (factor 2.2) + 33 head + 42 head \the\footskip
%\setlength{\marginparwidth}{7em}%
%\setlength{\marginparsep}{2em}%

% ********************************************************************
% Using different fonts
% ********************************************************************
%\usepackage[oldstylenums]{kpfonts} % oldstyle notextcomp
%\usepackage[osf]{libertine}
%\usepackage[light,condensed,math]{iwona}
%\renewcommand{\sfdefault}{iwona}
%\usepackage{lmodern} % <-- no osf support :-(
%\usepackage{cfr-lm} % 
%\usepackage[urw-garamond]{mathdesign} <-- no osf support :-(
%\usepackage[default,osfigures]{opensans} % scale=0.95 
%\usepackage[sfdefault]{FiraSans}
% ****************************************************************************************************

% ****************************************************************************************************  
% If you like the classicthesis, then I would appreciate a postcard. 
% My address can be found in the file ClassicThesis.pdf. A collection 
% of the postcards I received so far is available online at 
% http://postcards.miede.de
% ****************************************************************************************************


% ****************************************************************************************************
% 0. Set the encoding of your files. UTF-8 is the only sensible encoding nowadays. If you can't read
% äöüßáéçèê∂åëæƒÏ€ then change the encoding setting in your editor, not the line below. If your editor
% does not support utf8 use another editor!
% ****************************************************************************************************
\PassOptionsToPackage{utf8x}{inputenc}
	\usepackage{inputenc}

% ****************************************************************************************************
% 1. Configure classicthesis for your needs here, e.g., remove "drafting" below 
% in order to deactivate the time-stamp on the pages
% ****************************************************************************************************
\PassOptionsToPackage{eulerchapternumbers,listings,drafting,%
		pdfspacing,%floatperchapter,%linedheaders,%
                subfig,beramono,eulermath,parts,dottedtoc}{classicthesis}                                        
% ********************************************************************
% Available options for classicthesis.sty 
% (see ClassicThesis.pdf for more information):
% drafting
% parts nochapters linedheaders
% eulerchapternumbers beramono eulermath pdfspacing minionprospacing
% tocaligned dottedtoc manychapters
% listings floatperchapter subfig
% ********************************************************************

% ****************************************************************************************************
% 2. Personal data and user ad-hoc commands
% ****************************************************************************************************
\newcommand{\myTitle}{A Classic Thesis Style\xspace}
\newcommand{\mySubtitle}{An Homage to The Elements of Typographic Style\xspace}
\newcommand{\myDegree}{Doktor-Ingenieur (Dr.-Ing.)\xspace}
\newcommand{\myName}{André Miede\xspace}
\newcommand{\myProf}{Put name here\xspace}
\newcommand{\myOtherProf}{Put name here\xspace}
\newcommand{\mySupervisor}{Put name here\xspace}
\newcommand{\myFaculty}{Put data here\xspace}
\newcommand{\myDepartment}{Put data here\xspace}
\newcommand{\myUni}{Put data here\xspace}
\newcommand{\myLocation}{Saarbrücken\xspace}
\newcommand{\myTime}{September 2015\xspace}
%\newcommand{\myVersion}{version 4.2\xspace}

% ********************************************************************
% Setup, finetuning, and useful commands
% ********************************************************************
\newcounter{dummy} % necessary for correct hyperlinks (to index, bib, etc.)
\newlength{\abcd} % for ab..z string length calculation
\providecommand{\mLyX}{L\kern-.1667em\lower.25em\hbox{Y}\kern-.125emX\@}
\newcommand{\ie}{i.\,e.}
\newcommand{\Ie}{I.\,e.}
\newcommand{\eg}{e.\,g.}
\newcommand{\Eg}{E.\,g.} 
% ****************************************************************************************************


% ****************************************************************************************************
% 3. Loading some handy packages
% ****************************************************************************************************
% ******************************************************************** 
% Packages with options that might require adjustments
% ******************************************************************** 
%\PassOptionsToPackage{ngerman,american}{babel}   % change this to your language(s)
% Spanish languages need extra options in order to work with this template
% \PassOptionsToPackage{es-lcroman,spanish}{babel}
\usepackage[main=english]{babel}

%\usepackage{csquotes}
% \PassOptionsToPackage{%
%     %backend=biber, %instead of bibtex
% 	backend=bibtex8,bibencoding=ascii,%
% 	language=auto,%
% 	style=alpha,%
%     %style=authoryear-comp, % Author 1999, 2010
%     %bibstyle=authoryear,dashed=false, % dashed: substitute rep. author with ---
%     sorting=nyt, % name, year, title
%     maxbibnames=10, % default: 3, et al.
%     %backref=true,%
%     natbib=true % natbib compatibility mode (\citep and \citet still work)
% }{biblatex}
%     \usepackage{biblatex}

% \PassOptionsToPackage{fleqn}{amsmath}       % math environments and more by the AMS 
%     \usepackage{amsmath}

% ******************************************************************** 
% General useful packages
% ******************************************************************** 
\PassOptionsToPackage{T1}{fontenc} % T2A for cyrillics
    \usepackage{fontenc}     
\usepackage{textcomp} % fix warning with missing font shapes
\usepackage{scrhack} % fix warnings when using KOMA with listings package          
\usepackage{xspace} % to get the spacing after macros right  
\usepackage{mparhack} % get marginpar right
\usepackage{fixltx2e} % fixes some LaTeX stuff --> since 2015 in the LaTeX kernel (see below)
%\usepackage[latest]{latexrelease} % will be used once available in more distributions (ISSUE #107)
\PassOptionsToPackage{printonlyused,smaller}{acronym} 
    \usepackage{acronym} % nice macros for handling all acronyms in the thesis
    %\renewcommand{\bflabel}[1]{{#1}\hfill} % fix the list of acronyms --> no longer working
    %\renewcommand*{\acsfont}[1]{\textsc{#1}} 
    \renewcommand*{\aclabelfont}[1]{\acsfont{#1}}
% ****************************************************************************************************


% ****************************************************************************************************
% 4. Setup floats: tables, (sub)figures, and captions
% ****************************************************************************************************
\usepackage{tabularx} % better tables
    \setlength{\extrarowheight}{3pt} % increase table row height
\newcommand{\tableheadline}[1]{\multicolumn{1}{c}{\spacedlowsmallcaps{#1}}}
\newcommand{\myfloatalign}{\centering} % to be used with each float for alignment
\usepackage{caption}
% Thanks to cgnieder and Claus Lahiri
% http://tex.stackexchange.com/questions/69349/spacedlowsmallcaps-in-caption-label
% [REMOVED DUE TO OTHER PROBLEMS, SEE ISSUE #82]    
%\DeclareCaptionLabelFormat{smallcaps}{\bothIfFirst{#1}{~}\MakeTextLowercase{\textsc{#2}}}
%\captionsetup{font=small,labelformat=smallcaps} % format=hang,
\captionsetup{font=small} % format=hang,
\usepackage{subfig}  
% ****************************************************************************************************


% ****************************************************************************************************
% 5. Setup code listings
% ****************************************************************************************************
% \usepackage{listings} 
% %\lstset{emph={trueIndex,root},emphstyle=\color{BlueViolet}}%\underbar} % for special keywords
% \lstset{language={Haskell},morekeywords={PassOptionsToPackage,selectlanguage},keywordstyle=\color{RoyalBlue},basicstyle=\small\ttfamily,commentstyle=\color{Green}\ttfamily,stringstyle=\rmfamily,numbers=none,numberstyle=\scriptsize,stepnumber=5,numbersep=8pt,showstringspaces=false,breaklines=true,belowcaptionskip=.75\baselineskip} 
% ****************************************************************************************************             


% ****************************************************************************************************
% 6. PDFLaTeX, hyperreferences and citation backreferences
% ****************************************************************************************************
% ********************************************************************
% Using PDFLaTeX
% ********************************************************************
\PassOptionsToPackage{pdftex,hyperfootnotes=false,pdfpagelabels}{hyperref}
    \usepackage{hyperref}  % backref linktocpage pagebackref
\pdfcompresslevel=9
\pdfadjustspacing=1 
\PassOptionsToPackage{pdftex}{graphicx}
    \usepackage{graphicx} 
 

% ********************************************************************
% Hyperreferences
% ********************************************************************
\hypersetup{%
    %draft, % = no hyperlinking at all (useful in b/w printouts)
    colorlinks=true, linktocpage=true, pdfstartpage=3, pdfstartview=FitV,%
    % uncomment the following line if you want to have black links (e.g., for printing)
    %colorlinks=false, linktocpage=false, pdfstartpage=3, pdfstartview=FitV, pdfborder={0 0 0},%
    breaklinks=true, pdfpagemode=UseNone, pageanchor=true, pdfpagemode=UseOutlines,%
    plainpages=false, bookmarksnumbered, bookmarksopen=true, bookmarksopenlevel=1,%
    hypertexnames=true, pdfhighlight=/O,%nesting=true,%frenchlinks,%
    urlcolor=webbrown, linkcolor=RoyalBlue, citecolor=webgreen, %pagecolor=RoyalBlue,%
    %urlcolor=Black, linkcolor=Black, citecolor=Black, %pagecolor=Black,%
    pdftitle={\myTitle},%
    pdfauthor={\textcopyright\ \myName, \myUni, \myFaculty},%
    pdfsubject={},%
    pdfkeywords={},%
    pdfcreator={pdfLaTeX},%
    pdfproducer={LaTeX with hyperref and classicthesis}%
}   

% ********************************************************************
% Setup autoreferences
% ********************************************************************
% There are some issues regarding autorefnames
% http://www.ureader.de/msg/136221647.aspx
% http://www.tex.ac.uk/cgi-bin/texfaq2html?label=latexwords
% you have to redefine the makros for the 
% language you use, e.g., american, ngerman
% (as chosen when loading babel/AtBeginDocument)
% ********************************************************************
\makeatletter
\@ifpackageloaded{babel}%
    {%
       \addto\extrasamerican{%
			\renewcommand*{\figureautorefname}{Figure}%
			\renewcommand*{\tableautorefname}{Table}%
			\renewcommand*{\partautorefname}{Part}%
			\renewcommand*{\chapterautorefname}{Chapter}%
			\renewcommand*{\sectionautorefname}{Section}%
			\renewcommand*{\subsectionautorefname}{Section}%
			\renewcommand*{\subsubsectionautorefname}{Section}%     
                }%
       \addto\extrasngerman{% 
			\renewcommand*{\paragraphautorefname}{Absatz}%
			\renewcommand*{\subparagraphautorefname}{Unterabsatz}%
			\renewcommand*{\footnoteautorefname}{Fu\"snote}%
			\renewcommand*{\FancyVerbLineautorefname}{Zeile}%
			\renewcommand*{\theoremautorefname}{Theorem}%
			\renewcommand*{\appendixautorefname}{Anhang}%
			\renewcommand*{\equationautorefname}{Gleichung}%        
			\renewcommand*{\itemautorefname}{Punkt}%
                }%  
            % Fix to getting autorefs for subfigures right (thanks to Belinda Vogt for changing the definition)
            \providecommand{\subfigureautorefname}{\figureautorefname}%             
    }{\relax}
\makeatother


% ****************************************************************************************************
% 7. Last calls before the bar closes
% ****************************************************************************************************
% ********************************************************************
% Development Stuff
% ********************************************************************
\listfiles
%\PassOptionsToPackage{l2tabu,orthodox,abort}{nag}
%   \usepackage{nag}
%\PassOptionsToPackage{warning, all}{onlyamsmath}
%   \usepackage{onlyamsmath}

% ********************************************************************
% Last, but not least...
% ********************************************************************
\usepackage{classicthesis} 
% ****************************************************************************************************


% ****************************************************************************************************
% 8. Further adjustments (experimental)
% ****************************************************************************************************
% ********************************************************************
% Changing the text area
% ********************************************************************
\linespread{1.05} % a bit more for Palatino
% \areaset[current]{325pt}{680pt} % 686 (factor 2.2) + 33 head + 42 head \the\footskip
%\setlength{\marginparwidth}{7em}%
%\setlength{\marginparsep}{2em}%

% ********************************************************************
% Using different fonts
% ********************************************************************
%\usepackage[oldstylenums]{kpfonts} % oldstyle notextcomp
%\usepackage[osf]{libertine}
%\usepackage[light,condensed,math]{iwona}
%\renewcommand{\sfdefault}{iwona}
%\usepackage{lmodern} % <-- no osf support :-(
%\usepackage{cfr-lm} % 
%\usepackage[urw-garamond]{mathdesign} <-- no osf support :-(
%\usepackage[default,osfigures]{opensans} % scale=0.95 
%\usepackage[sfdefault]{FiraSans}
% ****************************************************************************************************

% ****************************************************************************************************  
% If you like the classicthesis, then I would appreciate a postcard. 
% My address can be found in the file ClassicThesis.pdf. A collection 
% of the postcards I received so far is available online at 
% http://postcards.miede.de
% ****************************************************************************************************


% ****************************************************************************************************
% 0. Set the encoding of your files. UTF-8 is the only sensible encoding nowadays. If you can't read
% äöüßáéçèê∂åëæƒÏ€ then change the encoding setting in your editor, not the line below. If your editor
% does not support utf8 use another editor!
% ****************************************************************************************************
\PassOptionsToPackage{utf8x}{inputenc}
	\usepackage{inputenc}

% ****************************************************************************************************
% 1. Configure classicthesis for your needs here, e.g., remove "drafting" below 
% in order to deactivate the time-stamp on the pages
% ****************************************************************************************************
\PassOptionsToPackage{eulerchapternumbers,listings,drafting,%
		pdfspacing,%floatperchapter,%linedheaders,%
                subfig,beramono,eulermath,parts,dottedtoc}{classicthesis}                                        
% ********************************************************************
% Available options for classicthesis.sty 
% (see ClassicThesis.pdf for more information):
% drafting
% parts nochapters linedheaders
% eulerchapternumbers beramono eulermath pdfspacing minionprospacing
% tocaligned dottedtoc manychapters
% listings floatperchapter subfig
% ********************************************************************

% ****************************************************************************************************
% 2. Personal data and user ad-hoc commands
% ****************************************************************************************************
\newcommand{\myTitle}{A Classic Thesis Style\xspace}
\newcommand{\mySubtitle}{An Homage to The Elements of Typographic Style\xspace}
\newcommand{\myDegree}{Doktor-Ingenieur (Dr.-Ing.)\xspace}
\newcommand{\myName}{André Miede\xspace}
\newcommand{\myProf}{Put name here\xspace}
\newcommand{\myOtherProf}{Put name here\xspace}
\newcommand{\mySupervisor}{Put name here\xspace}
\newcommand{\myFaculty}{Put data here\xspace}
\newcommand{\myDepartment}{Put data here\xspace}
\newcommand{\myUni}{Put data here\xspace}
\newcommand{\myLocation}{Saarbrücken\xspace}
\newcommand{\myTime}{September 2015\xspace}
%\newcommand{\myVersion}{version 4.2\xspace}

% ********************************************************************
% Setup, finetuning, and useful commands
% ********************************************************************
\newcounter{dummy} % necessary for correct hyperlinks (to index, bib, etc.)
\newlength{\abcd} % for ab..z string length calculation
\providecommand{\mLyX}{L\kern-.1667em\lower.25em\hbox{Y}\kern-.125emX\@}
\newcommand{\ie}{i.\,e.}
\newcommand{\Ie}{I.\,e.}
\newcommand{\eg}{e.\,g.}
\newcommand{\Eg}{E.\,g.} 
% ****************************************************************************************************


% ****************************************************************************************************
% 3. Loading some handy packages
% ****************************************************************************************************
% ******************************************************************** 
% Packages with options that might require adjustments
% ******************************************************************** 
%\PassOptionsToPackage{ngerman,american}{babel}   % change this to your language(s)
% Spanish languages need extra options in order to work with this template
% \PassOptionsToPackage{es-lcroman,spanish}{babel}
\usepackage[main=english]{babel}

%\usepackage{csquotes}
% \PassOptionsToPackage{%
%     %backend=biber, %instead of bibtex
% 	backend=bibtex8,bibencoding=ascii,%
% 	language=auto,%
% 	style=alpha,%
%     %style=authoryear-comp, % Author 1999, 2010
%     %bibstyle=authoryear,dashed=false, % dashed: substitute rep. author with ---
%     sorting=nyt, % name, year, title
%     maxbibnames=10, % default: 3, et al.
%     %backref=true,%
%     natbib=true % natbib compatibility mode (\citep and \citet still work)
% }{biblatex}
%     \usepackage{biblatex}

% \PassOptionsToPackage{fleqn}{amsmath}       % math environments and more by the AMS 
%     \usepackage{amsmath}

% ******************************************************************** 
% General useful packages
% ******************************************************************** 
\PassOptionsToPackage{T1}{fontenc} % T2A for cyrillics
    \usepackage{fontenc}     
\usepackage{textcomp} % fix warning with missing font shapes
\usepackage{scrhack} % fix warnings when using KOMA with listings package          
\usepackage{xspace} % to get the spacing after macros right  
\usepackage{mparhack} % get marginpar right
\usepackage{fixltx2e} % fixes some LaTeX stuff --> since 2015 in the LaTeX kernel (see below)
%\usepackage[latest]{latexrelease} % will be used once available in more distributions (ISSUE #107)
\PassOptionsToPackage{printonlyused,smaller}{acronym} 
    \usepackage{acronym} % nice macros for handling all acronyms in the thesis
    %\renewcommand{\bflabel}[1]{{#1}\hfill} % fix the list of acronyms --> no longer working
    %\renewcommand*{\acsfont}[1]{\textsc{#1}} 
    \renewcommand*{\aclabelfont}[1]{\acsfont{#1}}
% ****************************************************************************************************


% ****************************************************************************************************
% 4. Setup floats: tables, (sub)figures, and captions
% ****************************************************************************************************
\usepackage{tabularx} % better tables
    \setlength{\extrarowheight}{3pt} % increase table row height
\newcommand{\tableheadline}[1]{\multicolumn{1}{c}{\spacedlowsmallcaps{#1}}}
\newcommand{\myfloatalign}{\centering} % to be used with each float for alignment
\usepackage{caption}
% Thanks to cgnieder and Claus Lahiri
% http://tex.stackexchange.com/questions/69349/spacedlowsmallcaps-in-caption-label
% [REMOVED DUE TO OTHER PROBLEMS, SEE ISSUE #82]    
%\DeclareCaptionLabelFormat{smallcaps}{\bothIfFirst{#1}{~}\MakeTextLowercase{\textsc{#2}}}
%\captionsetup{font=small,labelformat=smallcaps} % format=hang,
\captionsetup{font=small} % format=hang,
\usepackage{subfig}  
% ****************************************************************************************************


% ****************************************************************************************************
% 5. Setup code listings
% ****************************************************************************************************
% \usepackage{listings} 
% %\lstset{emph={trueIndex,root},emphstyle=\color{BlueViolet}}%\underbar} % for special keywords
% \lstset{language={Haskell},morekeywords={PassOptionsToPackage,selectlanguage},keywordstyle=\color{RoyalBlue},basicstyle=\small\ttfamily,commentstyle=\color{Green}\ttfamily,stringstyle=\rmfamily,numbers=none,numberstyle=\scriptsize,stepnumber=5,numbersep=8pt,showstringspaces=false,breaklines=true,belowcaptionskip=.75\baselineskip} 
% ****************************************************************************************************             


% ****************************************************************************************************
% 6. PDFLaTeX, hyperreferences and citation backreferences
% ****************************************************************************************************
% ********************************************************************
% Using PDFLaTeX
% ********************************************************************
\PassOptionsToPackage{pdftex,hyperfootnotes=false,pdfpagelabels}{hyperref}
    \usepackage{hyperref}  % backref linktocpage pagebackref
\pdfcompresslevel=9
\pdfadjustspacing=1 
\PassOptionsToPackage{pdftex}{graphicx}
    \usepackage{graphicx} 
 

% ********************************************************************
% Hyperreferences
% ********************************************************************
\hypersetup{%
    %draft, % = no hyperlinking at all (useful in b/w printouts)
    colorlinks=true, linktocpage=true, pdfstartpage=3, pdfstartview=FitV,%
    % uncomment the following line if you want to have black links (e.g., for printing)
    %colorlinks=false, linktocpage=false, pdfstartpage=3, pdfstartview=FitV, pdfborder={0 0 0},%
    breaklinks=true, pdfpagemode=UseNone, pageanchor=true, pdfpagemode=UseOutlines,%
    plainpages=false, bookmarksnumbered, bookmarksopen=true, bookmarksopenlevel=1,%
    hypertexnames=true, pdfhighlight=/O,%nesting=true,%frenchlinks,%
    urlcolor=webbrown, linkcolor=RoyalBlue, citecolor=webgreen, %pagecolor=RoyalBlue,%
    %urlcolor=Black, linkcolor=Black, citecolor=Black, %pagecolor=Black,%
    pdftitle={\myTitle},%
    pdfauthor={\textcopyright\ \myName, \myUni, \myFaculty},%
    pdfsubject={},%
    pdfkeywords={},%
    pdfcreator={pdfLaTeX},%
    pdfproducer={LaTeX with hyperref and classicthesis}%
}   

% ********************************************************************
% Setup autoreferences
% ********************************************************************
% There are some issues regarding autorefnames
% http://www.ureader.de/msg/136221647.aspx
% http://www.tex.ac.uk/cgi-bin/texfaq2html?label=latexwords
% you have to redefine the makros for the 
% language you use, e.g., american, ngerman
% (as chosen when loading babel/AtBeginDocument)
% ********************************************************************
\makeatletter
\@ifpackageloaded{babel}%
    {%
       \addto\extrasamerican{%
			\renewcommand*{\figureautorefname}{Figure}%
			\renewcommand*{\tableautorefname}{Table}%
			\renewcommand*{\partautorefname}{Part}%
			\renewcommand*{\chapterautorefname}{Chapter}%
			\renewcommand*{\sectionautorefname}{Section}%
			\renewcommand*{\subsectionautorefname}{Section}%
			\renewcommand*{\subsubsectionautorefname}{Section}%     
                }%
       \addto\extrasngerman{% 
			\renewcommand*{\paragraphautorefname}{Absatz}%
			\renewcommand*{\subparagraphautorefname}{Unterabsatz}%
			\renewcommand*{\footnoteautorefname}{Fu\"snote}%
			\renewcommand*{\FancyVerbLineautorefname}{Zeile}%
			\renewcommand*{\theoremautorefname}{Theorem}%
			\renewcommand*{\appendixautorefname}{Anhang}%
			\renewcommand*{\equationautorefname}{Gleichung}%        
			\renewcommand*{\itemautorefname}{Punkt}%
                }%  
            % Fix to getting autorefs for subfigures right (thanks to Belinda Vogt for changing the definition)
            \providecommand{\subfigureautorefname}{\figureautorefname}%             
    }{\relax}
\makeatother


% ****************************************************************************************************
% 7. Last calls before the bar closes
% ****************************************************************************************************
% ********************************************************************
% Development Stuff
% ********************************************************************
\listfiles
%\PassOptionsToPackage{l2tabu,orthodox,abort}{nag}
%   \usepackage{nag}
%\PassOptionsToPackage{warning, all}{onlyamsmath}
%   \usepackage{onlyamsmath}

% ********************************************************************
% Last, but not least...
% ********************************************************************
\usepackage{classicthesis} 
% ****************************************************************************************************


% ****************************************************************************************************
% 8. Further adjustments (experimental)
% ****************************************************************************************************
% ********************************************************************
% Changing the text area
% ********************************************************************
\linespread{1.05} % a bit more for Palatino
% \areaset[current]{325pt}{680pt} % 686 (factor 2.2) + 33 head + 42 head \the\footskip
%\setlength{\marginparwidth}{7em}%
%\setlength{\marginparsep}{2em}%

% ********************************************************************
% Using different fonts
% ********************************************************************
%\usepackage[oldstylenums]{kpfonts} % oldstyle notextcomp
%\usepackage[osf]{libertine}
%\usepackage[light,condensed,math]{iwona}
%\renewcommand{\sfdefault}{iwona}
%\usepackage{lmodern} % <-- no osf support :-(
%\usepackage{cfr-lm} % 
%\usepackage[urw-garamond]{mathdesign} <-- no osf support :-(
%\usepackage[default,osfigures]{opensans} % scale=0.95 
%\usepackage[sfdefault]{FiraSans}
% ****************************************************************************************************
 % En classicthesis-config.tex se almacenan las opciones propias de la plantilla.

% Color institucional UGR
% \definecolor{ugrColor}{HTML}{ed1c3e} % Versión clara.
\definecolor{ugrColor}{HTML}{c6474b}  % Usado en el título.
\definecolor{ugrColor2}{HTML}{c6474b} % Usado en las secciones.

% Datos de portada
\usepackage{titling} % Facilita los datos de la portada
\author{Luis Antonio Ortega Andrés}
\date{\today}
\title{Statistical Models with Variational Methods}

% Portada
\usepackage{datetime}
\renewcommand\maketitle{
  \begin{titlepage}
    \begin{addmargin}[-2.5cm]{-3cm}
      \begin{center}
        \large  
        \hfill
        \vfill

        \begingroup
        \color{ugrColor}\spacedallcaps{\thetitle} \\ \bigskip
        \endgroup

        \spacedlowsmallcaps{\theauthor}

        \vfill

        Bachelor's Thesis \\ \medskip
        Computer Science and Mathematics \\  \bigskip\bigskip


        \textbf{Tutor}\\
        Serafín Moral Callejón \\ \bigskip

        \spacedlowsmallcaps{Faculty of Science} \\
        \spacedlowsmallcaps{H.T.S. of Computer Engineer and Telecommunications} \\ \medskip
        
        \textit{Granada, \today}

        \vfill                      

      \end{center}  
    \end{addmargin}       
  \end{titlepage}}

\usepackage{wallpaper}

\begin{document}
\ThisULCornerWallPaper{1}{ugrA4.pdf}
\maketitle

\chapter*{Abstract}

Some introduction about how important Variational methods are nowadays and what this project is about.

\tableofcontents

\ctparttext{
  \color{black}
  \begin{center}
    In this chapter we will introduce the underlying concepts of probability and
    graph theory that we will need.
  \end{center}
}
\part{Basic Concepts}

\chapter{Probability}


All our theory will be made under the assumption that there is a
\emph{referential set} \(\Omega\), set of all possible outcomes of an experiment. Any subset of
\(\Omega\) will be called an \emph{event}.

\begin{definition}
Let \(\mathcal{P}(\Omega)\) be the power set of \(\Omega\). Then, \(\mathcal{F} \subset \mathcal{P}(\Omega)\) is called a
\emph{\(\sigma\)-algebra} if it satisfies three conditions:
\begin{itemize}
\item It contains the full referential set, \(\Omega \in \mathcal{F}\).
\item \(\mathcal{F}\) is closed under complementation.
\item \(\mathcal{F}\) is closed under countable unions.
\end{itemize}
From these properties it follows that \(\emptyset \in \mathcal{F}\) and that \(\mathcal{F}\)
is closed under countable intersections.

The tuple \((\Omega, \mathcal{F})\) is called a \emph{measurable space}.
\end{definition}

\begin{definition}
A \emph{probability} \(P\) over a measurable space \((\Omega, \mathcal{F})\) is a mapping
\(P: \mathcal{F} \to [0,1]\) which satisfies:
\begin{itemize}
\item \(P(\alpha) \geq 0 \ \ \forall \alpha \in \mathcal{F}\).
\item \(P(\Omega) = 1\).
\item \(P\) is countably additive, that is, if \(\{\alpha_n\}_{n \in \mathbb{N}}
  \subset \mathcal{F}\) is a countable collection of pairwise disjoint sets,
  then
  \[
  P\Big(\bigcup_{n\in \mathbb{N}}\alpha_n \Big) = \sum_{n \in \mathbb{N}}P(\alpha_n).
  \]
\end{itemize}
\end{definition}

The first condition guarantees non negativity. The second one states that the
\emph{trivial event} has the maximal possible probability of 1.
The third condition implies that given a set of pairwise disjoint events,
the probability of either one of them occurring is equal to the sum of the
probabilities of each one.

These conditions imply the following two:
\begin{itemize}
\item \(P(\emptyset) = 0\).
\item \(P(\alpha \cup \beta) = P(\alpha) + P(\beta) - P(\alpha \cap \beta)\).
\end{itemize}

The triple \((\Omega, \mathcal{F}, P)\) is called a \emph{probability space}.

\begin{definition}
  Given two events \(\alpha, \beta \in \mathcal{F}\), with \(P(\beta) \neq 0\),
  the conditional probability of \(\alpha\) given \(\beta\) is defined as the
  quotient of the probability of the joint events and the probability of
  \(\beta\):
  \[
    P(\alpha \mid \beta) = \frac{P(\alpha \cap \beta)}{P(\beta)}.
  \]
\end{definition}

\begin{theorem}
  \textbf{(Bayes' theorem)}. Let \(\alpha, \beta\) be two events of an
  experiment, given that \(P(\beta) \neq 0\). Then
  \[
  P(\alpha \mid \beta)= \frac{P(\beta \mid \alpha)P(\alpha)}{P(\beta)}.
\]
\end{theorem}

\begin{exampleth}
Consider a study where the relation of a disease \(d\) and an habit \(h\)
is being investigated. Suppose that \(P(d)=10^{-5}\), \(P(h)=0.5\) and \(P(h\mid d) = 0.9\). What is the
probability that a person with habit \(h\) will have the disease \(d\)?

\[
P(d \mid h) = \frac{P(d \cap h)}{P(h)} = \frac{P(h \mid d)P(d)}{P(h)} =
\frac{ 0.9 \times 10^{-5}}{ 0.5 } = 1.8 \times 10^{-5}.
\]

If the probability of having habit \(h\) is set to a much lower value as \(P(h) =
0.001\), then the above calculation gives approximately \(1/100\). Intuitively, a smaller number of people have the habit and most of them have the
desease. This means that the relation between having the desease and the habit
is stronger now compared with the case where more people had the habit.
\end{exampleth}

\begin{definition}
  Two events \(\alpha, \beta \in \mathcal{F}\) are said
  \emph{independent} if knowing one of them does not give any extra information
  about the other. Mathematically,
  \[
    P(\alpha \cap \beta) = P(\alpha)P(\beta) \quad \text{and} \quad P(\alpha \mid \beta) = P(\alpha).
  \]
  Let \(\alpha, \beta, \gamma \in \mathcal{F}\), \(\alpha\) and \(\beta\) are said to be
  \emph{conditionally independent} on \(\gamma\), expressed as \(\alpha \bigCI \beta \mid \gamma\),  if and only if
  \[
    P(\alpha \cup \beta \mid \gamma) = P(\alpha \mid \gamma)P(\beta \mid \gamma).
  \]
  Otherwise, they are said to be \emph{conditionally dependent} on \(\gamma\), expressed as \(\alpha \bigCD \beta \mid \gamma\).
\end{definition}

\section*{Random variables}

\begin{definition}
A function \(f:\Omega_1 \to \Omega_2\) between two
measurable spaces \((\Omega_1, \mathcal{F}_1)\) and \((\Omega_2, \mathcal{F}_2)\) is said to be \emph{measurable} if \(f^{-1}(\alpha) \in \mathcal{F}_1\) for every \(\alpha \in \mathcal{F}_2\).
\end{definition}

\begin{definition}
  A \emph{random variable} is a measurable function \(X:\Omega \to E\) from a probability
  space \((\Omega, \mathcal{F}, P)\) to a measurable space \((E,
  \mathcal{F}')\) verifying \(X(\omega)\in \mathcal{F}' \ \forall \omega \in \Omega\).

  The probability of \(X\) taking a value on a measurable set \(S \in E\) is
  written as
  \[
    P_X(S) = P(X \in S) = P(\{a \in \Omega \ \mid  \ X(a) \in S \}).
  \]
  Where the sub-index is usually omitted. A \emph{probability distribution} \(P_{X}\) of a random
  variable \(X\) over the probability space \((\Omega, \mathcal{F}, P)\) is defined
  as the push-forward measure of it, that is, \(P_X = P \circ X^{-1}\).
\end{definition}

Questions like ``How likely is that the value of \(X\) equals
\(a\)?'' are equivalent to ask for the probability (measure) of the set \(\{\omega
\in \Omega \ \mid  \ X(w) = a\}\).

The following notation is going to be used: random variables will be
denoted with an upper case letter like \(X\) and a set of variables with a
bold symbol like \(\bm{X}\). The meaning of \(P(state)\) will be clear without a reference to the variable.
Otherwise \(P(X = state)\) will be used.
Using a lower case letter like \(P(x)\) will denote the probability of the
corresponding upper case variable \(X\) taking a specific value \(x\).

\begin{definition}
The \emph{cumulative distribution function} of a real-valued random variable \(X\) is defined as
\[
F_X (x) = P(X \leq x),
\]
where the right-hand side represents the probability of the random variable
taking value below or equal to \(x\).
\end{definition}

\begin{definition}
When the image of a random variable \(X\) is countable, the random variable
is called a
\emph{discrete random variable} and its \emph{probability mass function} \(p\) gives the
probability of it being equal to some value:
\[
p(x) = P(X = x).
\]
In case the image is uncountable and real, then \(X\) is called a \emph{continuous random
  variable} and if there exists is a non-negative
Lebesgue-integrable \(f\) such that
\[
F_X(x) = P(X \leq x) = \int_{-\infty}^x f(u) du,
\]
then it is called its \emph{probability density function}.

A \emph{mixed random variable} is a random variable who is neither discrete nor
continuous, it can be realized as the sum of a discrete and continuous random
variables. An example of a random variable of mixed type would be based on an
experiment where a coin is flipped and a random positive number is chose only if
the result of the coin toss is heads, $-1$ otherwise.
\end{definition}

From now on, \(P(x)\) will denote \(f(x)\) when \(X\) is a continuous random
variable.

Integrate notation will be used in both continuous and discrete cases, where the last one can be interpreted as integration with respect to the \emph{counting measure} defined as
\[
 \#(dx) = \sum_{n \in \I}\delta(x - n)dx,
\]
where \(\I\) is the set of values \(X\) can take, and \(\delta\) is the Dirac measure. Given this measure, integration corresponds to summation as
\[
  \int_{x} P(x) \#(dx) = \sum_{n \in \I}\int_{x} P(x) \delta(x-n) dx = \sum_{n \in \I}P(n).
\]

Where \(\int_{x} f(x)\delta(x - x_{0}) = f(x_{0})\) is used. Given this, from now on, integration notation will be used for both discrete and continuous variables given that the integrals will be respect to the counting measure when needed.

\begin{definition}
  The \emph{conditional probability} might be defined over
  random variables, let \(X, Y\) be two random variables, then
  \[
    P(x \mid y) = \frac{P(x,y)}{P(y)}.
  \]
  It is  required that \(P(y) \neq 0\) for the conditional probability to be defined.
\end{definition}

The \emph{Bayes' theorem} may be enunciated as
\[
  P(x,y) = \frac{P(y\mid x)P(x)}{P(y)}
\]
Clearly, an arbitrary number of variables can be considered in both cases.

\begin{definition}
  The \emph{marginal distribution} of a subset of random variables is the
  probability distribution of the variables contained in that subset.
\end{definition}

Let \(X, Y\) be two random variables, the marginal distribution of \(X\) is:
\[
  P(x) = \int_y P(x,y).
\]


\begin{definition}
  Let \(\bm{X} = (X_1, X_2,\dots,X_n)\) be a set of random variables, the
  \emph{joint probability distribution} for \(\bm{X}\) is function that gives the probability of each random variable \(X_n\)  falling in a particular range or discrete set of values for that variable. It is called a \emph{multi-variate distribution}.

  When using only two random variables, then is called a \emph{bi-variate
    distribution}.

  This distribution can be expressed either in terms of a joint cumulative distribution
  function
  \[
    F_{\bm{X}}(\bm{x}) = F_{X_1,\dots,X_n}(x_1,\dots,x_n) = P(X_1 \leq x_1, \dots,
    X_n \leq x_n) \footnote{Where \(\bm{x} = (x_1,\dots,x_n)\)},
  \]
  or using a probability density or mass function.
\end{definition}

\begin{definition}
Two random variables \(X\) and \(Y\) are said to be \emph{independent} if knowing one of them doesn't give any extra information about the other. Mathematically,
\[
P(x,y) = P(x)P(y).
\]
From this it follows that if \(X\) and \(Y\) are independent, then \(P(x\mid y) = P(x)\).
\end{definition}


\begin{definition}
Let \(X,Y\) and \(Z\) be three random variables, then \(X\) and \(Y\) are
\emph{conditionally independent} given \(Z\) if and only if
\[
P(x,y \mid  z) = P(x\mid z)P(y\mid z),
\]
in that case we will denote \(X \bigCI Y \mid Z\). If \(X\) and \(Y\) are not
conditionally independent, they are \emph{conditionally dependent} \(X \bigCD Y \mid Z\)

\end{definition}

Both independence definitions can be made over sets of variables \(\bm{X},
\bm{Y}\) and \(\bm{Z}\) in a straightforward way.


\begin{definition}
  A set of \(N\) random variables \(\{X_1,\dots,X_N\}\) defined to
  assume values in \(I \subset \R\) are said
  \emph{independent and identically distributed (i.i.d)}
  if and only if they are independent, i.e,
  \[
    F_{X_1,\dots,X_N}(x_1,\dots,x_n) = F_{X_1}(x_1)\dots F_{X_N}(x_N) \ \forall
    x_1,\dots,x_N \in I,
  \]
  and are identically distributed
  \[
    F_{X_1}(x) = F_{X_n}(x) \ \forall n \in \{2,\dots,N\} \text{ and } \forall x
    \in I.
  \]
\end{definition}


\begin{definition}
  A \emph{multi-variate random variable} or \emph{random vector} is a column vector \(\bm{X} =
  {(X_1,\dots,X_N)}^T\) whose components are random variables that can be defined
  over different probability spaces.

  Note that the same symbol \(\bm{X}\) is used for random vectors and sets of
  variables, but the meaning will be clear within the context.
\end{definition}


\chapter{Graph Theory}


\begin{definition}
A \emph{graph} \(G = (V,E)\) is a set of vertices or nodes \(V\) and edges \(E\subset
V\times V\) between them.
If \(V\) is a set of ordered pairs then the graph is called a \emph{directed
  graph}, otherwise if \(V\) is a set of unordered pairs it is called an \emph{undirected graph}.
\end{definition}

\begin{figure*}[h]
\centering
\begin{tikzpicture}[
  node distance=1cm and 0.5cm,
  mynode/.style={draw,circle,text width=0.5cm,align=center}
]

\node[mynode] (a) {A};
\node[mynode,below right=of a] (b) {B};
\node[mynode,above right=of b] (c) {C};

\node[mynode, right=of c] (d) {A};
\node[mynode,below right=of d] (e) {B};
\node[mynode,above right=of e] (f) {C};

\path (c) edge[-latex] (a)
(a) edge[-latex] (b)
(b) edge[latex-] (c);

\draw (d) -- (e) -- (f) -- (d);

\end{tikzpicture}
\caption{Example of directed and undirected graph, respectively.}
\label{fig:graphs}
\end{figure*}

\begin{definition}
In a directed graph \(G = (V, E)\), a \emph{directed path} \(A \to B\) is a sequence of vertices \({A = A_0,
  A_1,\dots,A_{n-1}, A_n = B}\) where \((A_i, A_{i+1}) \in E \ \forall i \in
0,\dots ,n-1\).

If \(G\) is a undirected graph, \(A \to B\) is an \emph{undirected path} if \(\{A_i, A_{i+1}\} \in E \ \forall i \in
0,\dots, n-1\)
\end{definition}

\begin{definition}
Let \(A,B\) be two vertices of a directed graph \(G\). If \(A \to B\) is a
directed path and \(B \not \to A\) (meaning there isn't a directed path from
\(B\) to \(A\)), then \(A\) is called an \emph{ancestor} of \(B\) and \(B\) is called a \emph{descendant} of \(A\).
\end{definition}

For example, in the figure \ref{fig:graphs}, \(C\) is an ancestor of \(B\).

\begin{definition}
A \emph{directed acyclic graph (DAG)} is a directed graph such that no directed path between any two nodes revisits a vertex.
\end{definition}


\begin{figure}[h]
\centering
\begin{tikzpicture}[
  node distance=1cm and 0.5cm,
  mynode/.style={draw,circle,text width=0.5cm,align=center}
]

\node[mynode] (a) {A};
\node[mynode,below right=of a] (b) {B};
\node[mynode,above right=of b] (c) {C};

\path (c) edge[-latex] (a)
(a) edge[-latex] (b)
(b) edge[-latex] (c);

\end{tikzpicture}
\captionof{figure}{Example of graph which isn't a DAG.}
\label{fig:not_dag}
\end{figure}

As we can see in the figure \ref{fig:not_dag}, \(A \to B \to C \to A \to B\) is a
path from \(A\) to \(B\) that revisits \(A\).

Now where are going to define some relations between nodes in a DAG.

\begin{definition}
The \emph{parents} of a node \(A\) is the set of nodes \(B\) such that there is a
directed edge from \(B\) to \(A\). The same applies for the \emph{children} of a node.

The \emph{Markov blanket} of a node is composed by the node itself, its children, its parents and the parents
of its children.
\end{definition}


\begin{figure}[h]
\centering
\begin{tikzpicture}[
  node distance=1cm and 0.5cm,
  mynode/.style={draw,circle,text width=0.5cm,align=center}
]

\node[mynode] (a) {A};
\node[mynode,below right=of a] (b) {B};
\node[mynode,above right=of b] (c) {C};
\node[mynode,below right=of b] (d) {D};
\node[mynode,below left=of b] (e) {E};
\node[mynode,above right=of d] (f) {F};
\node[mynode, above right=of f] (h) {H};

\path (c) edge[-latex] (a)
(a) edge[-latex] (b)
(b) edge[latex-] (c)
(b) edge[-latex] (e)
(c) edge[-latex] (f)
(b) edge[-latex] (d)
(f) edge[-latex] (d)
(h) edge[-latex] (f)
;

\end{tikzpicture}
\captionof{figure}{Directed acyclic graph}
\label{fig:relations}
\end{figure}

\begin{definition}
In a graph, the \emph{neighbors} of a node are those directly connected
to it.
\end{definition}

We can use figure \ref{fig:relations} to reflect on these definitions. The parents
of \(B\) are \(pa(B) = \{A,C\}\) and its children are \(ch(B) = \{E,D\}\). Taking this into account, its neighbors
are \(ne(B) = \{A,C,E,D\}\) and its Markov blanket is \(\{A,B,C,D,E,F\}\).

\begin{definition}
Let \(G\) be an undirected graph, a \emph{clique} is a maximally connected
subset of vertices. That is, all the members of the clique are connected to each
others and there is no bigger clique that constains another.

Formally, \(S \subset V\) is a \emph{clique} if and only if \(\forall A,B \in S,
\ \{A,B\} \in E\) and \(\nexists C \in V\backslash S\) such that \(\forall A \in
S, \ \{A, C\} \in E \).
\end{definition}


\ctparttext{
  \color{black}
  \begin{center}

A \emph{graphical model} is a statistical model for which a graph expresses the
conditional dependence structure between random variables.

Commonly, they provide a graph-based representation for encoding a multi-dimensional
distribution representing a set of independences that hold in the specific
distribution. The most commonly used are \emph{Bayesian networks} and \emph{Markov random
fields}, which differ in the set of independences they can encode and the
factorization of the distribution that they include.  \end{center}
}
\part{Graphical Models}

\chapter{Bayesian networks}

Consider we have \(N\) variables with the corresponding distribution
\(P(x_1,\dots,x_N)\). Let \(\mathcal{E}\) be a set of indexes such as \texttt{evidence}
\(=\{X_e = x_e \ | \ e \in \mathcal{E}\}\). Inference could be made by brute
force:

\[
P(X_i = x_i \ | \ \texttt{evidence}) = \frac{ \int_{ j \not \in
\mathcal{E}, j \neq i } P(\texttt{evidence}, x_j, X_i = x_i)}{ \int_{ j
\not \in \mathcal{E} } P(\texttt{evidence}, x_j)}
\]

The notation when using discrete variables is analogous replacing integration
with summations.

Lets suppose all these variables are binary, this calculation will require
\(O(2^{N-\#\mathcal{E}})\) operations. Also, all entries of a table \(P(x_1,\dots,
x_N)\) take \(O(2^N)\) space.

This is unpractical when taking into account millions of variables. The
underlying idea of belief networks is to specify which variables are independent
of others, factoring the joint probability distribution.

\begin{definition}
Let \(G=(V,E)\) be a graph where \(V = \{X_1,\dots,X_n\}\) is a set of random
variables. We say that the joint
probability \(P(x_1, \dots, x_n)\) \emph{factorizes} according to \(G\) if and
only if
\[
P(x_1,\dots,x_N) = \prod_{i=1}^{N}P(x_i | pa(x_i))
\]
\end{definition}

\begin{definition}
A \emph{belief network or Bayesian network} is a pair \((G, P)\)
where \(P\) factorizes over \(G\). It is a probabilistic graphical model
that represents conditional dependencies of a set of variables \(X_1,\dots, X_n\).
\end{definition}

\begin{figure}
  \centering
  \begin{tikzpicture}[
    node distance=1.5cm and 1.5cm,
    mynode/.style={draw,circle,text width=0.5cm,align=center}
    ]

    \node[mynode] (1) {\(X_1\)};
    \node[mynode,right=of 1] (2) {\(X_2\)};
    \node[mynode,right=of 2] (3) {\(X_3\)};
    \node[mynode,right=of 3] (4) {\(X_4\)};

    \path (4) edge[-latex][bend right] (1)
    (3) edge[-latex] (2)
    (4) edge[-latex][bend right] (2)
    ;

    \end{tikzpicture}
    \captionof{figure}{Bayesian Network factorizing \(P(x_1, x_2, x_3, x_4) = P(x_1 | x_4)P(x_2| x_3, x_4)P(x_3)P(x_4)\)}
    \label{fig:bn_example}
\end{figure}


Any probability distribution can be written as a Bayesian Network, even though
it may end up been a fully-connected DAG.
To set the specification of the Belief Network, we need to define all elements of the probability
tables \(P(x_i|pa(x_i))\). When the number of variables is large, this is still
intractable so the tables are generally parameterized is a low dimensional
manner.

Bayesian Networks are good for encoding conditional independence over the
variables, but aren't for encoding dependence. For example, with the following
network \(P(x,y) = P(y|x)P(x)\) represented as \(x \to y\) in a DAG.
It may appear to encode dependence between both variables but the
conditional \(P(y|x)\) could happen to equal \(P(y)\), giving \(P(x,y) = P(x)P(y)\).

How could we check if two variables are conditionally independent given a
Bayesian Network? For example in figure \ref{fig:relations}, \(X_1 \bigCI
X_2 \mid X_4\) as\footnote{Continuous variable notation is used}:
\[
\begin{aligned}
P(x_2 | x_4) &= \frac{1}{P(x_4)}\int_{x_1,x_3}P(x_1, x_2, x_3, x_4)
= \frac{1}{P(x_4)}\int_{x_1,x_3}P(x_1|x_4)P(x_2|x_3,x_4)P(x_3)P(x_4)\\
                 &= \int_{x_3}P(x_2|x_3, x_4)P(x_3)
\end{aligned}
\]
\[
\begin{aligned}
P(x_1, x_2 | x_4) &= \frac{1}{P(x_4)}\int_{x_3}P(x_1, x_2, x_3, x_4)
= \frac{1}{P(x_4)}\int_{x_3}P(x_1|x_4)P(x_2|x_3,x_4)P(x_3)P(x_4)\\
                 &= P(x_1|x_4)\int_{x_3}P(x_2|x_3, x_4)P(x_3) = P(x_1|x_4)P(x_2|x_4)
\end{aligned}
\]

Now we are going to define two central concepts to determine conditional
independence in any Bayesian Network, these are \emph{d-connection} and \emph{d-separation}.

\begin{definition}
Let \(G\) be a DAG where \(\bm{X}, \bm{Y} \text{ and } \bm{Z}\)
are disjoint sets of vertices. We say that \(\bm{X} \text{ and
} \bm{Y}\) are \emph{d-connected} by \(\bm{Z}\) if and only if there
exists an undirected path \(U\) from any vertex in \(\bm{X}\) to any
vertex in \(\bm{Y}\) such that:
\begin{itemize}
\item For any collider \(C\), itself or any it's descendants is in \(\bm{Z}\)
\item No non-collider on \(U\) is on \(\bm{Z}\)
\end{itemize}
\end{definition}

\begin{definition}
Let \(G\) be a DAG where \(\bm{X}, \bm{Y} \text{ and } \bm{Z}\)
are disjoint sets of vertices. \(\bm{X}\) and \(\bm{Y}\)
are \emph{d-separated} by \(\bm{Z}\) if and only if they are not
d-connected by \(\bm{Z}\) in \(G\)
\end{definition}

\begin{figure}[h]
\centering
\begin{tikzpicture}[
  node distance=1cm and 0.5cm,
  mynode/.style={draw,circle,text width=0.5cm,align=center}
]

\node[mynode] (a) {a};
\node[mynode, below right=of a] (d) {d};
\node[mynode,above right=of d] (b) {b};
\node[mynode, below right=of b] (e) {e};
\node[mynode,above right=of e] (c) {c};

\path (a) edge[-latex] (d)
(b) edge[-latex] (d)
(c) edge[-latex] (e)
(b) edge[-latex] (e)
;

\end{tikzpicture}
\caption{D-separation example}
\label{fig:d-sep}
\end{figure}

For example, in figure \ref{fig:d-sep} \(d\) d-separates \(a\) and \(c\) (\(b\)
is a collider in the path that isn't in \(\{d\}\)),
and \(\{d,e\}\) d-connect them.

\begin{theorem}[Verma and Pearl, 1988,  Geiger et al.,
1990 \cite{pearl_and_detcher}]
Let \(G\) be a DAG where \(\bm{X}, \bm{Y} \text{ and } \bm{Z}\)
are disjoint sets of vertices. If  \(\bm{X}\) and \(\bm{Y}\)
are d-separated by \(\bm{Z}\), then they are independent conditional
on \(\bm{Z}\) in all probability distributions that G can represent.
\end{theorem}

The Bayes Ball algorithm \cite{bayes_ball} provides a linear time complexity
algorithm that computes conditional independent using this theorem.


\begin{exampleth}
In this example we are modeling three discrete random variables: Sprinkler (\(S\)),
Rain (\(R\)) and Grass wet (\(G\)).

The joint probability function is:
\[
P(s,r,g) = P(s|r)P(g|s,r)P(r)
\]

The following DAG illustrates the Bayesian Network among with the probability
tables we are using.

\begin{tikzpicture}[
  node distance=1cm and 0cm,
  mynode/.style={draw,ellipse,text width=2cm,align=center}
]
\node[mynode] (sp) {Sprinkler};
\node[mynode,below right=of sp] (gw) {Grass wet};
\node[mynode,above right=of gw] (ra) {Rain};
\path (ra) edge[-latex] (sp)
(sp) edge[-latex] (gw)
(gw) edge[latex-] (ra);
\node[left=0.5cm of sp]
{
\begin{tabular}{cm{1cm}m{1cm}}
\toprule
& \multicolumn{2}{c}{Sprinkler} \\
Rain & \multicolumn{1}{c}{T} & \multicolumn{1}{c}{F} \\
\cmidrule(r){1-1}\cmidrule(l){2-3}
F & 0.4 & 0.6 \\
T & 0.01 & 0.99 \\
\bottomrule
\end{tabular}
};
\node[right=0.5cm of ra]
{
\begin{tabular}{m{1cm}m{1cm}}
\toprule
\multicolumn{2}{c}{Rain} \\
\multicolumn{1}{c}{T} & \multicolumn{1}{c}{F} \\
\cmidrule{1-2}
0.2 & 0.8 \\
\bottomrule
\end{tabular}
};
\node[below=0.5cm of gw]
{
\begin{tabular}{ccm{1cm}m{1cm}}
\toprule
& & \multicolumn{2}{c}{Grass wet} \\
\multicolumn{2}{l}{Sprinkler Rain} & \multicolumn{1}{c}{T} & \multicolumn{1}{c}{F} \\
\cmidrule(r){1-2}\cmidrule(l){3-4}
F & F & 0.0 & 1.0 \\
F & T & 0.8 & 0.2 \\
T & F & 0.9 & 0.1 \\
T & T & 0.99 & 0.01 \\
\bottomrule
\end{tabular}
};
\end{tikzpicture}

This model can answer questions about the presence of a cause given the presence
of an effect. For example, What is the probability that it has being raining
given the grass is wet?

\[
P(R = T | G = T) = \frac{P(G = T, R = T)}{P(G=T)} = \frac{\sum_{s}P(G=T, R=T,
s)}{\sum_{r,s} P(G=T, r, s)}
\]

Using the expression of the joint probability among with the tables we can
compute every term. For example:
\[
\begin{aligned}
P(G=T, R=T, S=T) &= P(S=T|R=T)P(G=T|R=T,S=T)P(R=T) \\
&= 0.01 * 0.99 * 0.2 = 0.00198
\end{aligned}
\]
\end{exampleth}


In some situations our Belief Networks will contain a number of nodes that are
essentially the same but repeated a number of times, for this, we are going to
introduce the \emph{plate notation}. Suppose we have the situation that figure
\ref{fig:plate_notation} shows on the left. The we can collapse all \(B_i\)
variables in a box, indicating there number of variables inside it.

\begin{figure}[h]
\centering
\begin{tikzpicture}[
  node distance=1cm and 0.5cm,
  mynode/.style={draw,circle,text width=0.5cm,align=center}
]

\node[mynode] (a) {A};
\node[mynode,below=of a] (d) {\(B_1\)};
\node[mynode,left=of d] (c) {\(B_2\)};
\node[mynode,left=of c] (b) {\(B_3\)};
\node[mynode,right=of d] (e) {\(\dots\)};
\node[mynode,right=of e] (f) {\(B_n\)};

\node[mynode,right=2cm of f] (g) {\(B_i\)};
\node[mynode, above=of g] (h) {A};
\plate{} {(g)} {\(n\)}; %


\path (a) edge[-latex] (b)
(a) edge[-latex] (c)
(a) edge[-latex] (d)
(a) edge[-latex] (e)
(a) edge[-latex] (f)
(h) edge[-latex] (g)
;

\end{tikzpicture}
\caption{Plate notation example. Standard notation on the left and plate on the right}
\label{fig:plate_notation}
\end{figure}


\clearpage
Cites so the references appear (testing) \cite{koller_friedman,barber,wainwright}
\bibliographystyle{plain}
\bibliography{refs}
\end{document}
