
% Plantilla para un Trabajo Fin de Grado de la Universidad de Granada,
% adaptada para el Doble Grado en Ingeniería Informática y Matemáticas.
%
%  Autor: Mario Román.
%  Licencia: GNU GPLv2.
%
% Esta plantilla es una adaptación al castellano de la plantilla
% classicthesis de André Miede, que puede obtenerse en:
%  https://ctan.org/tex-archive/macros/latex/contrib/classicthesis?lang=en
% La plantilla original se licencia en GNU GPLv2.
%
% Esta plantilla usa símbolos de la Universidad de Granada sujetos a la normativa
% de identidad visual corporativa, que puede encontrarse en:
% http://secretariageneral.ugr.es/pages/ivc/normativa
%
% La compilación se realiza con las siguientes instrucciones:
%   pdflatex --shell-escape main.tex
%   bibtex main
%   pdflatex --shell-escape main.tex
%   pdflatex --shell-escape main.tex

% Opciones del tipo de documento
\documentclass[oneside,openright,titlepage,numbers=noenddot,openany,headinclude,footinclude=true,
  cleardoublepage=empty,abstractoff,BCOR=5mm,paper=a4,fontsize=12pt]{scrreprt}

% Paquetes de latex que se cargan al inicio. Cubren la entrada de
% texto, gráficos, código fuente y símbolos.
\usepackage[utf8]{inputenc}
\usepackage[T1]{fontenc}
\usepackage{fixltx2e}
\usepackage{graphicx} % Inclusión de imágenes.
\usepackage{grffile}  % Distintos formatos para imágenes.
\usepackage{longtable} % Tablas multipágina.
\usepackage{wrapfig} % Coloca texto alrededor de una figura.
\usepackage{rotating}
\usepackage[normalem]{ulem}
\usepackage{amsmath}
\usepackage{textcomp}
\usepackage{amssymb}
\usepackage{capt-of}
\usepackage[colorlinks=true]{hyperref}
\usepackage{tikz} % Diagramas conmutativos.
\usepackage{dcolumn}
\usepackage{booktabs}
\usepackage{minted} % Código fuente.
\usepackage{natbib}
\usepackage{bm}
\usepackage{todonotes}
\usepackage{centernot}
\usetikzlibrary{positioning}
\usetikzlibrary{bayesnet}

\DeclareMathOperator*{\argmax}{arg\,max}
\DeclareMathOperator*{\argmin}{arg\,min}
% Plantilla classicthesis
\usepackage[beramono,eulerchapternumbers,linedheaders,parts,a5paper,dottedtoc,
manychapters,pdfspacing]{classicthesis}

% Geometría y espaciado de párrafos.
\setcounter{secnumdepth}{0}
\usepackage{enumitem}
\setitemize{noitemsep,topsep=0pt,parsep=0pt,partopsep=0pt}
\setlist[enumerate]{topsep=0pt,itemsep=-1ex,partopsep=1ex,parsep=1ex}
\usepackage[top=1in, bottom=1.5in, left=1in, right=1in]{geometry}
\setlength\itemsep{0em}
\setlength{\parindent}{0pt}
\usepackage{parskip}

% Todo notes
\usepackage{todonotes}
\let\marginpar\oldmarginpar
% Profundidad de la tabla de contenidos.
\setcounter{secnumdepth}{3}

% Usa el paquete minted para mostrar trozos de código.
% Pueden seleccionarse el lenguaje apropiado y el estilo del código.
\usepackage{minted}
\usemintedstyle{colorful}
\setminted{fontsize=\small}
\renewcommand{\theFancyVerbLine}{\sffamily\textcolor[rgb]{0.5,0.5,1.0}{\oldstylenums{\arabic{FancyVerbLine}}}}
\newcommand{\bigCI}{\mathrel{\text{\scalebox{1.07}{$\perp\mkern-10mu\perp$}}}}
\newcommand{\bigCD}{\centernot{\bigCI}}
% Archivos de configuración.
%------------------------
% Bibliotecas para matemáticas de latex
%------------------------
\usepackage{amsthm}
\usepackage{amsmath}
\usepackage{tikz}
\usepackage{tikz-cd}
\usetikzlibrary{shapes,fit}
\usepackage{bussproofs}
\EnableBpAbbreviations{}
\usepackage{mathtools}
\usepackage{scalerel}
\usepackage{stmaryrd}

%------------------------
% Estilos para los teoremas
%------------------------
\theoremstyle{plain}
\newtheorem{theorem}{Theorem}
\newtheorem{proposition}{Proposition}
\newtheorem{lemma}{Lemma}
\newtheorem{corollary}{Corollary}
\theoremstyle{definition}
\newtheorem{definition}{Definition}
\newtheorem{proofs}{Proof}
\theoremstyle{remark}
\newtheorem{remark}{Remark}
\newtheorem{exampleth}{Example}

\begingroup\makeatletter\@for\theoremstyle:=definition,remark,plain\do{\expandafter\g@addto@macro\csname th@\theoremstyle\endcsname{\addtolength\thm@preskip\parskip}}\endgroup

%------------------------
% Macros
% ------------------------

% Aquí pueden añadirse abreviaturas para comandos de latex
% frequentemente usados.
\newcommand*\diff{\mathop{}\!\mathrm{d}}
\newcommand{\R}{\mathbb{R}}
% \newcommand{\E}{\mathbb{E}}
\newcommand{\bmu}{\bm{\mu}}
\newcommand{\bx}{\bm{x}}
\newcommand{\bX}{\bm{X}}
\newcommand{\bz}{\bm{z}}
\newcommand{\bZ}{\bm{Z}}
\newcommand{\bv}{\bm{v}}
\newcommand{\bh}{\bm{h}}
\newcommand{\bSigma}{\bm{\Sigma}}
\newcommand{\bpi}{\bm{\pi}}
\newcommand{\bLambda}{\bm{\Lambda}}
\newcommand{\btheta}{\bm{\theta}}

\newcommand{\V}{\mathcal{V}}
\newcommand{\D}{\mathcal{D}}
\newcommand{\X}{\mathcal{X}}
\newcommand{\I}{\mathcal{I}}

\newcommand\ddfrac[2]{\frac{\displaystyle #1}{\displaystyle #2}}

\newcommand\E[2]{\mathbb{E}_{#1}\Big[#2\Big]}
\newcommand\KL[2]{KL\Big(#1 \bigm| #2\Big)}
\newcommand{\bigCI}{\mathrel{\text{\scalebox{1.07}{$\perp\mkern-10mu\perp$}}}}
\newcommand{\bigCD}{\centernot{\bigCI}}

\DeclareMathOperator*{\argmax}{arg\,max}
\DeclareMathOperator*{\argmin}{arg\,min}
  % En macros.tex se almacenan las opciones y comandos para escribir matemáticas.
% ****************************************************************************************************
% classicthesis-config.tex 
% formerly known as loadpackages.sty, classicthesis-ldpkg.sty, and classicthesis-preamble.sty 
% Use it at the beginning of your ClassicThesis.tex, or as a LaTeX Preamble 
% in your ClassicThesis.{tex,lyx} with % ****************************************************************************************************
% classicthesis-config.tex 
% formerly known as loadpackages.sty, classicthesis-ldpkg.sty, and classicthesis-preamble.sty 
% Use it at the beginning of your ClassicThesis.tex, or as a LaTeX Preamble 
% in your ClassicThesis.{tex,lyx} with % ****************************************************************************************************
% classicthesis-config.tex 
% formerly known as loadpackages.sty, classicthesis-ldpkg.sty, and classicthesis-preamble.sty 
% Use it at the beginning of your ClassicThesis.tex, or as a LaTeX Preamble 
% in your ClassicThesis.{tex,lyx} with \input{classicthesis-config}
% ****************************************************************************************************  
% If you like the classicthesis, then I would appreciate a postcard. 
% My address can be found in the file ClassicThesis.pdf. A collection 
% of the postcards I received so far is available online at 
% http://postcards.miede.de
% ****************************************************************************************************


% ****************************************************************************************************
% 0. Set the encoding of your files. UTF-8 is the only sensible encoding nowadays. If you can't read
% äöüßáéçèê∂åëæƒÏ€ then change the encoding setting in your editor, not the line below. If your editor
% does not support utf8 use another editor!
% ****************************************************************************************************
\PassOptionsToPackage{utf8x}{inputenc}
	\usepackage{inputenc}

% ****************************************************************************************************
% 1. Configure classicthesis for your needs here, e.g., remove "drafting" below 
% in order to deactivate the time-stamp on the pages
% ****************************************************************************************************
\PassOptionsToPackage{eulerchapternumbers,listings,drafting,%
		pdfspacing,%floatperchapter,%linedheaders,%
                subfig,beramono,eulermath,parts,dottedtoc}{classicthesis}                                        
% ********************************************************************
% Available options for classicthesis.sty 
% (see ClassicThesis.pdf for more information):
% drafting
% parts nochapters linedheaders
% eulerchapternumbers beramono eulermath pdfspacing minionprospacing
% tocaligned dottedtoc manychapters
% listings floatperchapter subfig
% ********************************************************************

% ****************************************************************************************************
% 2. Personal data and user ad-hoc commands
% ****************************************************************************************************
\newcommand{\myTitle}{A Classic Thesis Style\xspace}
\newcommand{\mySubtitle}{An Homage to The Elements of Typographic Style\xspace}
\newcommand{\myDegree}{Doktor-Ingenieur (Dr.-Ing.)\xspace}
\newcommand{\myName}{André Miede\xspace}
\newcommand{\myProf}{Put name here\xspace}
\newcommand{\myOtherProf}{Put name here\xspace}
\newcommand{\mySupervisor}{Put name here\xspace}
\newcommand{\myFaculty}{Put data here\xspace}
\newcommand{\myDepartment}{Put data here\xspace}
\newcommand{\myUni}{Put data here\xspace}
\newcommand{\myLocation}{Saarbrücken\xspace}
\newcommand{\myTime}{September 2015\xspace}
%\newcommand{\myVersion}{version 4.2\xspace}

% ********************************************************************
% Setup, finetuning, and useful commands
% ********************************************************************
\newcounter{dummy} % necessary for correct hyperlinks (to index, bib, etc.)
\newlength{\abcd} % for ab..z string length calculation
\providecommand{\mLyX}{L\kern-.1667em\lower.25em\hbox{Y}\kern-.125emX\@}
\newcommand{\ie}{i.\,e.}
\newcommand{\Ie}{I.\,e.}
\newcommand{\eg}{e.\,g.}
\newcommand{\Eg}{E.\,g.} 
% ****************************************************************************************************


% ****************************************************************************************************
% 3. Loading some handy packages
% ****************************************************************************************************
% ******************************************************************** 
% Packages with options that might require adjustments
% ******************************************************************** 
%\PassOptionsToPackage{ngerman,american}{babel}   % change this to your language(s)
% Spanish languages need extra options in order to work with this template
% \PassOptionsToPackage{es-lcroman,spanish}{babel}
\usepackage[main=english]{babel}

%\usepackage{csquotes}
% \PassOptionsToPackage{%
%     %backend=biber, %instead of bibtex
% 	backend=bibtex8,bibencoding=ascii,%
% 	language=auto,%
% 	style=alpha,%
%     %style=authoryear-comp, % Author 1999, 2010
%     %bibstyle=authoryear,dashed=false, % dashed: substitute rep. author with ---
%     sorting=nyt, % name, year, title
%     maxbibnames=10, % default: 3, et al.
%     %backref=true,%
%     natbib=true % natbib compatibility mode (\citep and \citet still work)
% }{biblatex}
%     \usepackage{biblatex}

% \PassOptionsToPackage{fleqn}{amsmath}       % math environments and more by the AMS 
%     \usepackage{amsmath}

% ******************************************************************** 
% General useful packages
% ******************************************************************** 
\PassOptionsToPackage{T1}{fontenc} % T2A for cyrillics
    \usepackage{fontenc}     
\usepackage{textcomp} % fix warning with missing font shapes
\usepackage{scrhack} % fix warnings when using KOMA with listings package          
\usepackage{xspace} % to get the spacing after macros right  
\usepackage{mparhack} % get marginpar right
\usepackage{fixltx2e} % fixes some LaTeX stuff --> since 2015 in the LaTeX kernel (see below)
%\usepackage[latest]{latexrelease} % will be used once available in more distributions (ISSUE #107)
\PassOptionsToPackage{printonlyused,smaller}{acronym} 
    \usepackage{acronym} % nice macros for handling all acronyms in the thesis
    %\renewcommand{\bflabel}[1]{{#1}\hfill} % fix the list of acronyms --> no longer working
    %\renewcommand*{\acsfont}[1]{\textsc{#1}} 
    \renewcommand*{\aclabelfont}[1]{\acsfont{#1}}
% ****************************************************************************************************


% ****************************************************************************************************
% 4. Setup floats: tables, (sub)figures, and captions
% ****************************************************************************************************
\usepackage{tabularx} % better tables
    \setlength{\extrarowheight}{3pt} % increase table row height
\newcommand{\tableheadline}[1]{\multicolumn{1}{c}{\spacedlowsmallcaps{#1}}}
\newcommand{\myfloatalign}{\centering} % to be used with each float for alignment
\usepackage{caption}
% Thanks to cgnieder and Claus Lahiri
% http://tex.stackexchange.com/questions/69349/spacedlowsmallcaps-in-caption-label
% [REMOVED DUE TO OTHER PROBLEMS, SEE ISSUE #82]    
%\DeclareCaptionLabelFormat{smallcaps}{\bothIfFirst{#1}{~}\MakeTextLowercase{\textsc{#2}}}
%\captionsetup{font=small,labelformat=smallcaps} % format=hang,
\captionsetup{font=small} % format=hang,
\usepackage{subfig}  
% ****************************************************************************************************


% ****************************************************************************************************
% 5. Setup code listings
% ****************************************************************************************************
% \usepackage{listings} 
% %\lstset{emph={trueIndex,root},emphstyle=\color{BlueViolet}}%\underbar} % for special keywords
% \lstset{language={Haskell},morekeywords={PassOptionsToPackage,selectlanguage},keywordstyle=\color{RoyalBlue},basicstyle=\small\ttfamily,commentstyle=\color{Green}\ttfamily,stringstyle=\rmfamily,numbers=none,numberstyle=\scriptsize,stepnumber=5,numbersep=8pt,showstringspaces=false,breaklines=true,belowcaptionskip=.75\baselineskip} 
% ****************************************************************************************************             


% ****************************************************************************************************
% 6. PDFLaTeX, hyperreferences and citation backreferences
% ****************************************************************************************************
% ********************************************************************
% Using PDFLaTeX
% ********************************************************************
\PassOptionsToPackage{pdftex,hyperfootnotes=false,pdfpagelabels}{hyperref}
    \usepackage{hyperref}  % backref linktocpage pagebackref
\pdfcompresslevel=9
\pdfadjustspacing=1 
\PassOptionsToPackage{pdftex}{graphicx}
    \usepackage{graphicx} 
 

% ********************************************************************
% Hyperreferences
% ********************************************************************
\hypersetup{%
    %draft, % = no hyperlinking at all (useful in b/w printouts)
    colorlinks=true, linktocpage=true, pdfstartpage=3, pdfstartview=FitV,%
    % uncomment the following line if you want to have black links (e.g., for printing)
    %colorlinks=false, linktocpage=false, pdfstartpage=3, pdfstartview=FitV, pdfborder={0 0 0},%
    breaklinks=true, pdfpagemode=UseNone, pageanchor=true, pdfpagemode=UseOutlines,%
    plainpages=false, bookmarksnumbered, bookmarksopen=true, bookmarksopenlevel=1,%
    hypertexnames=true, pdfhighlight=/O,%nesting=true,%frenchlinks,%
    urlcolor=webbrown, linkcolor=RoyalBlue, citecolor=webgreen, %pagecolor=RoyalBlue,%
    %urlcolor=Black, linkcolor=Black, citecolor=Black, %pagecolor=Black,%
    pdftitle={\myTitle},%
    pdfauthor={\textcopyright\ \myName, \myUni, \myFaculty},%
    pdfsubject={},%
    pdfkeywords={},%
    pdfcreator={pdfLaTeX},%
    pdfproducer={LaTeX with hyperref and classicthesis}%
}   

% ********************************************************************
% Setup autoreferences
% ********************************************************************
% There are some issues regarding autorefnames
% http://www.ureader.de/msg/136221647.aspx
% http://www.tex.ac.uk/cgi-bin/texfaq2html?label=latexwords
% you have to redefine the makros for the 
% language you use, e.g., american, ngerman
% (as chosen when loading babel/AtBeginDocument)
% ********************************************************************
\makeatletter
\@ifpackageloaded{babel}%
    {%
       \addto\extrasamerican{%
			\renewcommand*{\figureautorefname}{Figure}%
			\renewcommand*{\tableautorefname}{Table}%
			\renewcommand*{\partautorefname}{Part}%
			\renewcommand*{\chapterautorefname}{Chapter}%
			\renewcommand*{\sectionautorefname}{Section}%
			\renewcommand*{\subsectionautorefname}{Section}%
			\renewcommand*{\subsubsectionautorefname}{Section}%     
                }%
       \addto\extrasngerman{% 
			\renewcommand*{\paragraphautorefname}{Absatz}%
			\renewcommand*{\subparagraphautorefname}{Unterabsatz}%
			\renewcommand*{\footnoteautorefname}{Fu\"snote}%
			\renewcommand*{\FancyVerbLineautorefname}{Zeile}%
			\renewcommand*{\theoremautorefname}{Theorem}%
			\renewcommand*{\appendixautorefname}{Anhang}%
			\renewcommand*{\equationautorefname}{Gleichung}%        
			\renewcommand*{\itemautorefname}{Punkt}%
                }%  
            % Fix to getting autorefs for subfigures right (thanks to Belinda Vogt for changing the definition)
            \providecommand{\subfigureautorefname}{\figureautorefname}%             
    }{\relax}
\makeatother


% ****************************************************************************************************
% 7. Last calls before the bar closes
% ****************************************************************************************************
% ********************************************************************
% Development Stuff
% ********************************************************************
\listfiles
%\PassOptionsToPackage{l2tabu,orthodox,abort}{nag}
%   \usepackage{nag}
%\PassOptionsToPackage{warning, all}{onlyamsmath}
%   \usepackage{onlyamsmath}

% ********************************************************************
% Last, but not least...
% ********************************************************************
\usepackage{classicthesis} 
% ****************************************************************************************************


% ****************************************************************************************************
% 8. Further adjustments (experimental)
% ****************************************************************************************************
% ********************************************************************
% Changing the text area
% ********************************************************************
\linespread{1.05} % a bit more for Palatino
% \areaset[current]{325pt}{680pt} % 686 (factor 2.2) + 33 head + 42 head \the\footskip
%\setlength{\marginparwidth}{7em}%
%\setlength{\marginparsep}{2em}%

% ********************************************************************
% Using different fonts
% ********************************************************************
%\usepackage[oldstylenums]{kpfonts} % oldstyle notextcomp
%\usepackage[osf]{libertine}
%\usepackage[light,condensed,math]{iwona}
%\renewcommand{\sfdefault}{iwona}
%\usepackage{lmodern} % <-- no osf support :-(
%\usepackage{cfr-lm} % 
%\usepackage[urw-garamond]{mathdesign} <-- no osf support :-(
%\usepackage[default,osfigures]{opensans} % scale=0.95 
%\usepackage[sfdefault]{FiraSans}
% ****************************************************************************************************

% ****************************************************************************************************  
% If you like the classicthesis, then I would appreciate a postcard. 
% My address can be found in the file ClassicThesis.pdf. A collection 
% of the postcards I received so far is available online at 
% http://postcards.miede.de
% ****************************************************************************************************


% ****************************************************************************************************
% 0. Set the encoding of your files. UTF-8 is the only sensible encoding nowadays. If you can't read
% äöüßáéçèê∂åëæƒÏ€ then change the encoding setting in your editor, not the line below. If your editor
% does not support utf8 use another editor!
% ****************************************************************************************************
\PassOptionsToPackage{utf8x}{inputenc}
	\usepackage{inputenc}

% ****************************************************************************************************
% 1. Configure classicthesis for your needs here, e.g., remove "drafting" below 
% in order to deactivate the time-stamp on the pages
% ****************************************************************************************************
\PassOptionsToPackage{eulerchapternumbers,listings,drafting,%
		pdfspacing,%floatperchapter,%linedheaders,%
                subfig,beramono,eulermath,parts,dottedtoc}{classicthesis}                                        
% ********************************************************************
% Available options for classicthesis.sty 
% (see ClassicThesis.pdf for more information):
% drafting
% parts nochapters linedheaders
% eulerchapternumbers beramono eulermath pdfspacing minionprospacing
% tocaligned dottedtoc manychapters
% listings floatperchapter subfig
% ********************************************************************

% ****************************************************************************************************
% 2. Personal data and user ad-hoc commands
% ****************************************************************************************************
\newcommand{\myTitle}{A Classic Thesis Style\xspace}
\newcommand{\mySubtitle}{An Homage to The Elements of Typographic Style\xspace}
\newcommand{\myDegree}{Doktor-Ingenieur (Dr.-Ing.)\xspace}
\newcommand{\myName}{André Miede\xspace}
\newcommand{\myProf}{Put name here\xspace}
\newcommand{\myOtherProf}{Put name here\xspace}
\newcommand{\mySupervisor}{Put name here\xspace}
\newcommand{\myFaculty}{Put data here\xspace}
\newcommand{\myDepartment}{Put data here\xspace}
\newcommand{\myUni}{Put data here\xspace}
\newcommand{\myLocation}{Saarbrücken\xspace}
\newcommand{\myTime}{September 2015\xspace}
%\newcommand{\myVersion}{version 4.2\xspace}

% ********************************************************************
% Setup, finetuning, and useful commands
% ********************************************************************
\newcounter{dummy} % necessary for correct hyperlinks (to index, bib, etc.)
\newlength{\abcd} % for ab..z string length calculation
\providecommand{\mLyX}{L\kern-.1667em\lower.25em\hbox{Y}\kern-.125emX\@}
\newcommand{\ie}{i.\,e.}
\newcommand{\Ie}{I.\,e.}
\newcommand{\eg}{e.\,g.}
\newcommand{\Eg}{E.\,g.} 
% ****************************************************************************************************


% ****************************************************************************************************
% 3. Loading some handy packages
% ****************************************************************************************************
% ******************************************************************** 
% Packages with options that might require adjustments
% ******************************************************************** 
%\PassOptionsToPackage{ngerman,american}{babel}   % change this to your language(s)
% Spanish languages need extra options in order to work with this template
% \PassOptionsToPackage{es-lcroman,spanish}{babel}
\usepackage[main=english]{babel}

%\usepackage{csquotes}
% \PassOptionsToPackage{%
%     %backend=biber, %instead of bibtex
% 	backend=bibtex8,bibencoding=ascii,%
% 	language=auto,%
% 	style=alpha,%
%     %style=authoryear-comp, % Author 1999, 2010
%     %bibstyle=authoryear,dashed=false, % dashed: substitute rep. author with ---
%     sorting=nyt, % name, year, title
%     maxbibnames=10, % default: 3, et al.
%     %backref=true,%
%     natbib=true % natbib compatibility mode (\citep and \citet still work)
% }{biblatex}
%     \usepackage{biblatex}

% \PassOptionsToPackage{fleqn}{amsmath}       % math environments and more by the AMS 
%     \usepackage{amsmath}

% ******************************************************************** 
% General useful packages
% ******************************************************************** 
\PassOptionsToPackage{T1}{fontenc} % T2A for cyrillics
    \usepackage{fontenc}     
\usepackage{textcomp} % fix warning with missing font shapes
\usepackage{scrhack} % fix warnings when using KOMA with listings package          
\usepackage{xspace} % to get the spacing after macros right  
\usepackage{mparhack} % get marginpar right
\usepackage{fixltx2e} % fixes some LaTeX stuff --> since 2015 in the LaTeX kernel (see below)
%\usepackage[latest]{latexrelease} % will be used once available in more distributions (ISSUE #107)
\PassOptionsToPackage{printonlyused,smaller}{acronym} 
    \usepackage{acronym} % nice macros for handling all acronyms in the thesis
    %\renewcommand{\bflabel}[1]{{#1}\hfill} % fix the list of acronyms --> no longer working
    %\renewcommand*{\acsfont}[1]{\textsc{#1}} 
    \renewcommand*{\aclabelfont}[1]{\acsfont{#1}}
% ****************************************************************************************************


% ****************************************************************************************************
% 4. Setup floats: tables, (sub)figures, and captions
% ****************************************************************************************************
\usepackage{tabularx} % better tables
    \setlength{\extrarowheight}{3pt} % increase table row height
\newcommand{\tableheadline}[1]{\multicolumn{1}{c}{\spacedlowsmallcaps{#1}}}
\newcommand{\myfloatalign}{\centering} % to be used with each float for alignment
\usepackage{caption}
% Thanks to cgnieder and Claus Lahiri
% http://tex.stackexchange.com/questions/69349/spacedlowsmallcaps-in-caption-label
% [REMOVED DUE TO OTHER PROBLEMS, SEE ISSUE #82]    
%\DeclareCaptionLabelFormat{smallcaps}{\bothIfFirst{#1}{~}\MakeTextLowercase{\textsc{#2}}}
%\captionsetup{font=small,labelformat=smallcaps} % format=hang,
\captionsetup{font=small} % format=hang,
\usepackage{subfig}  
% ****************************************************************************************************


% ****************************************************************************************************
% 5. Setup code listings
% ****************************************************************************************************
% \usepackage{listings} 
% %\lstset{emph={trueIndex,root},emphstyle=\color{BlueViolet}}%\underbar} % for special keywords
% \lstset{language={Haskell},morekeywords={PassOptionsToPackage,selectlanguage},keywordstyle=\color{RoyalBlue},basicstyle=\small\ttfamily,commentstyle=\color{Green}\ttfamily,stringstyle=\rmfamily,numbers=none,numberstyle=\scriptsize,stepnumber=5,numbersep=8pt,showstringspaces=false,breaklines=true,belowcaptionskip=.75\baselineskip} 
% ****************************************************************************************************             


% ****************************************************************************************************
% 6. PDFLaTeX, hyperreferences and citation backreferences
% ****************************************************************************************************
% ********************************************************************
% Using PDFLaTeX
% ********************************************************************
\PassOptionsToPackage{pdftex,hyperfootnotes=false,pdfpagelabels}{hyperref}
    \usepackage{hyperref}  % backref linktocpage pagebackref
\pdfcompresslevel=9
\pdfadjustspacing=1 
\PassOptionsToPackage{pdftex}{graphicx}
    \usepackage{graphicx} 
 

% ********************************************************************
% Hyperreferences
% ********************************************************************
\hypersetup{%
    %draft, % = no hyperlinking at all (useful in b/w printouts)
    colorlinks=true, linktocpage=true, pdfstartpage=3, pdfstartview=FitV,%
    % uncomment the following line if you want to have black links (e.g., for printing)
    %colorlinks=false, linktocpage=false, pdfstartpage=3, pdfstartview=FitV, pdfborder={0 0 0},%
    breaklinks=true, pdfpagemode=UseNone, pageanchor=true, pdfpagemode=UseOutlines,%
    plainpages=false, bookmarksnumbered, bookmarksopen=true, bookmarksopenlevel=1,%
    hypertexnames=true, pdfhighlight=/O,%nesting=true,%frenchlinks,%
    urlcolor=webbrown, linkcolor=RoyalBlue, citecolor=webgreen, %pagecolor=RoyalBlue,%
    %urlcolor=Black, linkcolor=Black, citecolor=Black, %pagecolor=Black,%
    pdftitle={\myTitle},%
    pdfauthor={\textcopyright\ \myName, \myUni, \myFaculty},%
    pdfsubject={},%
    pdfkeywords={},%
    pdfcreator={pdfLaTeX},%
    pdfproducer={LaTeX with hyperref and classicthesis}%
}   

% ********************************************************************
% Setup autoreferences
% ********************************************************************
% There are some issues regarding autorefnames
% http://www.ureader.de/msg/136221647.aspx
% http://www.tex.ac.uk/cgi-bin/texfaq2html?label=latexwords
% you have to redefine the makros for the 
% language you use, e.g., american, ngerman
% (as chosen when loading babel/AtBeginDocument)
% ********************************************************************
\makeatletter
\@ifpackageloaded{babel}%
    {%
       \addto\extrasamerican{%
			\renewcommand*{\figureautorefname}{Figure}%
			\renewcommand*{\tableautorefname}{Table}%
			\renewcommand*{\partautorefname}{Part}%
			\renewcommand*{\chapterautorefname}{Chapter}%
			\renewcommand*{\sectionautorefname}{Section}%
			\renewcommand*{\subsectionautorefname}{Section}%
			\renewcommand*{\subsubsectionautorefname}{Section}%     
                }%
       \addto\extrasngerman{% 
			\renewcommand*{\paragraphautorefname}{Absatz}%
			\renewcommand*{\subparagraphautorefname}{Unterabsatz}%
			\renewcommand*{\footnoteautorefname}{Fu\"snote}%
			\renewcommand*{\FancyVerbLineautorefname}{Zeile}%
			\renewcommand*{\theoremautorefname}{Theorem}%
			\renewcommand*{\appendixautorefname}{Anhang}%
			\renewcommand*{\equationautorefname}{Gleichung}%        
			\renewcommand*{\itemautorefname}{Punkt}%
                }%  
            % Fix to getting autorefs for subfigures right (thanks to Belinda Vogt for changing the definition)
            \providecommand{\subfigureautorefname}{\figureautorefname}%             
    }{\relax}
\makeatother


% ****************************************************************************************************
% 7. Last calls before the bar closes
% ****************************************************************************************************
% ********************************************************************
% Development Stuff
% ********************************************************************
\listfiles
%\PassOptionsToPackage{l2tabu,orthodox,abort}{nag}
%   \usepackage{nag}
%\PassOptionsToPackage{warning, all}{onlyamsmath}
%   \usepackage{onlyamsmath}

% ********************************************************************
% Last, but not least...
% ********************************************************************
\usepackage{classicthesis} 
% ****************************************************************************************************


% ****************************************************************************************************
% 8. Further adjustments (experimental)
% ****************************************************************************************************
% ********************************************************************
% Changing the text area
% ********************************************************************
\linespread{1.05} % a bit more for Palatino
% \areaset[current]{325pt}{680pt} % 686 (factor 2.2) + 33 head + 42 head \the\footskip
%\setlength{\marginparwidth}{7em}%
%\setlength{\marginparsep}{2em}%

% ********************************************************************
% Using different fonts
% ********************************************************************
%\usepackage[oldstylenums]{kpfonts} % oldstyle notextcomp
%\usepackage[osf]{libertine}
%\usepackage[light,condensed,math]{iwona}
%\renewcommand{\sfdefault}{iwona}
%\usepackage{lmodern} % <-- no osf support :-(
%\usepackage{cfr-lm} % 
%\usepackage[urw-garamond]{mathdesign} <-- no osf support :-(
%\usepackage[default,osfigures]{opensans} % scale=0.95 
%\usepackage[sfdefault]{FiraSans}
% ****************************************************************************************************

% ****************************************************************************************************  
% If you like the classicthesis, then I would appreciate a postcard. 
% My address can be found in the file ClassicThesis.pdf. A collection 
% of the postcards I received so far is available online at 
% http://postcards.miede.de
% ****************************************************************************************************


% ****************************************************************************************************
% 0. Set the encoding of your files. UTF-8 is the only sensible encoding nowadays. If you can't read
% äöüßáéçèê∂åëæƒÏ€ then change the encoding setting in your editor, not the line below. If your editor
% does not support utf8 use another editor!
% ****************************************************************************************************
\PassOptionsToPackage{utf8x}{inputenc}
	\usepackage{inputenc}

% ****************************************************************************************************
% 1. Configure classicthesis for your needs here, e.g., remove "drafting" below 
% in order to deactivate the time-stamp on the pages
% ****************************************************************************************************
\PassOptionsToPackage{eulerchapternumbers,listings,drafting,%
		pdfspacing,%floatperchapter,%linedheaders,%
                subfig,beramono,eulermath,parts,dottedtoc}{classicthesis}                                        
% ********************************************************************
% Available options for classicthesis.sty 
% (see ClassicThesis.pdf for more information):
% drafting
% parts nochapters linedheaders
% eulerchapternumbers beramono eulermath pdfspacing minionprospacing
% tocaligned dottedtoc manychapters
% listings floatperchapter subfig
% ********************************************************************

% ****************************************************************************************************
% 2. Personal data and user ad-hoc commands
% ****************************************************************************************************
\newcommand{\myTitle}{A Classic Thesis Style\xspace}
\newcommand{\mySubtitle}{An Homage to The Elements of Typographic Style\xspace}
\newcommand{\myDegree}{Doktor-Ingenieur (Dr.-Ing.)\xspace}
\newcommand{\myName}{André Miede\xspace}
\newcommand{\myProf}{Put name here\xspace}
\newcommand{\myOtherProf}{Put name here\xspace}
\newcommand{\mySupervisor}{Put name here\xspace}
\newcommand{\myFaculty}{Put data here\xspace}
\newcommand{\myDepartment}{Put data here\xspace}
\newcommand{\myUni}{Put data here\xspace}
\newcommand{\myLocation}{Saarbrücken\xspace}
\newcommand{\myTime}{September 2015\xspace}
%\newcommand{\myVersion}{version 4.2\xspace}

% ********************************************************************
% Setup, finetuning, and useful commands
% ********************************************************************
\newcounter{dummy} % necessary for correct hyperlinks (to index, bib, etc.)
\newlength{\abcd} % for ab..z string length calculation
\providecommand{\mLyX}{L\kern-.1667em\lower.25em\hbox{Y}\kern-.125emX\@}
\newcommand{\ie}{i.\,e.}
\newcommand{\Ie}{I.\,e.}
\newcommand{\eg}{e.\,g.}
\newcommand{\Eg}{E.\,g.} 
% ****************************************************************************************************


% ****************************************************************************************************
% 3. Loading some handy packages
% ****************************************************************************************************
% ******************************************************************** 
% Packages with options that might require adjustments
% ******************************************************************** 
%\PassOptionsToPackage{ngerman,american}{babel}   % change this to your language(s)
% Spanish languages need extra options in order to work with this template
% \PassOptionsToPackage{es-lcroman,spanish}{babel}
\usepackage[main=english]{babel}

%\usepackage{csquotes}
% \PassOptionsToPackage{%
%     %backend=biber, %instead of bibtex
% 	backend=bibtex8,bibencoding=ascii,%
% 	language=auto,%
% 	style=alpha,%
%     %style=authoryear-comp, % Author 1999, 2010
%     %bibstyle=authoryear,dashed=false, % dashed: substitute rep. author with ---
%     sorting=nyt, % name, year, title
%     maxbibnames=10, % default: 3, et al.
%     %backref=true,%
%     natbib=true % natbib compatibility mode (\citep and \citet still work)
% }{biblatex}
%     \usepackage{biblatex}

% \PassOptionsToPackage{fleqn}{amsmath}       % math environments and more by the AMS 
%     \usepackage{amsmath}

% ******************************************************************** 
% General useful packages
% ******************************************************************** 
\PassOptionsToPackage{T1}{fontenc} % T2A for cyrillics
    \usepackage{fontenc}     
\usepackage{textcomp} % fix warning with missing font shapes
\usepackage{scrhack} % fix warnings when using KOMA with listings package          
\usepackage{xspace} % to get the spacing after macros right  
\usepackage{mparhack} % get marginpar right
\usepackage{fixltx2e} % fixes some LaTeX stuff --> since 2015 in the LaTeX kernel (see below)
%\usepackage[latest]{latexrelease} % will be used once available in more distributions (ISSUE #107)
\PassOptionsToPackage{printonlyused,smaller}{acronym} 
    \usepackage{acronym} % nice macros for handling all acronyms in the thesis
    %\renewcommand{\bflabel}[1]{{#1}\hfill} % fix the list of acronyms --> no longer working
    %\renewcommand*{\acsfont}[1]{\textsc{#1}} 
    \renewcommand*{\aclabelfont}[1]{\acsfont{#1}}
% ****************************************************************************************************


% ****************************************************************************************************
% 4. Setup floats: tables, (sub)figures, and captions
% ****************************************************************************************************
\usepackage{tabularx} % better tables
    \setlength{\extrarowheight}{3pt} % increase table row height
\newcommand{\tableheadline}[1]{\multicolumn{1}{c}{\spacedlowsmallcaps{#1}}}
\newcommand{\myfloatalign}{\centering} % to be used with each float for alignment
\usepackage{caption}
% Thanks to cgnieder and Claus Lahiri
% http://tex.stackexchange.com/questions/69349/spacedlowsmallcaps-in-caption-label
% [REMOVED DUE TO OTHER PROBLEMS, SEE ISSUE #82]    
%\DeclareCaptionLabelFormat{smallcaps}{\bothIfFirst{#1}{~}\MakeTextLowercase{\textsc{#2}}}
%\captionsetup{font=small,labelformat=smallcaps} % format=hang,
\captionsetup{font=small} % format=hang,
\usepackage{subfig}  
% ****************************************************************************************************


% ****************************************************************************************************
% 5. Setup code listings
% ****************************************************************************************************
% \usepackage{listings} 
% %\lstset{emph={trueIndex,root},emphstyle=\color{BlueViolet}}%\underbar} % for special keywords
% \lstset{language={Haskell},morekeywords={PassOptionsToPackage,selectlanguage},keywordstyle=\color{RoyalBlue},basicstyle=\small\ttfamily,commentstyle=\color{Green}\ttfamily,stringstyle=\rmfamily,numbers=none,numberstyle=\scriptsize,stepnumber=5,numbersep=8pt,showstringspaces=false,breaklines=true,belowcaptionskip=.75\baselineskip} 
% ****************************************************************************************************             


% ****************************************************************************************************
% 6. PDFLaTeX, hyperreferences and citation backreferences
% ****************************************************************************************************
% ********************************************************************
% Using PDFLaTeX
% ********************************************************************
\PassOptionsToPackage{pdftex,hyperfootnotes=false,pdfpagelabels}{hyperref}
    \usepackage{hyperref}  % backref linktocpage pagebackref
\pdfcompresslevel=9
\pdfadjustspacing=1 
\PassOptionsToPackage{pdftex}{graphicx}
    \usepackage{graphicx} 
 

% ********************************************************************
% Hyperreferences
% ********************************************************************
\hypersetup{%
    %draft, % = no hyperlinking at all (useful in b/w printouts)
    colorlinks=true, linktocpage=true, pdfstartpage=3, pdfstartview=FitV,%
    % uncomment the following line if you want to have black links (e.g., for printing)
    %colorlinks=false, linktocpage=false, pdfstartpage=3, pdfstartview=FitV, pdfborder={0 0 0},%
    breaklinks=true, pdfpagemode=UseNone, pageanchor=true, pdfpagemode=UseOutlines,%
    plainpages=false, bookmarksnumbered, bookmarksopen=true, bookmarksopenlevel=1,%
    hypertexnames=true, pdfhighlight=/O,%nesting=true,%frenchlinks,%
    urlcolor=webbrown, linkcolor=RoyalBlue, citecolor=webgreen, %pagecolor=RoyalBlue,%
    %urlcolor=Black, linkcolor=Black, citecolor=Black, %pagecolor=Black,%
    pdftitle={\myTitle},%
    pdfauthor={\textcopyright\ \myName, \myUni, \myFaculty},%
    pdfsubject={},%
    pdfkeywords={},%
    pdfcreator={pdfLaTeX},%
    pdfproducer={LaTeX with hyperref and classicthesis}%
}   

% ********************************************************************
% Setup autoreferences
% ********************************************************************
% There are some issues regarding autorefnames
% http://www.ureader.de/msg/136221647.aspx
% http://www.tex.ac.uk/cgi-bin/texfaq2html?label=latexwords
% you have to redefine the makros for the 
% language you use, e.g., american, ngerman
% (as chosen when loading babel/AtBeginDocument)
% ********************************************************************
\makeatletter
\@ifpackageloaded{babel}%
    {%
       \addto\extrasamerican{%
			\renewcommand*{\figureautorefname}{Figure}%
			\renewcommand*{\tableautorefname}{Table}%
			\renewcommand*{\partautorefname}{Part}%
			\renewcommand*{\chapterautorefname}{Chapter}%
			\renewcommand*{\sectionautorefname}{Section}%
			\renewcommand*{\subsectionautorefname}{Section}%
			\renewcommand*{\subsubsectionautorefname}{Section}%     
                }%
       \addto\extrasngerman{% 
			\renewcommand*{\paragraphautorefname}{Absatz}%
			\renewcommand*{\subparagraphautorefname}{Unterabsatz}%
			\renewcommand*{\footnoteautorefname}{Fu\"snote}%
			\renewcommand*{\FancyVerbLineautorefname}{Zeile}%
			\renewcommand*{\theoremautorefname}{Theorem}%
			\renewcommand*{\appendixautorefname}{Anhang}%
			\renewcommand*{\equationautorefname}{Gleichung}%        
			\renewcommand*{\itemautorefname}{Punkt}%
                }%  
            % Fix to getting autorefs for subfigures right (thanks to Belinda Vogt for changing the definition)
            \providecommand{\subfigureautorefname}{\figureautorefname}%             
    }{\relax}
\makeatother


% ****************************************************************************************************
% 7. Last calls before the bar closes
% ****************************************************************************************************
% ********************************************************************
% Development Stuff
% ********************************************************************
\listfiles
%\PassOptionsToPackage{l2tabu,orthodox,abort}{nag}
%   \usepackage{nag}
%\PassOptionsToPackage{warning, all}{onlyamsmath}
%   \usepackage{onlyamsmath}

% ********************************************************************
% Last, but not least...
% ********************************************************************
\usepackage{classicthesis} 
% ****************************************************************************************************


% ****************************************************************************************************
% 8. Further adjustments (experimental)
% ****************************************************************************************************
% ********************************************************************
% Changing the text area
% ********************************************************************
\linespread{1.05} % a bit more for Palatino
% \areaset[current]{325pt}{680pt} % 686 (factor 2.2) + 33 head + 42 head \the\footskip
%\setlength{\marginparwidth}{7em}%
%\setlength{\marginparsep}{2em}%

% ********************************************************************
% Using different fonts
% ********************************************************************
%\usepackage[oldstylenums]{kpfonts} % oldstyle notextcomp
%\usepackage[osf]{libertine}
%\usepackage[light,condensed,math]{iwona}
%\renewcommand{\sfdefault}{iwona}
%\usepackage{lmodern} % <-- no osf support :-(
%\usepackage{cfr-lm} % 
%\usepackage[urw-garamond]{mathdesign} <-- no osf support :-(
%\usepackage[default,osfigures]{opensans} % scale=0.95 
%\usepackage[sfdefault]{FiraSans}
% ****************************************************************************************************
 % En classicthesis-config.tex se almacenan las opciones propias de la plantilla.

% Color institucional UGR
% \definecolor{ugrColor}{HTML}{ed1c3e} % Versión clara.
\definecolor{ugrColor}{HTML}{c6474b}  % Usado en el título.
\definecolor{ugrColor2}{HTML}{c6474b} % Usado en las secciones.

% Datos de portada
\usepackage{titling} % Facilita los datos de la portada
\author{Luis Antonio Ortega Andrés}
\date{\today}
\title{Statistical Models with Variational Methods}

% Portada
\usepackage{datetime}
\renewcommand\maketitle{
  \begin{titlepage}
    \begin{addmargin}[-2.5cm]{-3cm}
      \begin{center}
        \large  
        \hfill
        \vfill

        \begingroup
        \color{ugrColor}\spacedallcaps{\thetitle} \\ \bigskip
        \endgroup

        \spacedlowsmallcaps{\theauthor}

        \vfill

        Bachelor's Thesis \\ \medskip
        Computer Science and Mathematics \\  \bigskip\bigskip


        \textbf{Tutor}\\
        Serafín Moral Callejón \\ \bigskip

        \spacedlowsmallcaps{Faculty of Science} \\
        \spacedlowsmallcaps{H.T.S. of Computer Engineer and Telecommunications} \\ \medskip
        
        \textit{Granada, \today}

        \vfill                      

      \end{center}  
    \end{addmargin}       
  \end{titlepage}}

\usepackage{wallpaper}

\begin{document}
\ThisULCornerWallPaper{1}{ugrA4.pdf}
\maketitle

\chapter*{Abstract}

Some introduction about how important Variational methods are nowadays and what this project is about.

\tableofcontents

\ctparttext{
  \color{black}
  \begin{center}
    In this chapter we will introduce the underlying concepts of probability and
    graph theory that we will need.
  \end{center}
}
\part{Basic Concepts}

\chapter{Probability}


All our theory will be made under the assumption that there is a
\emph{referential set} \(\Omega\), set of all possible outcomes of an experiment. Any subset of
\(\Omega\) will be called an \emph{event}.

\begin{definition}
Let \(\mathcal{P}(\Omega)\) be the power set of \(\Omega\). Then, \(\mathcal{F} \subset \mathcal{P}(\Omega)\) is called a
\emph{\(\sigma\)-algebra} if it satisfies:
\begin{itemize}
\item \(\Omega \in \mathcal{F}\).
\item \(\mathcal{F}\) is closed under complementation.
\item \(\mathcal{F}\) is closed under countable unions.
\end{itemize}
From these properties it follows that \(\emptyset \in \mathcal{F}\) and that \(\mathcal{F}\)
is closed under countable intersections.

The tuple \((\Omega, \mathcal{F})\) is called a \emph{measurable space}.
\end{definition}

\begin{definition}
A \emph{probability} \(P\) over \((\Omega, \mathcal{F})\) is a mapping
\(P: \mathcal{F} \to [0,1]\) which satisfies
\begin{itemize}
\item \(P(\alpha) \geq 0 \ \ \forall \alpha \in \mathcal{F}\).
\item \(P(\Omega) = 1\).
\item \(P\) is countably additive, that is, if \(\{\alpha_i\}_{i \in \mathbb{N}}
  \subset \mathcal{F}\), is a countable collection of pairwise disjoint sets,
  then
  \[
  P\big(\bigcup_{i\in \mathbb{N}}\alpha_i\big) = \sum_{i\in \mathbb{N}}P(\alpha_i).
  \]
\end{itemize}
\end{definition}

The first condition guarantees non negativity. The second one states that the
\emph{trivial event} has the maximal possible probability of 1.
The third condition implies that given a set of pairwise disjoint events,
the probability of either one of them occurring is equal to the sum of the
probabilities of each one.

From these conditions it follows that
\begin{itemize}
\item \(P(\emptyset) = 0\)
\item \(P(\alpha \cup \beta) = P(\alpha) + P(\beta) - P(\alpha \cap \beta)\)
\end{itemize}

The triple \((\Omega, \mathcal{F}, P)\) is called a \emph{probability space}.

\begin{definition}
  Given two events \(\alpha, \beta \in \mathcal{F}\), with \(P(\beta) \neq 0\),
  the conditional probability of \(\alpha\) given \(\beta\) is defined as the
  quotient of the probability of the joint events and the probability of
  \(\beta\):
  \[
    P(\alpha \mid \beta) = \frac{P(\alpha \cap \beta)}{P(\beta)}
  \]
\end{definition}



\begin{theorem}
  \textbf{(Bayes' theorem)}. Let \(\alpha, \beta\) be two events of an
  experiment, given that \(P(\beta) \neq 0\). Then
  \[
  P(\alpha \mid \beta)= \frac{P(\beta \mid \alpha)P(\alpha)}{P(\beta)}
\]
\end{theorem}



\begin{exampleth}
Consider a study where the relation of a disease \(d\) and an habit \(h\)
is being investigated. Suppose that \(P(d)=10^{-5}\), \(P(h)=0.5\) and \(P(h\mid d) = 0.9\). What is the
probability that a person with habit \(h\) will have the disease \(d\)?

\[
P(d \mid h) = \frac{P(d \cap h)}{P(h)} = \frac{P(h \mid d)P(d)}{P(h)} =
\frac{ 0.9 \times 10^{-5}}{ 0.5 } = 1.8 \times 10^{-5}
\]

If we set the probability of having habit \(h\) to a much lower value as \(P(h) =
0.001\), then the above calculation gives approximately \(1/100\). Intuitively, a smaller number of people have the habit and most of them have the
desease. This means that the relation between having the desease and the habit
is stronger now compared with the case where more people had the habit.
\end{exampleth}

\begin{definition}
  We say that two events \(\alpha, \beta \in \mathcal{F}\) are
  \emph{independent} if knowing one of them does not give any extra information
  about the other. Mathematically,

  \[
    P(\alpha \cap \beta) = P(\alpha)P(\beta) \hspace{2cm} P(\alpha \mid \beta) = P(\alpha)
  \]

  Let \(\gamma \in \mathcal{F}\), we say that \(\alpha\) and \(\beta\) are
  \emph{conditionally independent} on \(\gamma\), \(\alpha \bigCI \beta \mid \gamma\)
  if and only if
  \[
    P(\alpha \cup \beta \mid \gamma) = P(\alpha \mid \gamma)P(\beta \mid \gamma)
  \]
  Otherwise, they are said to be \emph{conditionally dependent} on \(\gamma\),  \(\alpha \bigCD \beta \mid \gamma\).

\end{definition}

Now we are going to introduce the concept of \emph{random variable} and some
properties as we have done with events.

\begin{definition}
A function \(f:\Omega_1 \to \Omega_2\) between two
measurable spaces \((\Omega_1, \mathcal{F}_1)\) and \((\Omega_2, \mathcal{F}_2)\) is said to be \emph{measurable} if \(f^{-1}(\alpha) \in \mathcal{F}_1\) for every \(\alpha \in \mathcal{F}_2\).
\end{definition}

\begin{definition}
  A \emph{random variable} is a measurable function \(X:\Omega \to E\) from a probability
  space \((\Omega, \mathcal{F}, P)\) to a measurable space \((E,
  \mathcal{F}')\) verifying \(X(\omega)\in \mathcal{F}' \ \forall \omega \in \Omega\).

The probability of \(X\) taking a value on a measurable set \(S \in E\) is
written as
\[
P_X(S) = P(X \in S) = P(\{a \in \Omega \ \mid  \ X(a) \in S \}).
\]
\end{definition}

We could make a question like ``How likely is that the value of \(X\) equals
\(a\)?''. This is the same as asking for the probability of the set \(\{\omega
\in \Omega \ \mid  \ X(w) = a\}\).

We will set the following notation that is going to be used, that is: random variables will be
denoted with an upper case letter like \(X\) and a set of variables with a
bold symbol like \(\bm{X}\). The meaning of \(P(state)\) will be clear without a reference to the variable.
Otherwise \(P(X = state)\) will be used.
Using a lower case letter like \(P(x)\) will denote the probability of the
corresponding upper case variable \(X\) taking a specific value.

\begin{definition}
The \emph{cumulative distribution function} of a real-valued random variable \(X\) is the
function given by:
\[
F_X (x) = P(X \leq x)
\]
where the right-hand side represents the probability of the random variable
taking value below or equal to \(x\).
\end{definition}

\begin{definition}
When the image of a random variable \(X\) is countable, the random variable it
is called a
\emph{discrete random variable}, its \emph{probability mass function} \(p\) gives the
probability of it being equal to some value.
\[
p(x) = P(X = x)
\]
If the image is uncountable and real, then \(X\) is called a \emph{continuous random
  variable} if there exists is a non-negative
Lebesgue-integrable \(f\), called its \emph{probability density function} such that
\[
F_X(x) = P(X \leq x) = \int_{-\infty}^x f(u) du
\]

A \emph{mixed random variable} is a random variable who is neither discrete nor
continuous, it can be realized as the sum of a discrete and continuous random
variables. An example of a random variable of mixed type would be based on an
experiment where a coin is flipped and a random positive number is chose only if
the result of the coin toss is heads, $-1$ otherwise.
\end{definition}

From now on, \(P(x)\) will denote \(f_X(x)\) when \(X\) is a continuous random
variable. We will define the \emph{probability distribution} \(P_X\) of a random
variable \(X\) over the probability space \((\Omega, \mathcal{F}, P)\)
as the pushforward measure of it, that is, \(P_X = PX^{-1}\).

As summation is integration with respect to the \emph{counting measure} defined as

\[
 \#(dx) = \sum_{n \in \I}\delta(x - n)dx
\]
Where \(\I\) is the set of values \(X\) can take, and \(\delta\) is the Dirac distribution.
Then

\[
  \int_{x} P(x) \#(dx) = \sum_{n \in \I}\int_{x} P(x) \delta(x-n) dx = \sum_{n \in I}P(n)
\]

Where we used that \(\int f(x)\delta(x - x_{0}) = f(x_{0})\). Given this, from now on, we will use the integration notation for both discrete and continuous variables given that the integrals will be respect to the counting measure when needed.

\begin{definition}
  As we did for events, we can define the \emph{conditional probability} over
  random variables, let \(X, Y\) be random variables,
  \[
    P(x \mid y) = \frac{P(x,y)}{P(y)}
  \]
  It is  required that \(P(y) \neq 0\) for the conditional probability to be defined.
\end{definition}

We can also enunciate the \emph{Bayes' theorem}.

\[
  P(x,y) = \frac{P(y\mid x)P(x)}{P(y)}
\]


\begin{definition}
  The \emph{marginal distribution} of a subset of random variables is the
  probability distribution of the variables contained in that subset.
\end{definition}

Let \(X, Y\) be two random variables, it follows that
\[
  P(x) = \int_y P(x,y)
\]


\begin{definition}
  Let \(\bm{X} = \{X_1, X_2,\dots,X_n\}\) be a set of random variables, the
  \emph{joint probability distribution} for \(\bm{X}\) is function that gives the probability of each random variable \(X_i\)
  falling in a particular range or discrete set of values for that variable. It is
  called a \emph{multi-variate distribution}.

  When using only two random variables, then is called a \emph{bivariate
    distribution}.

  This distribution can be expressed in terms of a joint cumulative distribution
  function
  \[
F_{\bm{X}}(\bm{x}) = F_{X_1,\dots,X_n}(x_1,\dots,x_n) = P(X_1 \leq x_1, \dots,
X_n \leq x_n) \footnote{Where \(\bm{x} = (x_1,\dots,x_n)\)}
\]
or using a probability density function (all variables must be continuous) or a
probability mass function (all variables must be discrete).
\end{definition}

\begin{definition}
We say that two random variables \(X\) and \(Y\) are \emph{independent} if knowing one of them doesn't give any extra information about the other. Mathematically,
\[
P(x,y) = P(x)P(y)
\]
From this it follows that if \(X\) and \(Y\) are independent, then \(P(x\mid y) = P(x)\).
\end{definition}


\begin{definition}
Let \(X,Y\) and \(Z\) be three random variables, then \(X\) and \(Y\) are
\emph{conditionally independent} given \(Z\) if and only if
\[
P(x,y \mid  z) = P(x\mid z)P(y\mid z)
\]
in that case we will denote \(X \bigCI Y \mid Z\). If \(X\) and \(Y\) are not
conditionally independent, they are \emph{conditionally dependent} \(X \bigCD Y \mid Z\)

\end{definition}

Both independence definitions can be made over sets of variables \(\bm{X},
\bm{Y}\) and \(\bm{Z}\) in a straight forward way.


\begin{definition}
  We say that a set of \(n\) random variables \(\{X_1,\dots,X_n\}\) defined to
  assume values in \(I \subset \R\) are
  \emph{independent and identically distributed (i.i.d)}
  if and only if they are independent
  \[
    F_{X_1,\dots,X_n}(x_1,\dots,x_n) = F_{X_1}(x_1)\dots F_{X_n}(x_n) \ \forall
    x_1,\dots,x_n \in I
  \]
  and are identically distributed
  \[
    F_{X_1}(x_1) = F_{X_k}(x_k) \ \forall k \in \{2,\dots,n\} \text{ and } \forall x
    \in I
  \]


\end{definition}


\begin{definition}
  A \emph{multi-variate random variable} or \emph{random vector} is a column vector \(\bm{X} =
  (X_1,\dots,X_n)^T\) whose components are random variables that can be defined
  over different probability spaces.

  Note that we use the same symbol \(\bm{X}\) for random vectors and sets of
  variables, but the meaning will be clear within the context.
\end{definition}


\chapter{Distributions}



In this section we will summarize some concepts concerning probability
distributions among with some of the most used ones.

From now on, let \(X\) be a random variable and \(P\) its probability distribution.

\begin{definition}
  The \emph{mode} \(X_*\) of the probability distribution \(P\) is the state
  of \(X\) where the distribution takes it's highest value
  \[
X_* = \argmax_x P(x)
\]
A distribution could have more than one mode, in this case we say it is \emph{multi-modal}.
\end{definition}

\begin{definition}
  The notation \(\mathbb{E}[X]\) is used to denote the \emph{average} or
  \emph{expectation} of the values a real-valued variable takes respect to its
  distribution.
  If \(X\) is non-negative, it is defined as
  \[
    \mathbb{E}[X] = \int_{\Omega} X(\omega)dP(\omega) \footnote{\(P\) is a
      measure over \(\Omega\)}
  \]
  For a general variable X it is defined as \(\mathbb{E}[X] = \mathbb{E}[X^+] -
  \mathbb{E}[X^-]\). Where
  \[
    X^+(\omega) = \max(X(\omega), 0) \hspace{2cm} X^-(\omega) = \min(X(\omega), 0)
  \]

  Suppose now that \(X\) is a real-valued random variable, in case it is also
  continuous
\[
\mathbb{E}[X] =  \int_{-\infty}^{+\infty} x f(x) dx
\]
and if it is discrete, let \(x_i\) be the values \(X\) can take
\[
\mathbb{E}[X] =  \sum_{i = 1}^{+\infty} x_i p(x_i) dx
\]


Let \(g:\mathbb{R} \to \mathbb{R}\) be a measurable function, then \(g \circ X\)
is another random variable and we can talk about \(\mathbb{E}[g(X)]\), so in
case \(X\) is continuous, we have that
\[
\mathbb{E}[g(X)] =  \int_{-\infty}^{+\infty} g(x) f(x) dx
\]
\end{definition}


\begin{definition}
  We define the \(k^{th}\) moment of a distribution as the average of \(X^k\)
  over the distribution
  \[
    \mu_k = \mathbb{E}[X^k]
  \]
  For \(k = 1\) it is typically denoted as \(\mu\).
\end{definition}


\begin{definition}
  The \emph{variance} of a distribution is defined as
  \[
    Var(X) = \sigma^2 = \mathbb{E}[(X - \mathbb{E}[X])^2] = \mathbb{E}[X]^2 - \mathbb{E}[X^2]
  \]

  The square root of the variance \(\sigma\) is called the \emph{standard deviation}.
\end{definition}

When using a multivariate distribution \(\bm{X} = (X_1,\dots,X_n)^T\) we can talk about the \emph{covariance
  matrix} \(\bSigma \) whose elements are

\[
\begin{aligned}
\bSigma_{ij} &= \mathbb{E}[ (X_i - \mathbb{E}[X_i])
(X_j - \mathbb{E}[X_j]) ] = \mathbb{E}[(X_i - \mu_i)(X_j - \mu_j)]\\
&=\mathbb{E}[X_iX_j]-\mathbb{E}[X_i]\mathbb{E}[X_j]
\end{aligned}
\]

The following result will be helpful later on

\begin{proposition}
  Let \(\mathcal{X} = \{X_1,\dots,X_n\}\) be a set of random variables,
  \(\mathcal{X}_{0} \subset \mathcal{X}\) and \(P(\mathcal{X}), P(\mathcal{X}_{0})\)
  their probability distributions.
  It follows that the expectation of a function \(g\) over \(\mathcal{X}_0\), verifies
    \[
      \E[g(\mathcal{X}_0)]_{P(\mathcal{X})} = \E[g(\mathcal{X}_0)]_{P(\mathcal{X}_0)}
    \]
    that is, we only need to know the marginal distribution of the subset in
    order to carry out the average.
\end{proposition}

\begin{proof}
  Let \(\I = (i_{1}, \dots, i_{k})\) be the indexes corresponding to \(\mathcal{X}_{0}\), then
  \[
    \begin{aligned}
      \E[g(\mathcal{X}_{0})]_{P(\mathcal{X})} &= \int_{x_{1}}\dots\int_{x_{n}}g(x_{i_{1}},\dots,x_{i_{k}})f(x_{1},\dots,x_{n}) \\
      &= \int_{x_{i_{1}}}\dots\int_{x_{i_{k}}} g(x_{i_{1},\dots,x_{i_{k}}}) \int \dots \int f(x_{1},\dots,x_{n}) d_{x_{1}},\dots,d_{x_{n}}\\
      &= \int_{x_{i_{1}}}\dots\int_{x_{i_{k}}} g(x_{i_{1},\dots,x_{i_{k}}}) f(x_{i_{1}},\dots,x_{i_{k}}) =  \E[g(\mathcal{X}_{0})]_{P(\mathcal{X}_{0})} \\
      \end{aligned}
  \]
Where in the second-last equality we used marginalization.
\end{proof}

We are going to discuss now some examples of probability distributions that are
going to be used from now on.

\section{Discrete Distributions}

\subsection{Bernoulli distribution}

The Bernoulli distribution describes a discrete binary variable \(X\) that takes
the value 1 with probability \(p\) and the value 0 with probability \(1-p\).
\[
  p(x) =
\left\{
  \begin{array}{ll}
    p  & \mbox{if } x = 1 \\
    1-p & \mbox{if } x = 0
  \end{array}
\right.
\]


\subsection{Binomial distribution}

The binomial distribution describes the number of successes in a sequence of
independent Bernoulli Trials. A discrete binary random variable \(X\) follows a
\emph{binomial distribution} of parameters \(n \in \mathbb{N}\) and \(p \in
[0,1]\), denoted as \(X \sim B(n, p)\) if and only if
\[
  P(x = k) = \binom{n}{k}p^k(1-p)^{n-k}
\]


\section{Continuous Distributions}

\subsection{Univariate Normal distribution}

The \emph{normal} or \emph{Gaussian distribution} is a type of continuous
probability distribution for real-valued random variables.

\begin{definition}
  We say the real valued random variable \(X\) follows a \emph{normal distribution} of
  parameters \(\mu, \sigma \in \mathbb{R}\), denoted as \(X \sim N(\mu,
  \sigma)\) if and only if, its probability density function exists and is
  \[
    f(x) = \frac{1}{\sigma \sqrt{2\pi}} e^{-\frac{1}{2}\big(\frac{x-\mu}{\sigma} \big)^2}
  \]

  The parameter \(\mu\) is the mean or expectation of the distribution and
  \(\sigma\) is its standard deviation.

\end{definition}

The simplest case of a normal distribution is known as \emph{standard normal
  distribution}, denoted as \(Z\). It is a special case where \(\mu = 0\) and \(\sigma = 1\), then
its density function is
\[
  f(x) = \frac{1}{\sqrt{2\pi}}e^{-\frac{1}{2}x^2}
\]

One of the properties of the normal distribution is that if \(X \sim N(\mu, \sigma)\), \(a,b \in \mathbb{R}\) and \(f:\mathbb{R} \to \mathbb{R}\) be defined
as \(f(x) = ax + b\), then \(f(X) \sim N(\mu + b, a^2 \sigma)\)

\subsection{Multivariate Normal Distribution}

This distribution plays a fundamental role in this project so we will discuss
its properties in more detail.

This distribution is an extension of the uni-variate one when having a
multivariate random variable.

\begin{definition}
We say that a random vector \(\bm{X} = (X_1,\dots,X_p)\) follows a \emph{multivariate normal
  distribution} of parameters \(\bm{\mu} \in \mathbb{R}^p\) and \(\bm{\Sigma}
\in \mathbb{M}_n(\mathbb{R})\), denoted as \(\bm{X} \sim N(\bm{\mu},
\bm{\Sigma})\) if and only if its probability density function is

\[
  f(\bm{x}) = \frac{1}{\sqrt{det(2\pi \bm{\Sigma})}}e^{-\frac{1}{2}(\bm{x} - \bm{\mu})^T\bm{\Sigma}^{-1}(\bm{x}-\bm{\mu})}
\]

Where \(\bm{\mu}\) is the mean vector of the distribution, and \(\bm{\Sigma}\)
the covariance matrix. The inverse matrix \(\bm{\sigma}^{-1}\) is called \emph{precision}.
It also satisfies that
\[
\bm{\mu} = \mathbb{E}[\bm{X}] \hspace{2cm} \bSigma = \E[(\bx - \bmu)(\bx - \bmu)^T]
\]

As \(\bSigma\) is a real symmetric matrix, it can be eigendecomposed
\[
  \bSigma = \bm{E}\bm{\Delta}\bm{E}^T
\]
where \(\bm{E}^T\bm{E} = \bm{I}\) and \(\bm{\Delta} =
diag(\lambda_1,\dots,\lambda_n)\).

Using the transformation
\[
  \bm{y} = \bm{\Delta}^{\frac{1}{2}}\bm{E}^T(\bx - \bmu)
\]
we get that

\[
  (\bx - \bmu)^T\bSigma(\bx - \bmu) = \bm{y}^T\bm{y}
\]

Using this, the multivariate Normal Distribution reduces to a product of \(n\)
univariate standard normal distributions.

\todo[inline]{TODO: Add more properties as they are needed}

\end{definition}

\subsection{Beta Distrbution}

Another continuous distribution that we are going to use is the \emph{Beta
  distribution}.

\begin{definition}
We say that a continuous random variable \(X\) defined on the
interval \([0,1]\) follows a \emph{Beta distribution} of parameters \(\alpha,
\beta > 0\), denoted as \(X \sim Beta(\alpha, \beta)\) if and only if its
density function is

\[
  f(x) = \frac{1}{B(\alpha, \beta)}x^{\alpha - 1}(1-x)^{\beta -1}
\]
where \(B(\alpha, \beta)\) is the \emph{beta function} defined as
\[
  B(\alpha, \beta) = \int_0^1 x^{\alpha - 1}(1-x)^{\beta -1} dx
\]
\end{definition}

The mean is given by \(\E[X] = \frac{\alpha}{\alpha + \beta}\)

\subsection{Dirichlet Distribution}

The Dirichlet distribution is a family of continuous multivariate probability
distributions parameterized by a vector \(\bm{\alpha}\) of positive reals. It is
a multivariate generalization of the Beta Distribution.

\begin{definition}
  We say that a continuous random multivariate variable \(\bX\) with order
  \(K \geq 2\), follows a \emph{Dirichlet
    Distribution} with parameters \(\bm{\alpha} = (\alpha_{1}, \dots, \alpha_{K})\), if and
  only if its density function is defined as
  \[
    f(\bx) = \frac{1}{B(\bm{\alpha})}\prod_{k = 1}^{K}x_{k}^{\alpha_{k}-1}
  \]
  and it satisfies that
  \[
    \sum_{k=1}^{K} x_{k} = 1 \text{ and } x_{k} > 0 \ \forall k=1,\dots,K
  \]
\end{definition}

Where the normalization constant is the beta function
\[
  B(\bm{\alpha}) = \frac{\prod_{k} \Gamma (\alpha_{k})}{\Gamma \big( \sum_{k}\alpha_{k} \big)}
\]



\section{Kullback-Leibler Divergence}

\begin{definition}
  Let \(P\) and \(Q\) be two probability distributions over the same set of
  random variables \(\bm{X}\), the \emph{Kullback-Leibler divergence}
  \(KL(Q|P)\) measures the `difference' between both distributions as
  \[
    KL(Q|P) = \mathbb{E}[\log Q(x) - \log P(x)]_Q
  \]
\end{definition}

\begin{proposition}
The Kullback-Leibler divergence is always non-negative.
\end{proposition}
\begin{proof}
  As the logarithm is bounded by \(x - 1\), we can bound \(\log{\frac{P(x)}{Q(x)}}\)
  \[
    \log{x} \leq x - 1 \implies \frac{P(x)}{Q(x)} - 1 \geq \log{\frac{P(x)}{Q(x)}}
  \]

  Since probabilities are non-negative, we can multiply by \(Q(x)\)
  \[
    P(x) - Q(x) \geq Q(x) \log{P(x)} - Q(x) \log{Q(x)}
  \]
  Now we integrate (sum in case of discrete variables) both sides

  \[
    0 \geq \mathbb{E}[\log{P(x)} - \log{Q(x)}]_Q \implies \mathbb{E}[\log{Q(x)}
    - \log{P(x)}]_Q \geq 0
  \]
\end{proof}
As a result, the Kullback-Leibler divergence is \(0\) if and only if the two
distributions are equal almost everywhere.



\chapter{Graph Theory}


\begin{definition}
A \emph{graph} \(G = (V,E)\) is a set of vertices or nodes \(V\) and edges \(E\subset
V\times V\) between them.
If \(V\) is a set of ordered pairs then the graph is called a \emph{directed
  graph}, otherwise if \(V\) is a set of unordered pairs it is called an \emph{undirected graph}.
\end{definition}

\begin{figure*}[h]
\centering
\begin{tikzpicture}[
  node distance=1cm and 0.5cm,
  mynode/.style={draw,circle,text width=0.5cm,align=center}
]

\node[mynode] (a) {A};
\node[mynode,below right=of a] (b) {B};
\node[mynode,above right=of b] (c) {C};

\node[mynode, right=of c] (d) {A};
\node[mynode,below right=of d] (e) {B};
\node[mynode,above right=of e] (f) {C};

\path (c) edge[-latex] (a)
(a) edge[-latex] (b)
(b) edge[latex-] (c);

\draw (d) -- (e) -- (f) -- (d);

\end{tikzpicture}
\caption{Example of directed and undirected graph, respectively.}
\label{fig:graphs}
\end{figure*}

\begin{definition}
In a directed graph \(G = (V, E)\), a \emph{directed path} \(A \to B\) is a sequence of vertices \({A = A_0,
  A_1,\dots,A_{n-1}, A_n = B}\) where \((A_i, A_{i+1}) \in E \ \forall i \in
0,\dots ,n-1\).

If \(G\) is a undirected graph, \(A \to B\) is an \emph{undirected path} if \(\{A_i, A_{i+1}\} \in E \ \forall i \in
0,\dots, n-1\)
\end{definition}

\begin{definition}
Let \(A,B\) be two vertices of a directed graph \(G\). If \(A \to B\) is a
directed path and \(B \not \to A\) (meaning there isn't a directed path from
\(B\) to \(A\)), then \(A\) is called an \emph{ancestor} of \(B\) and \(B\) is called a \emph{descendant} of \(A\).
\end{definition}

For example, in the figure \ref{fig:graphs}, \(C\) is an ancestor of \(B\).

\begin{definition}
A \emph{directed acyclic graph (DAG)} is a directed graph such that no directed path between any two nodes revisits a vertex.
\end{definition}


\begin{figure}[h]
\centering
\begin{tikzpicture}[
  node distance=1cm and 0.5cm,
  mynode/.style={draw,circle,text width=0.5cm,align=center}
]

\node[mynode] (a) {A};
\node[mynode,below right=of a] (b) {B};
\node[mynode,above right=of b] (c) {C};

\path (c) edge[-latex] (a)
(a) edge[-latex] (b)
(b) edge[-latex] (c);

\end{tikzpicture}
\captionof{figure}{Example of graph which isn't a DAG.}
\label{fig:not_dag}
\end{figure}

As we can see in the figure \ref{fig:not_dag}, \(A \to B \to C \to A \to B\) is a
path from \(A\) to \(B\) that revisits \(A\).

Now where are going to define some relations between nodes in a DAG.

\begin{definition}
The \emph{parents} of a node \(A\) is the set of nodes \(B\) such that there is a
directed edge from \(B\) to \(A\). The same applies for the \emph{children} of a node.

The \emph{Markov blanket} of a node is composed by the node itself, its children, its parents and the parents
of its children.
\end{definition}


\begin{figure}[h]
\centering
\begin{tikzpicture}[
  node distance=1cm and 0.5cm,
  mynode/.style={draw,circle,text width=0.5cm,align=center}
]

\node[mynode] (a) {A};
\node[mynode,below right=of a] (b) {B};
\node[mynode,above right=of b] (c) {C};
\node[mynode,below right=of b] (d) {D};
\node[mynode,below left=of b] (e) {E};
\node[mynode,above right=of d] (f) {F};
\node[mynode, above right=of f] (h) {H};

\path (c) edge[-latex] (a)
(a) edge[-latex] (b)
(b) edge[latex-] (c)
(b) edge[-latex] (e)
(c) edge[-latex] (f)
(b) edge[-latex] (d)
(f) edge[-latex] (d)
(h) edge[-latex] (f)
;

\end{tikzpicture}
\captionof{figure}{Directed acyclic graph}
\label{fig:relations}
\end{figure}

\begin{definition}
In a graph, the \emph{neighbors} of a node are those directly connected
to it.
\end{definition}

We can use figure \ref{fig:relations} to reflect on these definitions. The parents
of \(B\) are \(pa(B) = \{A,C\}\) and its children are \(ch(B) = \{E,D\}\). Taking this into account, its neighbors
are \(ne(B) = \{A,C,E,D\}\) and its Markov blanket is \(\{A,B,C,D,E,F\}\).

\begin{definition} Let \(G\) be a DAG, \(U\) be a path between two vertex and \(A \in U\)
  \begin{itemize}
  \item \( A \) is called a \emph{collider} if \(\forall B \in ne(A)\cap U, (B,A)\in
    E\).
  \item \( A \) is called a \emph{fork} if \(\forall B \in ne(A) \cap U, (A,B)\in
    E\).
  \end{itemize}
  Notice, a vertex can be a collider for a path but not for others.

  For example in figure \ref{fig:relations}, \(D\) is a collider and \( C \) is
  a fork.
\end{definition}

\begin{definition}
Let \(G\) be an undirected graph, a \emph{clique} is a maximally connected
subset of vertices. That is, all the members of the clique are connected to each
others and there is no bigger clique that constains another.

Formally, \(S \subset V\) is a \emph{clique} if and only if \(\forall A,B \in S,
\ \{A,B\} \in E\) and \(\nexists C \in V\backslash S\) such that \(\forall A \in
S, \ \{A, C\} \in E \).
\end{definition}



\ctparttext{
  \color{black}
  \begin{center}

A \emph{graphical model} is a statistical model for which a graph expresses the
conditional dependence structure between random variables.

Commonly, they provide a graph-based representation for encoding a multi-dimensional
distribution representing a set of independences that hold in the specific
distribution. The most commonly used are \emph{Bayesian networks} and \emph{Markov random
fields}, which differ in the set of independences they can encode and the
factorization of the distribution that they include.  \end{center}
}
\part{Graphical Models}

\chapter{Bayesian networks}

Consider we have \(N\) variables with the corresponding distribution
\(P(x_1,\dots,x_N)\). Let \(\mathcal{E}\) be a set of indexes such as \texttt{evidence}
\(=\{X_e = x_e \ | \ e \in \mathcal{E}\}\). Inference could be made by brute
force:

\[
P(X_i = x_i \ | \ \texttt{evidence}) = \frac{ \int_{ j \not \in
\mathcal{E}, j \neq i } P(\texttt{evidence}, x_j, X_i = x_i)}{ \int_{ j
\not \in \mathcal{E} } P(\texttt{evidence}, x_j)}
\]

The notation when using discrete variables is analogous replacing integration
with summations.

Lets suppose all these variables are binary, this calculation will require
\(O(2^{N-\#\mathcal{E}})\) operations. Also, all entries of a table \(P(x_1,\dots,
x_N)\) take \(O(2^N)\) space.

This is unpractical when taking into account millions of variables. The
underlying idea of belief networks is to specify which variables are independent
of others, factoring the joint probability distribution.

\begin{definition}
Let \(G=(V,E)\) be a graph where \(V = \{X_1,\dots,X_n\}\) is a set of random
variables. We say that the joint
probability \(P(x_1, \dots, x_n)\) \emph{factorizes} according to \(G\) if and
only if
\[
P(x_1,\dots,x_N) = \prod_{i=1}^{N}P(x_i | pa(x_i))
\]
\end{definition}

\begin{definition}
A \emph{belief network or Bayesian network} is a pair \((G, P)\)
where \(P\) factorizes over \(G\). It is a probabilistic graphical model
that represents conditional dependencies of a set of variables \(X_1,\dots, X_n\).
\end{definition}

\begin{figure}
  \centering
  \begin{tikzpicture}[
    node distance=1.5cm and 1.5cm,
    mynode/.style={draw,circle,text width=0.5cm,align=center}
    ]

    \node[mynode] (1) {\(X_1\)};
    \node[mynode,right=of 1] (2) {\(X_2\)};
    \node[mynode,right=of 2] (3) {\(X_3\)};
    \node[mynode,right=of 3] (4) {\(X_4\)};

    \path (4) edge[-latex][bend right] (1)
    (3) edge[-latex] (2)
    (4) edge[-latex][bend right] (2)
    ;

    \end{tikzpicture}
    \captionof{figure}{Bayesian Network factorizing \(P(x_1, x_2, x_3, x_4) = P(x_1 | x_4)P(x_2| x_3, x_4)P(x_3)P(x_4)\)}
    \label{fig:bn_example}
\end{figure}


Any probability distribution can be written as a Bayesian Network, even though
it may end up been a fully-connected DAG.
To set the specification of the Belief Network, we need to define all elements of the probability
tables \(P(x_i|pa(x_i))\). When the number of variables is large, this is still
intractable so the tables are generally parameterized is a low dimensional
manner.

Bayesian Networks are good for encoding conditional independence over the
variables, but aren't for encoding dependence. For example, with the following
network \(P(x,y) = P(y|x)P(x)\) represented as \(x \to y\) in a DAG.
It may appear to encode dependence between both variables but the
conditional \(P(y|x)\) could happen to equal \(P(y)\), giving \(P(x,y) = P(x)P(y)\).

How could we check if two variables are conditionally independent given a
Bayesian Network? For example in figure \ref{fig:relations}, \(X_1 \bigCI
X_2 \mid X_4\) as\footnote{Continuous variable notation is used}:
\[
\begin{aligned}
P(x_2 | x_4) &= \frac{1}{P(x_4)}\int_{x_1,x_3}P(x_1, x_2, x_3, x_4)
= \frac{1}{P(x_4)}\int_{x_1,x_3}P(x_1|x_4)P(x_2|x_3,x_4)P(x_3)P(x_4)\\
                 &= \int_{x_3}P(x_2|x_3, x_4)P(x_3)
\end{aligned}
\]
\[
\begin{aligned}
P(x_1, x_2 | x_4) &= \frac{1}{P(x_4)}\int_{x_3}P(x_1, x_2, x_3, x_4)
= \frac{1}{P(x_4)}\int_{x_3}P(x_1|x_4)P(x_2|x_3,x_4)P(x_3)P(x_4)\\
                 &= P(x_1|x_4)\int_{x_3}P(x_2|x_3, x_4)P(x_3) = P(x_1|x_4)P(x_2|x_4)
\end{aligned}
\]

Now we are going to define two central concepts to determine conditional
independence in any Bayesian Network, these are \emph{d-connection} and \emph{d-separation}.

\begin{definition}
Let \(G\) be a DAG where \(\bm{X}, \bm{Y} \text{ and } \bm{Z}\)
are disjoint sets of vertices. We say that \(\bm{X} \text{ and
} \bm{Y}\) are \emph{d-connected} by \(\bm{Z}\) if and only if there
exists an undirected path \(U\) from any vertex in \(\bm{X}\) to any
vertex in \(\bm{Y}\) such that:
\begin{itemize}
\item For any collider \(C\), itself or any it's descendants is in \(\bm{Z}\)
\item No non-collider on \(U\) is on \(\bm{Z}\)
\end{itemize}
\end{definition}

\begin{definition}
Let \(G\) be a DAG where \(\bm{X}, \bm{Y} \text{ and } \bm{Z}\)
are disjoint sets of vertices. \(\bm{X}\) and \(\bm{Y}\)
are \emph{d-separated} by \(\bm{Z}\) if and only if they are not
d-connected by \(\bm{Z}\) in \(G\)
\end{definition}

\begin{figure}[h]
\centering
\begin{tikzpicture}[
  node distance=1cm and 0.5cm,
  mynode/.style={draw,circle,text width=0.5cm,align=center}
]

\node[mynode] (a) {a};
\node[mynode, below right=of a] (d) {d};
\node[mynode,above right=of d] (b) {b};
\node[mynode, below right=of b] (e) {e};
\node[mynode,above right=of e] (c) {c};

\path (a) edge[-latex] (d)
(b) edge[-latex] (d)
(c) edge[-latex] (e)
(b) edge[-latex] (e)
;

\end{tikzpicture}
\caption{D-separation example}
\label{fig:d-sep}
\end{figure}

For example, in figure \ref{fig:d-sep} \(d\) d-separates \(a\) and \(c\) (\(b\)
is a collider in the path that isn't in \(\{d\}\)),
and \(\{d,e\}\) d-connect them.

\begin{theorem}[Verma and Pearl, 1988,  Geiger et al.,
1990 \cite{pearl_and_detcher}]
Let \(G\) be a DAG where \(\bm{X}, \bm{Y} \text{ and } \bm{Z}\)
are disjoint sets of vertices. If  \(\bm{X}\) and \(\bm{Y}\)
are d-separated by \(\bm{Z}\), then they are independent conditional
on \(\bm{Z}\) in all probability distributions that G can represent.
\end{theorem}

The Bayes Ball algorithm \cite{bayes_ball} provides a linear time complexity
algorithm that computes conditional independent using this theorem.


\begin{exampleth}
In this example we are modeling three discrete random variables: Sprinkler (\(S\)),
Rain (\(R\)) and Grass wet (\(G\)).

The joint probability function is:
\[
P(s,r,g) = P(s|r)P(g|s,r)P(r)
\]

The following DAG illustrates the Bayesian Network among with the probability
tables we are using.

\begin{tikzpicture}[
  node distance=1cm and 0cm,
  mynode/.style={draw,ellipse,text width=2cm,align=center}
]
\node[mynode] (sp) {Sprinkler};
\node[mynode,below right=of sp] (gw) {Grass wet};
\node[mynode,above right=of gw] (ra) {Rain};
\path (ra) edge[-latex] (sp)
(sp) edge[-latex] (gw)
(gw) edge[latex-] (ra);
\node[left=0.5cm of sp]
{
\begin{tabular}{cm{1cm}m{1cm}}
\toprule
& \multicolumn{2}{c}{Sprinkler} \\
Rain & \multicolumn{1}{c}{T} & \multicolumn{1}{c}{F} \\
\cmidrule(r){1-1}\cmidrule(l){2-3}
F & 0.4 & 0.6 \\
T & 0.01 & 0.99 \\
\bottomrule
\end{tabular}
};
\node[right=0.5cm of ra]
{
\begin{tabular}{m{1cm}m{1cm}}
\toprule
\multicolumn{2}{c}{Rain} \\
\multicolumn{1}{c}{T} & \multicolumn{1}{c}{F} \\
\cmidrule{1-2}
0.2 & 0.8 \\
\bottomrule
\end{tabular}
};
\node[below=0.5cm of gw]
{
\begin{tabular}{ccm{1cm}m{1cm}}
\toprule
& & \multicolumn{2}{c}{Grass wet} \\
\multicolumn{2}{l}{Sprinkler Rain} & \multicolumn{1}{c}{T} & \multicolumn{1}{c}{F} \\
\cmidrule(r){1-2}\cmidrule(l){3-4}
F & F & 0.0 & 1.0 \\
F & T & 0.8 & 0.2 \\
T & F & 0.9 & 0.1 \\
T & T & 0.99 & 0.01 \\
\bottomrule
\end{tabular}
};
\end{tikzpicture}

This model can answer questions about the presence of a cause given the presence
of an effect. For example, What is the probability that it has being raining
given the grass is wet?

\[
P(R = T | G = T) = \frac{P(G = T, R = T)}{P(G=T)} = \frac{\sum_{s}P(G=T, R=T,
s)}{\sum_{r,s} P(G=T, r, s)}
\]

Using the expression of the joint probability among with the tables we can
compute every term. For example:
\[
\begin{aligned}
P(G=T, R=T, S=T) &= P(S=T|R=T)P(G=T|R=T,S=T)P(R=T) \\
&= 0.01 * 0.99 * 0.2 = 0.00198
\end{aligned}
\]
\end{exampleth}


In some situations our Belief Networks will contain a number of nodes that are
essentially the same but repeated a number of times, for this, we are going to
introduce the \emph{plate notation}. Suppose we have the situation that figure
\ref{fig:plate_notation} shows on the left. The we can collapse all \(B_i\)
variables in a box, indicating there number of variables inside it.

\begin{figure}[h]
\centering
\begin{tikzpicture}[
  node distance=1cm and 0.5cm,
  mynode/.style={draw,circle,text width=0.5cm,align=center}
]

\node[mynode] (a) {A};
\node[mynode,below=of a] (d) {\(B_1\)};
\node[mynode,left=of d] (c) {\(B_2\)};
\node[mynode,left=of c] (b) {\(B_3\)};
\node[mynode,right=of d] (e) {\(\dots\)};
\node[mynode,right=of e] (f) {\(B_n\)};

\node[mynode,right=2cm of f] (g) {\(B_i\)};
\node[mynode, above=of g] (h) {A};
\plate{} {(g)} {\(n\)}; %


\path (a) edge[-latex] (b)
(a) edge[-latex] (c)
(a) edge[-latex] (d)
(a) edge[-latex] (e)
(a) edge[-latex] (f)
(h) edge[-latex] (g)
;

\end{tikzpicture}
\caption{Plate notation example. Standard notation on the left and plate on the right}
\label{fig:plate_notation}
\end{figure}


\chapter{Markov Random Fields}



\begin{definition}
A \emph{potential} \(\phi\) is a non-negative function. It is worth to mention
that a probability distribution is a special case of a potential.
\end{definition}

\begin{definition}
Let \(\bm{X}\) be a set of random variables, \(G\) an undirected graph,
\(\bm{X}_c, c \in \{1,\dots,C\}\) be the maximal cliques of \(G\) and \(P\) a
probability distribution over \(\bm{X}\). The pair \((G,
P)\) is said to be a \emph{Markov network or Markov random field} if, and only if
\[
P(x_1,\dots,x_n) = \frac{1}{Z}\prod_{ c = 1 }^{C}\phi_c(\bm{X}_c)
\]
where \(Z\) is a constant that ensures normalization.
\end{definition}

\begin{figure}[h]
\centering
\begin{tikzpicture}[
  node distance=1cm and 1.5cm,
  mynode/.style={draw,circle,text width=0.5cm,align=center}
]

\node[mynode] (1) {\(X_1\)};
\node[mynode,right=of 1] (2) {\(X_2\)};
\node[mynode,below=of 1] (3) {\(X_3\)};
\node[mynode,right=of 3] (4) {\(X_4\)};
\path (1) edge (2)
(1) edge (3)
(2) edge (3)
(2) edge (4)
(1) edge (4)
;
\end{tikzpicture}
\captionof{figure}{Markov Network \(P(x_1, x_2, x_3, x_4) = \phi(x_1, x_2,
  x_3)\phi(x_2, x_3, x_4)/Z\)}
\label{fig:mn_example}
\end{figure}


In figure \ref{fig:mn_example} we can see an example of the factorization, without
  giving any reference of the potentials.

  Let \((G,P)\) be a Markov network, then it satisfies the following properties
  known as Markov properties:
  \begin{itemize}
  \item Pairwise Markov property. Any two non-adjacent variables are
    conditionally independent given all other variables.
  \item Local Markov property. A variable is conditionally independent over all
    other variables given it's neighbors. That is,
        \[
      P(x_i | x_{\backslash i}) = p(x_i | ne(x_i)) \footnote{\(X_{\backslash i} =
        \{X_j \ | \ j \neq i\} \subset \bm{X} \) }
    \]
  \item Global Markov property. Any two subsets of variables are conditionally
    independent given a separating subset (any path from one set to the other
    passes trough this one).
  \end{itemize}

  \begin{remark}
    A Markov network can also be defined as a pair \((G, P)\) such as all Markov
    properties are satisfied. The clique factorization definition is a special
    case of these properties.
  \end{remark}


  \begin{definition}
    Let \(G\) be an undirected graph and \(P\) a probability distribution over a
    set of random variables \(\bm{X}\). The pair \((G, P)\) is called a Markov
    Random Field if and only if it follows the local Markov Property.


  \end{definition}



\ctparttext{
  \color{black}
  \begin{center}

  \end{center}
}
\part{Name this part}

\chapter{Learning as Inference}



In Machine Learning and related fields, the distributions are not fully specified
and need to be learned from the data.

From now on, \(\mathcal{V}\) will denote the known data and \(\theta\) the set
of parameters of the data distributions. The main task is to determine this set
of parameters using the information given by the data.

\begin{definition}
\emph{Priors} and \emph{posteriors} typically refer to the parameter
distribution before and after seeing the data, respectively. Using Bayes' rule
\[
  P(\theta \mid  \mathcal{V}) = \frac{P(\mathcal{V}  \mid  \theta)P(\theta)}{P(\mathcal{V})}
\]
The factor \(P(\mathcal{V} \mid \theta)\) is called the \emph{likelihood}.
\end{definition}

Let us see an example of our goal, in it we will try to learn the bias of a coin,
given a set of tossing results.

\begin{exampleth}
  Let \(\V = \{v_n\}_{n \in 0,\dots,N}\) be the results of tossing a coin \(N \in
  \mathbb{N}\) times, let \(1\) symbolize \emph{heads} and \(0\) \emph{tails}.

  Our objective is to estimate the probability \(\theta\) that the coin will be
  head \(P(v_n = 1  \mid  \theta)\), for this we have the i.i.d random variables \(v_1,\dots,v_n\)
  and \(\theta\), and we require a model \(P(v_1,\dots,v_n,\theta)\). We have a
  Belief Network shown in figure \ref{fig:learning_coin}
  \[
    P(\V,\theta) = P(\theta)\prod_{n=1}^N P(v_n \mid \theta)
  \]

\begin{figure}[H]
\centering
\begin{tikzpicture}[
  node distance=1cm and 0.5cm,
  mynode/.style={draw,circle,text width=0.5cm,align=center}
]

\node[mynode] (a) {\(\theta\)};
\node[mynode, below=of a] (b) {\(v_i\)};
\plate{} {(b)} {\(N\)}; %
\path (a) edge[-latex] (b)
;

\end{tikzpicture}
\caption{Belief network for coin tossing}
\label{fig:learning_coin}
\end{figure}

We want to calculate
\[
  P(\theta \mid \V) = \frac{P(\V \mid \theta)P(\theta)}{P(\V)}
\]
to do so, we need to specify the prior \(P(\theta)\), we are using a discrete
model where
\[
  P(\theta = 0.2) = 0.1 \hspace{2cm} P(\theta = 0.5) = 0.7 \hspace{2cm} P(\theta =
  0.8) = 0.2
\]
This means that we have a \(70\%\) belief that the coin is fair, a \(10\%\)
belied that is biased to tails and \(20\%\) that is biased to heads.
Notice that \(P(v_n = 1 \mid \theta) = \theta\) and \(P(v_n = 0 \mid \theta) = 1 - \theta\).

Let \(n_h\) be the number of heads in our observed data and \(n_t\)
the number of tails
\[
  n_{h} = \#\{v= 1\} \hspace{2cm} n_{t} = \#\{v = 0\}
\]

then the posterior has the form

\[
  P(\theta  \mid \V) = \frac{P(\theta)}{P(\mathcal{V})} \theta^{n_h}(1-\theta)^{n_t}
\]

Suppose now that \(n_h = 2\) and \(n_t = 8\), then
\begin{gather*}
  P(\theta = 0.2  \mid  \mathcal{V}) = \frac{1}{P(\mathcal{V})}\times 0.1 \times 0.2^{2}
  \times 0.8^{8} = \frac{1}{P(\mathcal{V})} \times 6.71\times10^{-4} \\
   P(\theta = 0.5  \mid  \mathcal{V}) = \frac{1}{P(\mathcal{V})}\times 0.7 \times 0.5^{2}
   \times 0.5^{8} = \frac{1}{P(\mathcal{V})} \times 6.83\times10^{-4}\\
    P(\theta = 0.8  \mid  \mathcal{V}) = \frac{1}{P(\mathcal{V})}\times 0.2 \times 0.2^{2}
  \times 0.8^{8} = \frac{1}{P(\mathcal{V})} \times 3.27\times10^{-7}
\end{gather*}

Now, we can compute
\[
   \frac{1}{P(\mathcal{V})} =  6.71\times10^{-4} +   6.83\times10^{-4} +
   3.27\times10^{-7} = 0.00135
 \]
 So,
\begin{gather*}
  P(\theta = 0.2  \mid  \mathcal{V}) = 0.4979\\
  P(\theta = 0.5  \mid  \mathcal{V}) = 0.5059\\
  P(\theta = 0.8  \mid  \mathcal{V}) = 0.00024
\end{gather*}

These are the posterior parameter beliefs of our experiment. Given this, it we
were to choose a single value for the posterior it would be \(\theta = 0.5\).
This result is intuitive, we had a strong belief of the coin being fair
and even though the number of tails was quite bigger than heads, it
was not enough to make the difference. Obviously the posterior of the coin being
biased to tails is now bigger than the prior.

Suppose an uniform prior distribution so that \(P(\theta) = k \implies \int_0^1 P(\theta) d\theta
= k = 1\) due to normalization.

Using the previous calculations we have
\[
  P(\theta \mid  \mathcal{V}) = \frac{1}{P(\mathcal{V})} \theta^{n_h}(1-\theta)^{n_t}
\]
where
\[
  P(\mathcal{V}) = \int_0^1 \theta^{n_h}(1-\theta)^{n_t} d\theta
\]
this implies that
\[
  P(\theta \mid \mathcal{V}) = \frac{\theta^{n_h}(1-\theta)^{n_t} }{ \int_0^1 u^{n_h}(1-u)^{n_t}
    du} \implies \theta \mid \V \sim Beta(n_h + 1, n_t + 1)
\]
\end{exampleth}

\begin{definition}
If the posterior distribution is in the same probability distribution family as
the prior distribution, they are then called \emph{conjugate distributions}, and
the prior is called a \emph{conjugate prior} of the likelihood distribution.
\end{definition}

Let's use a Beta distribution as the prior in the last example

\[
  \theta \sim \text{Beta}(\alpha, \beta) \implies P(\theta) = \frac{1}{B(\alpha, \beta)}\theta^{\alpha - 1}(1 - \theta)^{\beta -
    1}
\]
then, repeating the same as before we get that
\[
  P(\theta, \mathcal{V}) = \frac{1}{B(\alpha + n_h, \beta + n_t)}\theta^{\alpha
    + n_h - 1}(1 - \theta)^{\beta + n_t - 1} \implies (\theta, \V) \sim Beta(\alpha + n_h, \beta + n_t)
\]

So both the prior and posterior are Beta distributions, then the Beta
distribution is called ``conjugate'' of the Binomial distribution.


\section{Utility}

The Bayesian posterior says nothing about how to benefit from the beliefs it
represents, in order to do this we need to specify the utility of each decision.

With this idea we define an utility function over the parameters

\[
  U(\theta, \theta_{true}) = \alpha \mathbb{I}[\theta = \theta_{true}] - \beta
  \mathbb{I}[\theta \neq \theta_{true}]
\]
where \(\alpha, \beta \in \R\). This symbolizes the gains or looses of choosing
the parameter \(\theta\), when the true value of the parameter is supposed to be
\(\theta_{true}\). Then the expected utility of a parameter \(\theta_0\) is
calculated as
\[
  U(\theta = \theta_0) = \sum_{\theta_{true}}U(\theta = \theta_0,
  \theta_{true})P(\theta = \theta_{true}  \mid  \mathcal{V})
\]

Using the last example, we may define out utility function as
\[
  U(\theta, \theta_{true}) = 10\mathbb{I}[\theta = \theta_{true}] - 20
  \mathbb{I}[\theta \neq \theta_{true}]
\]
Where we interpret that the loss of choosing the wrong parameter is twice as
important as the gains from doing it right.

The expected utility of the decision that the parameter is \(\theta = 0.2\)
in our discrete example would be
\[
  \begin{aligned}
  U(\theta = 0.2) &= U(\theta = 0.2, \theta_{true} = 0.2)P(\theta_{true} = 0.2  \mid
  \mathcal{V})\\
  &+ U(\theta = 0.2, \theta_{true} = 0.5)P(\theta_{true} = 0.5  \mid
  \mathcal{V}) \\
  & +  U(\theta = 0.2, \theta_{true} = 0.8)P(\theta_{true} = 0.8  \mid  \mathcal{V})\\
  &= 10 \times 0.4979 - 20\times 0.5059 -20 \times 0.00024 \\
  &= -5.1438\\
  U(\theta = 0.5) &= -4.9038 \\
  U(\theta = 0.8) &= -20.0736
\end{aligned}
\]
 

This illustrate how an utility function can affect the results of the inference.
The most probable value for \(\theta\) was \(0.2\), but, using this utility
function, \(0.5\) is the one with which we expect minor losses.

\section{Directed Models}
\subsection{Maximum Likelihood training}

In this section we will introduce two concepts, Maximum Likelihood and Maximum a
Posteriori, showing that \emph{training} a model's parameter to maximize the
Maximum Likelihood equals to take the empirical distribution as it.

\begin{definition}
  Maximum Likelihood is calculated as
  \[
    \theta^{ML} = \argmax_\theta p(\mathcal{V} \mid \theta)
  \]
   it refers to the value of the parameter
\(\theta\) for which the observed data better fits the model.
\end{definition}

\begin{definition}
  Maximum A Posteriori refers to
  \[
    \theta^{MAP} = \argmax_\theta p(\mathcal{V} \mid \theta)P(\theta)
  \]
\end{definition}

The decision of taking the Maximum A Posteriori can be motivated using an
utility that equals zero for all but the correct parameter
\[
  U(\theta, \theta_{true}) = \mathbb{I}[\theta = \theta_{true}]
\]

using this, the expected utility of a parameter \(\theta = \theta_0\) is

\[
  U(\theta = \theta_0) = \sum_{\theta_{true}}\mathbb{I}[\theta_{true} = \theta_0]P(\theta = \theta_{true}  \mid  \mathcal{V}) = P(\theta_0  \mid  \mathcal{V})
\]

This means that the maximum utility decision is to take the value \(\theta_0\)
with the highest posterior value.

\begin{remark}
When using a flat prior \(\theta^{ML}= \theta ^{MAP}\).
\end{remark}



Now, we are going to show the relation between the Maximum Likelihood and the
Kullback-Leibler divergence of the empirical distribution and our model.
Firstly, we define the empirical distribution of a set

Let \(\{X_1, \dots, X_m\}\) be a set of real i.i.d random variables and
\(\{x_{1}, \dots, x_{m}\}\) a set of observations of those variables, we can define the
empirical distribution as a distribution whose probability mass function
\(Q\) is
\[
  Q(x) = \frac{1}{m}\sum_{i = 1}^m \mathbb{I}[x = x_i]
\]

 We may calculate this Kullback-Leibler divergence and study their functional independence.
\[
  KL(Q \mid P) = \E[\log(Q(x))]_{Q} - \E[\log(P(x))]_Q
\]

Notice the term \(\E[\log(Q(x))]_{Q} \) is a constant as the variables are
i.i.d it follows that
\[
   \E[\log(P(x))]_Q = \frac{1}{m}\sum_{i = 1}^mlogP(x_i)
 \]
 where the right side is the log likelihood under \(Q\). As the logarithm is
 a strictly increasing function, maximizing the log likelihood equals to
 maximize the likelihood itself, and we can see here how it is equivalent to
 minimize the Kullback-Leibler divergence between the empirical distribution and
 our distribution.

 In case \(P(x)\) is unconstrained, the optimal choice is \(P(x) = Q(x)\), that
 is, the maximum likelihood distribution corresponds to the empirical distribution.

 For a Belief Network we know there is the following constraint
 \[
   P(x_{1}, \dots, x_{m}) = \prod_{i = 1}^K P(x_i  \mid  pa(x_i))
 \]
 We now want to minimize the Kullback-Leibler divergence between the empirical
 distribution \(Q(x)\) and \(P(x)\) in order to get the Maximum Likelihood.

 \[
   \begin{aligned}
   KL(Q \mid P) &= - \E\big[\sum_{i = 1}^K\log{P(x_i \mid pa(X_i))}\big]_Q +
   \E\big[\sum_{i = 1}^K\log{P(x_i \mid pa(X_i))}\big]_P
   \\ &= - \sum_{i =
     1}^K \E\big[\log{P(x_i \mid pa(X_i))}\big]_Q + \sum_{i =
     1}^K \E\big[\log{P(x_i \mid pa(X_i))}\big]_P
   \end{aligned}
 \]

 We can not use Proposition \ref{prop:expectation_over_marginal} on \(\log{P(x_i \mid pa(x_i))}\).

 \[
   \begin{aligned}
     KL(Q \mid P) &= \sum_{i = 1}^K \E\Big[ \log{Q(x_i \mid pa(x_i))}\Big]_{Q(x_i,pa(x_i))} - \E\Big[
     \log{P(x_i \mid pa(x_i))}\Big]_{Q(x_i,pa(x_i))} \\
     &= \sum_{i = 1}^K \E \Big[ KL\Big(Q(x_i \mid pa(x_i)) \mid P(x_i \mid pa(x_i))\Big) \Big]_{Q(x_i,pa(x_i))}
   \end{aligned}
 \]

 The minimal setting is then
 \[
   P(x_i \mid pa(x_i)) = Q(x_i \mid pa(x_i))
 \]
 in terms of the initial data it is to set \(P(x_i \mid pa(x_i))\) to the number of
 times the state appears in it.

 \subsection{Bayesian Belief Network Training}

A Bayesian approach where we set a distribution over the parameters is an
alternative to Maximum Likelihood training of a Bayesian Network, as we did in
the coin tossing example. We go deep into it using the following scenario, consider a disease
\(D\) and two habits \(A\) and \(B\).

\begin{figure}[!ht]
  \begin{tabular}{*{2}{>{\centering\arraybackslash}b{\dimexpr0.5\linewidth-2\tabcolsep\relax}}}
  \centering
  \begin{tikzpicture}[
    node distance=1cm and 0.5cm,
    mynode/.style={draw,circle,text width=0.5cm,align=center}
    ]

    \node[mynode] (d) {\(D_{n}\)};
    \node[mynode, above left=of d] (a) {\(A_{n}\)};
    \node[mynode, above right=of d] (b) {\(B_{n}\)};
    \node[mynode, above=of a] (ta) {\(\theta_{A}\)};
    \node[mynode, above=of b] (tb) {\(\theta_{B}\)};
    \node[mynode, below=of d] (td) {\(\theta_{D}\)};
    \plate{} {(d)(a)(b)} {\(1\dots N\)}; %
    \path (a) edge[-latex] (d)
    (b) edge[-latex] (d)
    (ta) edge[-latex] (a)
    (tb) edge[-latex] (b)
    (td) edge[-latex] (d)
    ;

  \end{tikzpicture}
    \caption{Bayesian parameter model for the relation between \(A,B,D\)}
    \label{fig:bayesian_example}
    &
      \renewcommand{\arraystretch}{1.3}
      \begin{tabular}{|l|l|l|}
    \hline
    A & B & D \\ \hline
    1 & 1 & 1 \\ \hline
    1 & 0 & 0 \\ \hline
    0 & 1 & 1 \\ \hline
    0 & 1 & 0 \\ \hline
    1 & 1 & 1 \\ \hline
    0 & 0 & 0 \\ \hline
    1 & 0 & 1 \\ \hline
  \end{tabular}\captionof{table}{Observations}
\end{tabular}
    \end{figure}



\[
P(a,b,d) = P(d|a,b)P(a)P(b)
\]

Consider the set of i.i.d random variables
\(\{A_{n}, B_{n}, D_{n}\}_{n = 1,\dots, N}\) and the
corresponding set of observations
\(\mathcal{V} = \{(a_{n}, b_{n}, d_{n}), n = 1,\dots , N\}\)

We need a notation for the parameters, as all the variables are binary we are
going to use
\[
  P(A = 1 \mid \theta_{A}) = \theta_{A}, \hspace{1cm} P(B = 1 \mid \theta_{B} = \theta_{B}), \hspace{1cm} P(D = 1 \mid A = 0, B = 1, \theta_{D}) = \theta_{1}
\]
\[\theta_{D} = (\theta_{0}, \theta_{1}, \theta_{2}, \theta_{3})\]
Using the states of \(A\) and \(B\) to create \((01)\) and its correspondent
decimal form in the sub-index of \(\theta\).

We need to specify a prior and since dealing with multi-dimensional continuous
distributions is computationally problematic it is normal to use uni-variate
distributions.

A convenient assumption is that the prior factorizes, this is usually called
\emph{global parameter independence}. We assume then
\[
  P(\theta_{A}, \theta_{B}, \theta_{D}) = P(\theta_{A})P(\theta_{B})P(\theta_{D})
\]
Assuming our data is i.i.d, we have
\[
  P(\theta_{A}, \theta_{B}, \theta_{D}, \mathcal{V}) = P(\theta_{A})P(\theta_{B})P(\theta_{D})\prod_{n}P(a_{n}\mid \theta_{A})P(b_{n} \mid \theta_{B})P(d_{n}\mid a_{n}, b_{n}, \theta_{D})
\]

Learning then corresponds to inference

\[
  \begin{aligned}
    P(\theta_{A}, \theta_{B}, \theta_{D}\mid \mathcal{V}) &= \frac{P(\mathcal{V} \mid \theta_{a}, \theta_{B}, \theta_{D})P(\theta_{A}, \theta_{B}, \theta_{D})}{P(\mathcal{V})} =\frac{P(\mathcal{V} \mid \theta_{A}, \theta_{B}, \theta_{D})P(\theta_{A}) P(\theta_{B})P( \theta_{D})}{P(\mathcal{V})}\\
    &= \frac{1}{P(\mathcal{V})}P(\theta_{A})\prod_{n}P(a_{n}\mid \theta_{A})P(\theta_{B})\prod_{n}P(b_{n}\mid \theta_{B})P(\theta_{D})\prod_{n}P(d_{n}\mid a_{n}, b_{n},\theta_{D})\\
    &= P(\theta_{A} \mid \V_{A} )P(\theta_{B}\mid \V_{B})P(\theta_{D} \mid \V)
  \end{aligned}
\]

Where \(V_{i}\) is the subset of the data restricted to the variable \(i\). If
we further assume that \(P(\theta_{D})\) factorizes as
\(P(\theta_{D}) = P(\theta_{0})P(\theta_{1})P(\theta_{2})P(\theta_{3})\),
this is called \emph{local parameter independence}, then it follows that
\[
  P(\theta_{D}\mid \V) = P(\theta_{0} \mid \V )P(\theta_{1} \mid \V )P(\theta_{2} \mid \V )P(\theta_{3} \mid \V )
\]

The simplest cases to continue are \(P(a\mid \theta_{A})\) and
\(P(b \mid \theta_{b})\) since they require only a uni-variate prior distribution
\(P(\theta_{A})\) or \(P(\theta_{b})\). We use \(P(\theta_{A})\) as the other
case is analogous.

The posterior is
\[
  P(\theta_{A} \mid \V_{A}) = \frac{1}{P(\V_{A})}P(\theta_{A})\theta_{A}^{\#(a=1)}(1-\theta_{A})^{\#(a=0)}
\]

The most convenient choice for the prior is a Beta distribution as conjugacy
will hold.

\[
  \theta_{A} \sim \text{Beta}(\alpha_{A}, \beta_{A}) \implies P(\theta_{A})  = \frac{1}{B(\alpha_{A}, \beta_{A})}\theta_{A}^{\alpha_{A}-1}(1-\theta_{A})^{\beta_{A} - 1}
\]
So it follows that
\[
  (\theta_{A} \mid \V_{A}) \sim \text{Beta}(\theta_{A} \mid \alpha_{A} + \#(A=1), \beta_{A} + \#(A = 0))
\]

The marginal is then
\[
  \begin{aligned}
    P(A = 1 \mid \V_{A})
    &= \frac{P(A = 1, \V_{A})}{P(\V_{A})} = \int_{\theta_{A}}  \frac{P(A = 1, \V_{A}, \theta_{A})}{P(\V_{A})} =  \int_{\theta_{A}}  \frac{P(A = 1 \mid \V_{A}, \theta_{A}) P(\V_{A}, \theta_{A})}{P(\V_{A})} \\
    &=  \int_{\theta_{A}}  \frac{P(A = 1 \mid \V_{A}, \theta_{A}) P(\theta_{A} \mid \V_{A})P(\V_{A})}{P(\V_{A})} = \int_{\theta_{A}}P(\theta_{A}\mid \V_{A})\theta_{A} = \E[\theta_{A} \mid \V_{A}] \\
    &= \frac{\alpha_{A} + \#(A= 1)}{\alpha_{A} + \#(A=1) + \beta_{A} + \#(A=0)}
  \end{aligned}
\]

For \(P(d \mid a ,b)\) the situation is more complex, the most convenient way is
to specify a Beta prior for each one of the four components of \(\theta_{D}\).
Lets focus on \(P(D = 1 \mid A = 1, B = 0)\), notice the parameters \(\alpha\)
and \(\beta\) we used before now depend on \(a\) and \(b\), for this reason we
are using \(\alpha_{D}(a,b)\) and \(\beta_{D}(a,b)\) as prior parameters, these
are called \emph{hyperparameters}.
\[
  P(\theta_{2}) = B(\theta_{2} \mid \alpha_{D}(1,0) + \#(D = 1, A = 1, B = 0), \beta_{D}(1,0) + \#(D = 0, A = 1, B = 0))
\]

As before we got that

\[
  P(D = 1 \mid A = 1, B = 0, \V) = \frac{\alpha_{D}(1,0) + \#(D = 1, A = 1, B = 0)}{\alpha_{D}(1,0) + \beta_{D}(1,0) + \#(A=1, B = 0)}
\]

In case we had no preference, we could set all hyperparameters to the same
value, and, a complete ignorance prior would correspond to set them to 1.

Let now consider two limit possibilities, the one where we have no data at all,
and the one where we have infinite data.

In case we have no data, the marginal probability corresponds to the prior which
in the last case is
\[
   P(D = 1 \mid A = 1, B = 0, \V) = \frac{\alpha_{D}(1,0)}{\alpha_{D}(1,0) + \beta_{D}(1,0)}
 \]
 Note that equal hyperparameters would give a result of \(0.5\).

 When infinite data is available, the marginal is generally dominated by it,
 this corresponds to the Maximum Likelihood solution.
 \[
   P(D = 1 \mid A = 1, B = 0, \V) = \frac{\#(D = 1, A = 1 , B = 0)}{\#(A = 1, B = 0)}
 \]
 This happens unless the prior has a pathologically strong effect.

 Consider the data given in the table in figure \ref{fig:bayesian_example}, and
 equal parameters and hyperparameters \(1\). Then we can compute the differences
 between this and using the Maximum Likelihood technique.
 \[
   P(A = 1 \mid \V) = \frac{1 + \#(A = 1)}{2 + N} = \frac{5}{9} \approx 0.556
 \]
 By comparison, the Maximum Likelihood result is \(4/7 = 0.571\), the Bayesian
 result is more prudent than this one, which fits in with our belief that any
 setting is equally probable i.e \(0.5\).


 The natural generalization to more than two states is using a Dirichlet
 distribution as prior, assuming i.i.d data and local and global prior
 independence. We are considering two different scenarios, firstly one where the
 variable has no parents, as the case for \(A\) and \(B\) in the previous
 example. Secondly, we will consider a variable with a non void set of parents,
 as in the case with the disease \(D\).

 Consider a variable \(X\) with
 \(Dom(X) = \{1, \dots, I\}, \ \theta = (\theta_{1},\dots, \theta_{I})\), then
 \[
   P(x \mid \theta) = \prod_{i = 1}^{I}\theta_{i}^{\mathbb{I}[x = i]} \text{
   with  } \sum_{i=1}^{I}\theta_{i} = 1
\]
So that the posterior (considering \(N\) observations of the variable
\((x_{1}, \dots, x_{N}) = \V\)) is
\[
  P(\theta \mid x_{1},\dots,x_{N}) = \frac{1}{P(\V)} P(\theta) \prod_{n = 1}^{N}\prod_{i =1 }^{I}\theta_{i}^{\mathbb{I}[x_{n} = i]} =  \frac{1}{P(\V)} P(\theta) \prod_{i = 1}^{I} \theta_{i}^{\sum_{n} \mathbb{I}[x_{n}=i]}
\]
Then assuming a Dirichlet prior with hyperparameters \(\bm{u} = (u_{1}, \dots, u_{I})\)
\[
  P(\theta) = \frac{1}{B(\bm{u})}\prod_{i =1}^{I}\theta_{i}^{u_{i}-1} \implies P(\theta \mid \V) = \frac{1}{B(\bm{u})P(\V)}\prod_{i=1}^{I}\theta_{i}^{u_{i}-1 + \sum_{n}\mathbb{I}[x_{n} = i]}
\]

Which means that, defining \(\bm{c} = ( \sum_{n=1}^{N}\mathbb{I}[x_{n} = i])_{i = 1,\dots,I}\)
\[
  P(\theta \mid \V) \sim \text{Dirichlet}(\bm{u} + \bm{c})
\]

The marginal is then given by
\[
  \begin{aligned}
    P(X=i \mid \V) &= \int_{\theta}P(X=i \mid \theta)P(\theta \mid \V) =  \int_{\theta}\theta_{i}P(\theta \mid \V)\\
    &=  \int_{\theta_{i}}\theta_{i}P(\theta_{i} \mid \V) = \E[\theta_{i} \mid \V]
\end{aligned}
\]
Where we used that
\[\int_{\theta_{j \neq i}}\theta_{i} P(\theta \mid \V) = \theta_{i}\prod_{k\neq j}P(\theta_{k} \mid V) \int_{\theta_{j}}P(\theta_{j}\mid \V) = \theta_{i} \prod_{k \neq j}P(\theta_{k} \mid \V)\]


As we already know from Proposition \ref{prop:dirichlet_marginal}, the univariate marginal of a Dirichlet distribution is a
Beta Distribution, then
\[
  (\theta_{i} \mid \V) \sim \text{Beta}(u_{i} + c_{i}, \sum_{j\neq i} u_{j} + c_{j})
\]
So the marginal is
\[
  P(X = i \mid \V) = \frac{u_{i} + c_{i}}{\sum_{j}u_{j} + c_{j}}
\]


Consider now that \(X\) has a set of parent variables \(pa(X)\), in this case,
we want to compute the marginal given a state of its parents and the data
\[
  P(X = i \mid pa(X) = \bm{j}, \V)
\]
Let set the following notation for the parameters
\[
  P(X = i \mid pa(X) = \bm{j}, \theta) = \theta_{i,\bm{j}} \hspace{2cm} \bm{\theta_{j}} = (\theta_{1,\bm{j}},\dots, \theta_{I,\bm{j}})
\]
Local independence means that
\[
  P(\bm{\theta}) = \prod_{j}P(\bm{\theta_{j}})
\]

As we did before, we consider a Dirichlet prior
\[
  \bm{\theta_{j}} \sim Dirichlet(\bm{u_{j}})
\]
the posterior is then
\[
  \begin{aligned}
    P(\bm{\theta} \mid \V) &= \frac{P(\bm{\theta})P(\V \mid \bm{\theta}) }{P(\V)} = \frac{1}{P(\V)}\Big(\prod_{\bm{j}}P(\bm{\theta_{j}}) \Big)P(\V \mid \bm{\theta}) \\
    &= \frac{1}{P(\V)}\Big(\prod_{\bm{j}}\frac{1}{B(\bm{u_{j}})}\prod_{i}\theta_{i,\bm{j}}^{u_{i,\bm{j}}-1}\Big) P(\V \mid \bm{\theta})\\
    &= \frac{1}{P(\V)}\Big(\prod_{\bm{j}}\frac{1}{B(\bm{u_{j}})}\prod_{i}\theta_{i,\bm{j}}^{u_{i,\bm{j}}-1}\Big) \Big(\prod_{n}\prod_{\bm{j}}\prod_{i} \theta_{i,\bm{j}}^{\mathbb{I}[x_{n} = i, pa(x_{n}) = \bm{j}]}\Big)\\
    &= \frac{1}{P(\V)}\prod_{\bm{j}}\frac{1}{B(\bm{u_{j}})}\prod_{i}\theta_{i,\bm{j}}^{u_{i,\bm{j}}-1 + \#(X = i,pa(X)=\bm{j})}
  \end{aligned}
\]
Naming \(\bm{v_{j}} = \bm{u_{j}} + \#(X = i, pa(X) = \bm{j})\), the posterior
is
\[
  (\bm{\theta} \mid \V) \sim \prod_{j}\text{Dirichlet}(\bm{v_{j}})
\]

Noting \(v_{i,j}\) the components of \(\bm{v_{j}}\), the marginal is then
\[
  P(X=i, pa(X) = \bm{j}, \V) = \frac{v_{i,j}}{\sum_{i}v_{i,j}}
\]

\subsection{Structure Learning}

\section{Undirected Models}


\clearpage
Cites so the references appear (testing) \cite{koller_friedman,barber,wainwright}
\bibliographystyle{plain}
\bibliography{refs}
\end{document}
