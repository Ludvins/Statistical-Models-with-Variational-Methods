
In this section we will introduce two concepts, Maximum Likelihood and Maximum a
Posteriori, showing that \emph{training} a model's parameter to maximize the
Maximum Likelihood equals to take the empirical distribution as it.

Let \(\{X_{1},\dots, X_{N}\}\) be a set of real i.i.d random variables and \(\V = \{x_{1}, \dots, x_{N}\}\) be the set of observations and \(\theta\) the considered parameters of the model.

\begin{definition}
  Maximum Likelihood is calculated as
  \[
    \theta^{ML} = \argmax_\theta P(\mathcal{V} \mid \theta)
  \]
   it refers to the value of the parameter
\(\theta\) for which the observed data better fits the model.
\end{definition}

\begin{definition}
  Maximum A Posteriori refers to
  \[
    \theta^{MAP} = \argmax_\theta P(\mathcal{V} \mid \theta)P(\theta)
  \]
\end{definition}

The decision of taking the Maximum A Posteriori can be motivated using an
utility that equals zero for all but the correct parameter
\[
  U(\theta, \theta_{true}) = \mathbb{I}[\theta = \theta_{true}]
\]

using this, the expected utility of a parameter \(\theta = \theta_0\) is

\[
  U(\theta = \theta_0) = \sum_{\theta_{true}}\mathbb{I}[\theta_{true} = \theta_0]P(\theta = \theta_{true}  \mid  \mathcal{V}) = P(\theta_0  \mid  \mathcal{V})
\]

This means that the maximum utility decision is to take the value \(\theta_0\)
with the highest posterior value.

\begin{remark}
  When using a flat prior, i.e, \(P(\theta)\) is constant, \(\theta^{ML}= \theta ^{MAP}\).
\end{remark}

\section{ML and KL divergence}

Now, we are going to show the relation between the Maximum Likelihood and the
Kullback-Leibler divergence of the empirical distribution and our model.
Firstly, we define the empirical distribution as a distribution whose probability mass function
\(Q\) is
\[
  Q(x) = \frac{1}{N}\sum_{i = 1}^N \mathbb{I}[x = x_i]
\]

We may calculate the Kullback-Leibler divergence between the empirical and our considered model \(P(x \mid \theta)\) and study their functional independence.
\[
  \KL{Q}{P} = \E{Q}{\log(Q(x))} - \E{Q}{\log(P(x \mid \theta))}
\]

Notice the term \(\E{Q}{\log(Q(x))}\) is a constant and the log likelihood under \(Q\) takes the form
\[
   \E{Q}{\log(P(x \mid \theta))} = \frac{1}{N}\int_{x}\sum_{i=1}^{N}\mathbb{I}[x = x_{i}]\log{P(x \mid \theta)} = \frac{1}{N}\sum_{i = 1}^N \log{P(x_i \mid \theta)}
 \]
 As the logarithm is
 a strictly increasing function, maximizing the log likelihood equals to
 maximize the likelihood itself, and we can see here how it is equivalent to
 minimize the Kullback-Leibler divergence between the empirical distribution and
 our distribution.
 \[
   \begin{aligned}
     &\argmin_{\theta} \KL{Q}{P} = \argmin_{\theta} - \E{Q}{\log(P(x \mid \theta))} =  \argmax_{\theta} \E{Q}{\log(P(x \mid \theta))}\\
     &\argmax_{\theta} \frac{1}{N}\sum_{i=1}^{N}\log{P(x_{i} \mid \theta)} =  \argmax_{\theta} \frac{1}{N}\sum_{i=1}^{N}P(x_{i} \mid \theta) = \argmax_{\theta} \sum_{i=1}^{N}P(x_{i}\mid \theta)= \theta^{ML}
   \end{aligned}
 \]

 In case \(P(x \mid \theta)\) is unconstrained, the optimal choice is \(P(x \mid \theta) = Q(x)\), that
 is, the \textbf{maximum likelihood distribution corresponds to the empirical distribution}.

 For a Belief Network we know there is the following constraint
 \[
   P(x_{1}, \dots, x_{m} \mid \theta) = \prod_{i = 1}^K P(x_i  \mid  pa(x_i), \theta)
 \]
 We now want to minimize the Kullback-Leibler divergence between the empirical
 distribution \(Q(x)\) and \(P(x \mid \theta)\) in order to get the Maximum Likelihood.
 \[
   \begin{aligned}
   \KL{Q}{P} &= - \E{Q}{\sum_{i = 1}^K\log{P(x_i \mid pa(x_i), \theta)}} +
   \E{Q}{\sum_{i = 1}^K\log{Q(x_i \mid pa(x_i))}}
   \\ &= - \sum_{i =
     1}^K \E{Q}{\log{P(x_i \mid pa(x_i), \theta)}} + \sum_{i =
     1}^K \E{Q}{\log{Q(x_i \mid pa(x_i))}}
   \end{aligned}
 \]

 Notice the expectation is over the full distribution \(Q(x_{1},\dots, x_{k})\), we can use proposition \ref{prop:expectation_over_marginal} on \(\log{P(x_i \mid pa(x_i), \theta)}\) and \(Q(x_{i}, pa(x_{i}))\).

 \[
   \begin{aligned}
     \KL{Q}{P} &= \sum_{i = 1}^K \E{Q(x_i,pa(x_i))}{\log{Q(x_i \mid pa(x_i))}} - \E{Q(x_i,pa(x_i))}{\log{P(x_i \mid pa(x_i), \theta)}}\\
     &= \sum_{i = 1}^K \E{Q(x_i,pa(x_i))}{\KL{Q(x_i \mid pa(x_i))}{P(x_i \mid pa(x_i), \theta)}}
   \end{aligned}
 \]

 The minimal setting is then
 \[
   P(x_i \mid pa(x_i), \theta) = Q(x_i \mid pa(x_i))
 \]
 in terms of the initial data it is to set \(P(x_i \mid pa(x_i))\) to the number of
 times the state appears in it.
