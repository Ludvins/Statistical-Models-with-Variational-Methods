


A Bayesian approach where we set a distribution over the parameters is an
alternative to Maximum Likelihood training of a Bayesian Network, as we did in
the coin tossing example.

We go deep into it using the following scenario, consider a disease
\(D\) and two habits \(A\) and \(B\). Let \(N = 7\) be the number of observations of the variables as shown in the table below, that means we are considering the following i.i.d variables \(\{A_{1},\dots, A_{N}\}\), \(\{B_{1},\dots,b_{N}\}\) and \(\{D_{1},\dots, D_{N}\}\) governed by the parameters \(\theta_{A}, \theta_{B}\) and \(\theta_{D}\) as shown in the following Bayesian Network.

\begin{figure}[!ht]
  \begin{tabular}{*{2}{>{\centering\arraybackslash}b{\dimexpr0.5\linewidth-2\tabcolsep\relax}}}
  \centering
  \begin{tikzpicture}[
    node distance=1cm and 0.5cm,
    mynode/.style={draw,circle,text width=0.5cm,align=center}
    ]

    \node[mynode] (d) {\(D_{n}\)};
    \node[mynode, above left=of d] (a) {\(A_{n}\)};
    \node[mynode, above right=of d] (b) {\(B_{n}\)};
    \node[mynode, above=of a] (ta) {\(\theta_{A}\)};
    \node[mynode, above=of b] (tb) {\(\theta_{B}\)};
    \node[mynode, below=of d] (td) {\(\theta_{D}\)};
    \plate{} {(d)(a)(b)} {\(1\dots N\)}; %
    \path (a) edge[-latex] (d)
    (b) edge[-latex] (d)
    (ta) edge[-latex] (a)
    (tb) edge[-latex] (b)
    (td) edge[-latex] (d)
    ;

  \end{tikzpicture}
    \caption{Bayesian parameter model for the relation between \(A,B,D\)}
    \label{fig:bayesian_example}
    &
      \renewcommand{\arraystretch}{1.3}
      \begin{tabular}{|l|l|l|}
    \hline
    A & B & D \\ \hline
    1 & 1 & 1 \\ \hline
    1 & 0 & 0 \\ \hline
    0 & 1 & 1 \\ \hline
    0 & 1 & 0 \\ \hline
    1 & 1 & 1 \\ \hline
    0 & 0 & 0 \\ \hline
    1 & 0 & 1 \\ \hline
  \end{tabular}\captionof{table}{Observations}
\end{tabular}
\end{figure}

The graph gives the following joint probability distribution over \(A, B\) and \(D\)
\[
P(a,b,d) = P(d\mid a,b)P(a)P(b)
\]

Let \(\mathcal{V} = \{(a_{n}, b_{n}, d_{n}), n = 1,\dots , N\}\) be the set of observations.

We need a notation for the parameters, as all the variables are binary we are
going to use
\[
  P(A = 1 \mid \theta_{A}) = \theta_{A}, \hspace{1cm} P(B = 1 \mid \theta_{B} = \theta_{B}), \hspace{1cm} P(D = 1 \mid A = 0, B = 1, \theta_{D}) = \theta_{1}
\]
\[
  \theta_{D} = ( \theta_{0}, \theta_{1}, \theta_{2},\theta_{3})
\]
Where we are using a binary to decimal transformation between the states of \(A\) and \(B\) and the sub-index of \(\theta\).

We need to specify a prior and since dealing with multidimensional continuous
distributions is computationally problematic it is normal to use uni-variate
distributions.

\section{Global and local parameter independence}

A convenient assumption is that the prior factorizes, this is usually called
\emph{global parameter independence}.
\[
  P(\theta_{A}, \theta_{B}, \theta_{D}) = P(\theta_{A})P(\theta_{B})P(\theta_{D})
\]
Using this
\[
  P(\theta_{A}, \theta_{B}, \theta_{D}, \mathcal{V}) = P(\theta_{A})P(\theta_{B})P(\theta_{D})\prod_{n}P(a_{n}\mid \theta_{A})P(b_{n} \mid \theta_{B})P(d_{n}\mid a_{n}, b_{n}, \theta_{D})
\]

Learning then corresponds to inference
\[
  \begin{aligned}
    P(\theta_{A}, \theta_{B}, \theta_{D}\mid \mathcal{V}) &= \frac{P(\mathcal{V} \mid \theta_{a}, \theta_{B}, \theta_{D})P(\theta_{A}, \theta_{B}, \theta_{D})}{P(\mathcal{V})} =\frac{P(\mathcal{V} \mid \theta_{A}, \theta_{B}, \theta_{D})P(\theta_{A}) P(\theta_{B})P( \theta_{D})}{P(\mathcal{V})}\\
    &= \frac{1}{P(\mathcal{V})}P(\theta_{A})P(\theta_{B})P(\theta_{D})\prod_{n}P(a_{n}\mid \theta_{A})P(b_{n}\mid \theta_{B})P(d_{n}\mid a_{n}, b_{n},\theta_{D})\\
    &= P(\theta_{A} \mid \V_{A} )P(\theta_{B}\mid \V_{B})P(\theta_{D} \mid \V)
  \end{aligned}
\]

If we further assume that \(P(\theta_{D})\) factorizes as
\(P(\theta_{D}) = P(\theta_{0})P(\theta_{1})P(\theta_{2})P(\theta_{3})\),
this is called \emph{local parameter independence}, then it follows that
\[
  P(\theta_{D}\mid \V) = P(\theta_{0} \mid \V )P(\theta_{1} \mid \V )P(\theta_{2} \mid \V )P(\theta_{3} \mid \V )
\]

\section{Learning binary variables}

The simplest cases to continue are \(P(a\mid \theta_{A})\) and
\(P(b \mid \theta_{b})\) since they require only a uni-variate prior distribution
\(P(\theta_{A})\) or \(P(\theta_{B})\). We use \(P(\theta_{A})\) as the other
case is analogous.

The posterior is
\[
  P(\theta_{A} \mid \V_{A}) = \frac{1}{P(\V_{A})}P(\theta_{A})\theta_{A}^{\#(a=1)}(1-\theta_{A})^{\#(a=0)}
\]

The most convenient choice for the prior is a Beta distribution as conjugacy
will hold.

\[
  \theta_{A} \sim \text{Beta}(\alpha_{A}, \beta_{A}) \implies P(\theta_{A})  = \frac{1}{B(\alpha_{A}, \beta_{A})}\theta_{A}^{\alpha_{A}-1}(1-\theta_{A})^{\beta_{A} - 1}
\]
So it follows that
\[
  (\theta_{A} \mid \V_{A}) \sim \text{Beta}(\alpha_{A} + \#(A=1), \beta_{A} + \#(A = 0))
\]

The marginal is then
\[
  \begin{aligned}
    P(A = 1 \mid \V_{A})
    &= \frac{P(A = 1, \V_{A})}{P(\V_{A})} = \int_{\theta_{A}}  \frac{P(A = 1, \V_{A}, \theta_{A})}{P(\V_{A})} =  \int_{\theta_{A}}  \frac{P(A = 1 \mid \V_{A}, \theta_{A}) P(\V_{A}, \theta_{A})}{P(\V_{A})} \\
    &=  \int_{\theta_{A}}  \frac{P(A = 1 \mid \V_{A}, \theta_{A}) P(\theta_{A} \mid \V_{A})P(\V_{A})}{P(\V_{A})} = \int_{\theta_{A}}P(\theta_{A}\mid \V_{A})\theta_{A} = \E{}{\theta_{A} \mid \V_{A}} \\
    &= \frac{\alpha_{A} + \#(A= 1)}{\alpha_{A} + \#(A=1) + \beta_{A} + \#(A=0)}
  \end{aligned}
\]
Where the last equality is the expected value of a Beta distribution.

For \(P(d \mid a ,b)\) the situation is more complex, the simplest approach
to specify a Beta prior for each of the components of \(\theta_{D}\).
Focus on \(\theta_{2}\), notice the parameters \(\alpha\)
and \(\beta\) we used before are now dependent from \(A\) and \(B\), these
are called \emph{hyperparameters}.
\[
  \theta_{2} \sim Beta(\alpha_{D}(1,0) + \#(D = 1, A = 1, B = 0), \beta_{D}(1,0) + \#(D = 0, A = 1, B = 0))
\]

Repeating the procedure we used with \(A\) we get that
\[
  P(D = 1 \mid A = 1, B = 0, \V) = \frac{\alpha_{D}(1,0) + \#(D = 1, A = 1, B = 0)}{\alpha_{D}(1,0) + \beta_{D}(1,0) + \#(A=1, B = 0)}
\]

In case we had no preference, we could set all hyperparameters to the same
value, where a complete ignorance prior would correspond to set them to 1.

Once we get to this situation there are two limit possibilities depending on the amount of data we got.
\begin{itemize}
  \item \textbf{No data}. The marginal probability corresponds to the prior which
in the last case is
    \[
    P(D = 1 \mid A = 1, B = 0, \V) = \frac{\alpha_{D}(1,0)}{\alpha_{D}(1,0) + \beta_{D}(1,0)}
    \]
    Note that equal hyperparameters would give a result of \(0.5\).\newline
   
  \item \textbf{Infinite data}. When infinite data is available, the marginal is generally dominated by it,
    this corresponds to the Maximum Likelihood solution.
    \[
    P(D = 1 \mid A = 1, B = 0, \V) = \frac{\#(D = 1, A = 1 , B = 0)}{\#(A = 1, B = 0)}
    \]
    This happens unless the prior has a pathologically strong effect.
\end{itemize}

 Consider the data given in the table in figure \ref{fig:bayesian_example}, and
 equal parameters and hyperparameters \(1\). Then we can compute the differences
 between this and using the Maximum Likelihood technique.
 \[
   P(A = 1 \mid \V) = \frac{1 + \#(A = 1)}{2 + N} = \frac{5}{9} \approx 0.556
 \]
 By comparison, the Maximum Likelihood result is \(4/7 = 0.571\), the Bayesian
 result is more prudent than this one, which fits in with our belief that any
 setting is equally probable i.e \(0.5\).

 \section{Learning discrete variables}

 The natural generalization to more than two states is using a Dirichlet
 distribution as prior, assuming i.i.d data, local and global prior
 independence. We are considering two different scenarios, firstly one where the
 variable has no parents, as the case for \(A\) and \(B\) in the previous
 example. Secondly, we will consider a variable with a non void set of parents.

 \subsection{No parents}

 Consider a variable \(X \sim Categorical(\theta)\) with
 \(Dom(X) = \{1, \dots, I\} \) and \(\theta = (\theta_{1},\dots, \theta_{I})\) so that
 \[
   P(x) = \prod_{i = 1}^{I}\theta_{i}^{\mathbb{I}[x = i]} \text{
   with  } \sum_{i=1}^{I}\theta_{i} = 1
\]
So that the posterior (considering \(N\) observations of the variable
\(\V = (x_{1}, \dots, x_{N})\)) is
\[
  P(\theta \mid \V) = \frac{1}{P(\V)} P(\theta) \prod_{n = 1}^{N}\prod_{i =1 }^{I}\theta_{i}^{\mathbb{I}[x_{n} = i]} =  \frac{1}{P(\V)} P(\theta) \prod_{i = 1}^{I} \theta_{i}^{\sum_{n} \mathbb{I}[x_{n}=i]}
\]
Then assuming a prior \(\theta \sim Dirichlet(\bm{u})\) with \( \bm{u} = (u_{1}, \dots, u_{I})\)
\[
  P(\theta) = \frac{1}{B(\bm{u})}\prod_{i =1}^{I}\theta_{i}^{u_{i}-1} \implies P(\theta \mid \V) = \frac{1}{B(\bm{u})P(\V)}\prod_{i=1}^{I}\theta_{i}^{u_{i}-1 + \sum_{n}\mathbb{I}[x_{n} = i]}
\]

Which means that, defining \(\bm{c} = ( \sum_{n=1}^{N}\mathbb{I}[x_{n} = i])_{i = 1,\dots,I}\)
\[
  (\theta \mid \V) \sim Dirichlet(\bm{u} + \bm{c})
\]
It may not be easy to decompose \(B(\bm{u} + \bm{c})\) as \(B(\bm{u})P(\V)\), but using that it ensures normalization of the Beta distribution we may use that
\[
  \int_{\theta} P(\theta \mid \V) = 1 \implies \int_{\theta}\prod_{i=1}^{I}\theta_{i}^{u_{i}+c_{i}-1} = B(\bm{u} + \bm{c}) = B(\bm{u})P(\V)
\]

\begin{remark}
Summarizing the above information, we just proved that the Dirichlet distribution is the conjugate prior of the Categorical Distribution.
\end{remark}

The marginal is then given by
\[
  \begin{aligned}
    P(X=i \mid \V) &= \int_{\theta}P(X=i \mid \theta)P(\theta \mid \V) =  \int_{\theta}\theta_{i}P(\theta \mid \V)\\
    &=  \int_{\theta_{i}}\theta_{i}P(\theta_{i} \mid \V) = \E{}{\theta_{i} \mid \V}
\end{aligned}
\]
Where we used that
\[\int_{\theta_{j \neq i}}\theta_{i} P(\theta \mid \V) = \theta_{i}\prod_{k\neq j}P(\theta_{k} \mid V) \int_{\theta_{j}}P(\theta_{j}\mid \V) = \theta_{i} \prod_{k \neq j}P(\theta_{k} \mid \V)\]

As we already know from Proposition \ref{prop:dirichlet_marginal}, the uni-variate marginal of a Dirichlet distribution is a
Beta Distribution of parameters
\[
  (\theta_{i} \mid \V) \sim \text{Beta}(u_{i} + c_{i}, \sum_{j\neq i} u_{j} + c_{j})
\]
Using the expectation of a Beta distribution we get that
\[
  P(X = i \mid \V) = \frac{u_{i} + c_{i}}{\sum_{j}u_{j} + c_{j}}
\]

\subsection{Parents}
Consider now that \(X\) has a set of parent variables \(pa(X)\), in this case,
we want to compute the marginal given a state of its parents and the data
\[
  P(X = i \mid pa(X) = \bm{j}, \V)
\]
We are using the following notation for the parameters
\[
  P(X = i \mid pa(X) = \bm{j}, \theta) = \theta_{i,\bm{j}} \hspace{2cm} \bm{\theta_{j}} = (\theta_{1,\bm{j}},\dots, \theta_{I,\bm{j}})
\]
Local independence means that
\[
  P(\bm{\theta}) = \prod_{j}P(\bm{\theta_{j}})
\]

As we did before, we consider a Dirichlet prior
\[
  \bm{\theta_{j}} \sim Dirichlet(\bm{u_{j}})
\]

The posterior is then
\[
  \begin{aligned}
    P(\bm{\theta} \mid \V) &= \frac{P(\bm{\theta})P(\V \mid \bm{\theta}) }{P(\V)} = \frac{1}{P(\V)}\Big(\prod_{\bm{j}}P(\bm{\theta_{j}}) \Big)P(\V \mid \bm{\theta}) \\
    &= \frac{1}{P(\V)}\Big(\prod_{\bm{j}}\frac{1}{B(\bm{u_{j}})}\prod_{i}\theta_{i,\bm{j}}^{u_{i,\bm{j}}-1}\Big) P(\V \mid \bm{\theta})\\
    &= \frac{1}{P(\V)}\Big(\prod_{\bm{j}}\frac{1}{B(\bm{u_{j}})}\prod_{i}\theta_{i,\bm{j}}^{u_{i,\bm{j}}-1}\Big) \Big(\prod_{n}\prod_{\bm{j}}\prod_{i} \theta_{i,\bm{j}}^{\mathbb{I}[x_{n} = i, pa(x_{n}) = \bm{j}]}\Big)\\
    &= \frac{1}{P(\V)}\prod_{\bm{j}}\frac{1}{B(\bm{u_{j}})}\prod_{i}\theta_{i,\bm{j}}^{u_{i,\bm{j}}-1 + \#(X = i,pa(X)=\bm{j})}
  \end{aligned}
\]
Naming \(\bm{v_{j}} = \bm{u_{j}} + \#(X = i, pa(X) = \bm{j})\) and using the same argument as we did in the \emph{no parents} case with the normalization constants, the posterior then follows
\[
  (\bm{\theta} \mid \V) \sim \prod_{j}\text{Dirichlet}(\bm{v_{j}})
\]
Noting \(v_{i,j}\) the components of \(\bm{v_{j}}\), the marginal is then
\[
  P(X=i, pa(X) = \bm{j}, \V) = \frac{v_{i,j}}{\sum_{i}v_{i,j}}
\]
Notice all the above has been done using a fixed variable \(X\), so that all the parameters depend on that variable.

We can define the data likelihood under a model, usually called the \emph{model likelihood} using the same calculations as we did with \(P(\bm{\theta} \mid \V)\) but applied to all the variables of the model
\[
  \begin{aligned}
    P(\V \mid \mathcal{M}) &= \int_{\bm{\theta}}P(\bm{\theta})P(\V \mid \bm{\theta}, \mathcal{M}) = \prod_{x} \prod_{\bm{j}} \frac{1}{B(\bm{u_{j}})} \int_{\bm{\theta}} \prod_{i}\theta_{i,\bm{j}}^{u_{i,\bm{j}}-1 + \#(x = i,pa(x)=\bm{j})}\\
    &= \prod_{x}\prod_{\bm{j}} \frac{B(\bm{v_{j}})}{B(\bm{u_{j}})}
  \end{aligned}
\]
