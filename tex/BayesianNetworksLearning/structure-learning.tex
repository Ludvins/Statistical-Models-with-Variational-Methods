
The Bayesian network structure is not always given and has to be learned from the data, to achieve this, there are some issues that need to be considered. These are:
\begin{itemize}\setlength{\itemsep}{0.15cm}
  \item The number of Bayesian networks is exponential over the number of variables so a brute force algorithm would not be viable.
  \item Testing dependencies requires a large amount of data, therefore, a threshold needs to be established to measure when a dependence is significant.
  \item The presence of hidden variables might not be learned from the data.
\end{itemize}


\begin{algorithm}[t]
  \SetAlgoLined\KwData{Complete undirected graph \(G\), with vertices \(V\)}
  \KwResult{\(G\) with removed links}
  \(i = 0\)\;
  \While{all nodes have \(\leq i\) neighbors}{
    \For{\(X \in V\)}{
      \For{\(Y \in ne(x)\)}{
        \If{\(\exists S \subset ne(X)\backslash Y\) such that \(\#S = i\) and
          \(X \bigCI Y \mid S\)}{
          Remove \(X-Y\) from \(G\)\;
          \(S_{XY} = S\)\;
        }
      }
    }
    \(i = i+1\)\;
  }
  \caption{PC Algorithm for skeleton learning}\label{alg:pc}
\end{algorithm}

\section{PC algorithm}

The \emph{PC algorithm} (\cite{spirtes2000causation} Chapter 5.4.2) learns the skeleton structure given a complete graph \(G=(V,E)\) (constructed from the considered set of variables) and orients its edges using variable independence in the empirical distribution.

The skeleton learning part (Algorithm~\ref{alg:pc}) iterates over subsets of neighborhoods, from smaller to bigger ones. It chooses a linked pair of variables \((X,Y) \in E\) and
a subset \(S_{XY} \subset ne(X)\), verifying \(Y \notin S_{XY}\). If \(X \bigCI Y \mid S_{XY}\), then the link is removed and \(S_{XY}\) is stored.

The main idea behind this procedure is using the following theorem and the d-separation Theorem~\ref{th:d-separation}.
\begin{theorem}[\cite{Le_2019} Theorem 1]
  Let \(X\) and \(Y\) be two non-adjacent nodes in a Bayesian network, then, there exists \(\bZ\) a subset of \(ne(X)\backslash Y\) or \(ne(Y)\backslash X\) that d-separates them.
\end{theorem}

Where if \(X \bigCI Y \mid S_{XY}\), the d-separation theorem implies that \(S_{XY}\) d-separates \(X\) and \(Y\), which is a necessary condition to erase their link.

This procedure results in the skeleton of the Bayesian Network, no more edges will be removed or added. The directed graph may be constructed following two rules:
\begin{enumerate}
  \item For any undirected link \(X - Y - Z\), if \(Y \notin S_{XZ}\) then set
    \(X \to Y \leftarrow Z\) (we are creating a collider for that path).
  \item The rest of links may oriented arbitrarily not
creating cycles or colliders.
\end{enumerate}
The reasoning behind is the d-separation Theorem~\ref{th:d-separation},  if \(Y \notin S_{XZ}\) then \(X \bigCD Z \mid Y\) (if \(Y\) did, it would be the d-separation set \(S_{XZ} = \{Y\}\) as they are checked from smaller to bigger ones) so \(Y\) must d-connect them, to achieve this, it must be set as a collider. This is because, using the d-connection definition, the only known undirected path is \(U = X - Y - Z\) and no non-collider on \(U\) must belong to \(\{Y\}\).


On the other hand, if \(Y \in S_{XZ}\) and \(X \bigCI Z \mid S_{XZ}\) then \(S_{XZ}\) should d-separate them, that is, using any configuration that doesn't create a collider in \(S_{XZ}\).

\section{Independence learning}

The PC algorithms assumes there exists a procedure of testing conditional independence of variables, that is, given three variables \(X, Y\) and \( Z \),  measure \(X \bigCI Y \mid Z\). One approach is to measure the empirical \emph{conditional mutual information} of the variables.

\begin{definition}
  Given two random variables \(X, Y\), we define their \emph{mutual information} as the Kullback-Leibler divergence of their joint distribution and the product of their marginals,
  \[
    MI(X;Y) = \KL{P_{X,Y}}{P_{X}P_{Y}}.
  \]
\end{definition}

\begin{definition}
  Given three random variables \(X, Y, Z\) we define the \emph{conditional mutual information} of \(X\) and \(Y\) over \(Z\) as
  \[
    MI(X;Y\mid Z) = \E{Z}{\KL{P_{X,Y \mid Z}}{P_{X\mid Z} P_{Y \mid Z}}}.
  \]
\end{definition}
Where \(MI(X;Y \mid Z) \geq 0\) and
\[
MI(X;Y \mid Z) = 0 \iff P_{X,Y \mid Z} = P_{X\mid Z} P_{Y \mid Z} \iff X\bigCI Y \mid Z.
\]
These values might be estimated using their empirical distributions, however, this \emph{empirical} mutual information will be typically greater than \(0\) even when \(X\bigCI Y \mid Z\). Thus, a threshold must be established.

A Bayesian approach would consist on comparing the model likelihood under independence and dependence hypothesis:
\[
  P_{indep}(x,y,z \mid \btheta) = P(x\mid z, \btheta)P(y \mid z, \btheta)P(z \mid \btheta)P(\btheta),
\]
\[
P_{dep}(x,y,z \mid \btheta) = P(x,y,z \mid \btheta)P(\btheta).
\]
Which means checking which assumptions is most probable to have generated the data.

