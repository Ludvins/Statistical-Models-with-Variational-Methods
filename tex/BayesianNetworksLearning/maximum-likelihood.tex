
Consider a set of i.i.d variables \(\bX = (X_{1},\dots,X_{N})\), when reviewing maximum likelihood training, we concluded that, in case \(P(x \mid \theta)\) is unconstrained, the maximum likelihood distribution corresponds to the empirical distribution.

However, Bayesian networks are constraint by:
 \[
   P(x_{1}, \dots, x_{N} \mid \theta) = \prod_{n = 1}^N P(x_n  \mid  pa(x_n), \theta).
 \]
This restriction requires to re-study the Kullback-Leibler divergence between the empirical
 distribution \(Q(x_1,\dots,x_N)\) and \(P(x_1, \dots, x_N \mid \theta)\) in order to extract the maximum likelihood value. Then,
 \[
   \begin{aligned}
   \KL{Q}{P} &= - \E{Q}{\sum_{n = 1}^N\log{P(x_n \mid pa(x_n), \theta)}} +
   \E{Q}{\sum_{n = 1}^N\log{Q(x_n \mid pa(x_n))}}
   \\ &= - \sum_{n = 1}^N \E{Q}{\log{P(x_n \mid pa(x_n), \theta)}} + \sum_{n =
     1}^N \E{Q}{\log{Q(x_n \mid pa(x_n))}}.
   \end{aligned}
 \]

Proposition~\ref{prop:expectation_over_marginal} might be used over each term of the summation, which says that as each term of the summation depends only on \((x_{n}, pa(x_{n}))\), the expectation distribution can be marginalized into them, resulting:
 \[
   \begin{aligned}
     \KL{Q}{P} &= \sum_{n = 1}^N \E{Q(x_n,pa(x_n))}{\log{Q(x_n \mid pa(x_n))}} - \E{Q(x_n,pa(x_n))}{\log{P(x_n \mid pa(x_n), \theta)}}\\
     &= \sum_{n = 1}^N \E{Q(x_n,pa(x_n))}{\KL{Q(x_n \mid pa(x_n))}{P(x_n \mid pa(x_n), \theta)}}.
   \end{aligned}
 \]

 Thus, the optimal setting is
 \[
   P(x_n \mid pa(x_n), \theta) = Q(x_n \mid pa(x_n)).
 \]
