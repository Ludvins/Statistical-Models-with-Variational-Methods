
Until this moment we have assumed that the data we are given is completed but in practice this data is not in two different ways. There may be unobserved or \emph{hidden} variables that affect the visible ones, and there may be \emph{missing} information, that is, states of visible variables that are missing.

Think about the example with the disease and the two habits we used in the last section, missing data would be a row in the table where some entry is missing, for example \(x_{3} = \{D = 1, A = 1\}\), where we know that this person got the disease and had habit \(A\) but we have no information about habit \(B\), this is an example of \emph{missing} data.

One approach to handle this situation would be marginalizing over that variable:
\[
  P(x_{3} \mid \theta) = \int_{b}P(d_{3}, a_{3}, b \mid \theta) = P(a_{3} \mid \theta_{A})\int_{b}P(b \mid \theta_{B})P(d_{3} \mid b, a_{3}, \theta_{D}).
\]

This leads to a non-factorized form of the posterior which is computationally difficult to handle, notice the problem is not conceptual but computational. Using the marginal to handle missing information does not always lead to this situation, in fact, marginalizing over a collider (\(D\) in our example) would mean loosing that variable as the integral simply equals \(1\).
\[
  P(x_{3} \mid \theta) = \int_{d}P(d, a_{3}, b_{3} \mid \theta) = P(a_{3} \mid \theta_{A})P(b_{3}\mid \theta_{B})\int_{d}P(d \mid b_{3}, a_{3}, \theta_{D}) = P(a_{3} \mid \theta_{A}) P(b_{3} \mid \theta_{B});
\]

There are three main types of missing data:
\begin{itemize}
  \item \textbf{Missing completely at random (MCAR)}. If the events that lead to any particular data to be missing is independent from both the observed and the unobserved variables, and occur at random.
  \item \textbf{Missing at random (MAR)}. When the absence is not random but can be explained with the observed variables.
  \item \textbf{Missing not at random (MNAR)}. The missing data is related with the reason why it is missing. For example, skipping a question in a survey for being ashamed of the answer.
\end{itemize}

To express this mathematically, split the variables \(\X\) into visible \(\X_{vis}\) and hidden \(\X_{hid}\), let \(M\) be a variable denoting that the state of the hidden variables is known \((0)\) or unknown \((1)\).
So the difference between the three types resides on how \(P(M = 1 \mid x_{vis}, x_{hid}, \theta)\) simplifies. When data is \emph{missing at random}, we assume that we can explain the missing information with the visible one, so the probability of being missing only depends on the visible data, that is
\[
  P(M = 1 \mid x_{vis}, x_{hid}, \theta) = P(M = 1 \mid x_{vis}),
\]
so that,
\[
  P(x_{vis}, M = 1 \mid \theta) = P(M = 1 \mid x_{vis})P(x_{vis} \mid \theta)
\]

Assuming the data is \emph{missing completely at random} is stronger, as we are supposing that there is no reason behind the missing data, so that it being missing is independent from the visible and hidden data:
\[
  P(M = 1 \mid x_{vis}, x_{hid}, \theta) = P(M = 1),
\]
so now
\[
    P( x_{vis}, M = 1 \mid \theta) = P(M = 1)P( x_{vis} \mid \theta).
\]
In both cases we may simply use the marginal \(P(x_{vis} \mid \theta)\) to assess parameters as \(P( x_{vis}, M = 1 \mid \theta)\) does not depend on the missing variables.

In case data is \emph{missing not at random}, no independence assumption is made over the probability of the data being unknown, meaning it depends on both the visible and the hidden information. From now on, we will assume missing information is either MAR or MCAR, even though this could lead to a misunderstanding of the problem as in the following simple example.

\begin{exampleth}
  Consider a situation where data is obtained from a survey where people are asked to choose between 3 options \(A, B\) and \(C\). Assume that no one chose option \(C\) because they are ashamed of the answer, and the answers are uniform between \(A\), \(B\) and not answering.

  Normalizing the missing information would lead to setting \(P(A \mid \V) = 0.5 = P(B \mid \V)\) and \(P(C \mid \V) = 0\) when the reasonable result is that not answering equals to choosing \(C\) so that \(P(A \mid V) = P(B \mid \V) = P(C \mid \V) = \frac{1}{3}\)
\end{exampleth}
