In this section Bayesian inference is used to solve an inference problem governed by a Bayesian network, the problem statement is the following:

A disease \(D\) and two habits \(A\) and \(B\) are being studied.  Consider the following i.i.d variables \(\{A_{1},\dots, A_{N}\}\), \(\{B_{1},\dots,B_{N}\}\) and \(\{D_{1},\dots, D_{N}\}\) governed by the parameters \(\theta_{A}, \theta_{B}\) and \(\btheta_{D}\) as shown in figure~\ref{fig:bayesian_example}. Let \(N = 7\) be the number of observations of the variables as shown in Table~\ref{tab:bn_ex} and \( \bx = \{(a_n, b_n, d_n),\ n=1,\dots,N\} \) the set of observations.

All the variables are binary satisfying
\[
  P(A_{n} = 1 \mid \theta_{A}) = \theta_{A}, \quad P(B_{n} = 1 \mid \theta_{B}) = \theta_{B} \quad \forall n=1,\dots,N,
\]
\[
  P(D_{n} = 1 \mid A_{n} = 0, B_{n} = 1, \btheta_{D}) = \theta_{1}, \quad \forall n = 1,\dots,N,
\]
\[
  \btheta_{D} = ( \theta_{0}, \theta_{1}, \theta_{2},\theta_{3}).
\]
Where a binary to decimal transformation between the states of \(A\) and \(B\) and the sub-index of \(\theta\) is being used.

\begin{figure}[!ht]
  \begin{tabular}{*{2}{>{\centering\arraybackslash}b{\dimexpr0.5\linewidth-2\tabcolsep\relax}}}
  \centering
  \begin{tikzpicture}[
    node distance=1cm and 0.5cm,
    mynode/.style={draw,circle,text width=0.5cm,align=center}
    ]

    \node[mynode] (d) {\(D_{n}\)};
    \node[mynode, above left=of d] (a) {\(A_{n}\)};
    \node[mynode, above right=of d] (b) {\(B_{n}\)};
    \node[mynode, above=of a] (ta) {\(\theta_{A}\)};
    \node[mynode, above=of b] (tb) {\(\theta_{B}\)};
    \node[mynode, below=of d] (td) {\(\btheta_{D}\)};
    \plate[inner sep=.3cm,xshift=.02cm,yshift=.2cm]  {} {(d)(a)(b)} {\(n= 1\dots N\)}; %
    \path (a) edge[-latex] (d)
    (b) edge[-latex] (d)
    (ta) edge[-latex] (a)
    (tb) edge[-latex] (b)
    (td) edge[-latex] (d)
    ;

  \end{tikzpicture}
    \caption{Bayesian parameter model for the relation between \(A,B\) and \(D\).}\label{fig:bayesian_example}
    &
      \renewcommand{\arraystretch}{1.3}
      \begin{tabular}{ccc}
        \hline
        A & B & D \\ \hline
        \(1\) & \(1\) & \(1\) \\
        \(1\) & \(0\) & \(0\) \\
        \(0\) & \(1\) & \(1\) \\
        \(0\) & \(1\) & \(0\) \\
        \(1\) & \(1\) & \(1\) \\
        \(0\) & \(0\) & \(0\) \\
        \(1\) & \(0\) & \(1\) \\ \hline
  \end{tabular}\captionof{table}{Set of observations, where \(1\) means true and \(0\) means false.}\label{tab:bn_ex}
\end{tabular}
\end{figure}

The graph gives the following joint probability distribution the variables:
\[
P(a_{n},b_{n},d_{n}, \theta_A, \theta_B, \theta_D) = P(d_{n}\mid a_{n},b_{n},\theta_D)P(a_{n} \mid \theta_A)P(b_{n} \mid \theta_A).
\]
A prior distribution must be specified and since dealing with multidimensional continuous
distributions is computationally problematic, it is usual to use uni-variate
distributions.

\section{Global and local parameter independence}

A convenient assumption is that the prior factorizes, this is usually called
\emph{global parameter independence}:
\[
  P(\theta_{A}, \theta_{B}, \btheta_{D}) = P(\theta_{A})P(\theta_{B})P(\btheta_{D}).
\]
Under this assumption, the joint probability factorizes as
\[
  P(\bx, \theta_{A}, \theta_{B}, \btheta_{D}) = P(\theta_{A})P(\theta_{B})P(\btheta_{D})\prod_{n}P(a_{n}\mid \theta_{A})P(b_{n} \mid \theta_{B})P(d_{n}\mid a_{n}, b_{n}, \btheta_{D}).
\]
The posterior distribution is then
\[
  \begin{aligned}
    P(\theta_{A}, \theta_{B}, \btheta_{D}\mid \bx)
    &= \frac{P(\bx \mid \theta_{a}, \theta_{B}, \btheta_{D})P(\theta_{A}, \theta_{B}, \btheta_{D})}{P(\bx)} =\frac{P(\bx \mid \theta_{A}, \theta_{B}, \btheta_{D})P(\theta_{A}) P(\theta_{B})P( \btheta_{D})}{P(\bx)}\\
    &= \frac{1}{P(\bx)}P(\theta_{A})P(\theta_{B})P(\btheta_{D})\prod_{n=1}^{N}P(a_{n}\mid \theta_{A})P(b_{n}\mid \theta_{B})P(d_{n}\mid a_{n}, b_{n},\btheta_{D})\\
    &= P(\theta_{A} \mid \bx_{A} )P(\theta_{B}\mid \bx_{B})P(\btheta_{D} \mid \bx).
  \end{aligned}
\]

Where \(\bx_{A}\) is the subset of observations related to the variable \(A\). If we further assume that \(P(\btheta_{D})\) factorizes as
\(P(\btheta_{D}) = P(\theta_{0})P(\theta_{1})P(\theta_{2})P(\theta_{3})\),
which is called \emph{local parameter independence}, \(P(\btheta_{D} \mid \bx)\) factorizes as
\[
  P(\btheta_{D}\mid \bx) = P(\theta_{0} \mid \bx )P(\theta_{1} \mid \bx )P(\theta_{2} \mid \bx )P(\theta_{3} \mid \bx ).
\]

\section{Learning binary variables}

The simplest cases to continue are \(P(\theta_{A} \mid \bx_{A})\) and
\(P(\theta_{B} \mid \bx_{B})\) since they require only a uni-variate prior distribution
\(P(\theta_{A})\) or \(P(\theta_{B})\). The procedure is shown using \( \theta_A \) and it is analogous when using \( \theta_B \). 

The posterior is
\[
  P(\theta_{A} \mid \bx_{A}) = \frac{1}{P(\bx_{A})}P(\theta_{A})\theta_{A}^{\#(A=1)}{(1-\theta_{A})}^{\#(A=0)}.
\]
The most convenient choice for the prior is a Beta distribution as conjugacy
will hold:
\[
  \theta_{A} \sim \text{Beta}(\alpha_{A}, \beta_{A}) \implies P(\theta_{A})  = \frac{1}{B(\alpha_{A}, \beta_{A})}\theta_{A}^{\alpha_{A}-1}(1-\theta_{A})^{\beta_{A} - 1}.
\]
Therefore, it follows that
\[
  \theta_{A} \mid \bx_{A} \sim \text{Beta}(\alpha_{A} + \#(A=1), \beta_{A} + \#(A = 0)).
\]

The predictive marginal is then
\[
  \begin{aligned}
    P(A = 1 \mid \bx_{A})
    &= \frac{P(A = 1, \bx_{A})}{P(\bx_{A})} = \int_{\theta_{A}}  \frac{P(A = 1, \bx_{A}, \theta_{A})}{P(\bx_{A})} =  \int_{\theta_{A}}  \frac{P(A = 1 \mid \bx_{A}, \theta_{A}) P(\bx_{A}, \theta_{A})}{P(\bx_{A})} \\
    &=  \int_{\theta_{A}}  \frac{P(A = 1 \mid \bx_{A}, \theta_{A}) P(\theta_{A} \mid \bx_{A})P(\bx_{A})}{P(\bx_{A})} \\
    &= \int_{\theta_{A}}P(\theta_{A}\mid \bx_{A})\theta_{A} = \E{}{\theta_{A} \mid \bx_{A}} \\
    &= \frac{\alpha_{A} + \#(A= 1)}{\alpha_{A} + \#(A=1) + \beta_{A} + \#(A=0)}.
  \end{aligned}
\]
Where the last equality is given by the expected value of a Beta distribution.

For \(P(d \mid a ,b)\) the situation is more complex, the simplest approach
is to specify a Beta prior for each of the components of \(\btheta_{D}\).
Focus on \(\theta_{2}\), notice the parameters \(\alpha\)
and \(\beta\) we used before now do depend on \(A\) and \(B\), these
are called \emph{hyperparameters}.
\[
  \theta_{2} \sim Beta\Big(\alpha_{D}(1,0) + \#(D = 1, A = 1, B = 0), \ \beta_{D}(1,0) + \#(D = 0, A = 1, B = 0)\Big).
\]

Repeating the procedure we used with \(A\) we get that
\[
  P(D = 1 \mid A = 1, B = 0, \bx) = \frac{\alpha_{D}(1,0) + \#(D = 1, A = 1, B = 0)}{\alpha_{D}(1,0) + \beta_{D}(1,0) + \#(A=1, B = 0)}.
\]

All hyperparameters could be set to the same
value, where a complete ignorance prior would correspond to set them to 1.

There are two limit possibilities depending on the amount of data available.
\begin{itemize}
  \item \textbf{No data}. The marginal probability corresponds to the prior, which
in the last case is
    \[
    P(D = 1 \mid A = 1, B = 0, \bx) = \frac{\alpha_{D}(1,0)}{\alpha_{D}(1,0) + \beta_{D}(1,0)}.
    \]
    Note that equal hyperparameters would give a result of \(0.5\).\newline
   
  \item \textbf{Infinite data}. When infinite data is available, the marginal is generally dominated by it,
    this corresponds to the Maximum Likelihood solution.
    \[
    P(D = 1 \mid A = 1, B = 0, \bx) = \frac{\#(D = 1, A = 1 , B = 0)}{\#(A = 1, B = 0)}.
    \]
    This happens unless the prior has a pathologically strong effect.
\end{itemize}

 Consider the data given in the table in figure \ref{fig:bayesian_example}, and
 equal parameters and hyperparameters \(1\) and a prior belief that any setting is equally probable, i.e, \( P(A=1) = 0.5\) . 
 
 We may illustrate the different results that are obtained using using Bayesian inference and Maximum likelihood training. The former is
 \[
   P(A = 1 \mid \bx) = \frac{1 + \#(A = 1)}{2 + N} = \frac{5}{9} \approx 0.556.
 \]
 and the latter is \(4/7 = 0.571\). In conclusion, the Bayesian
 result is more prudent than this one, which fits in with our prior belief.

 \section{Learning discrete variables}

 The natural generalization to more than two states is using a Dirichlet
 distribution as prior, assuming i.i.d data, local and global prior
 independence. Two different scenarios are being considered, firstly one where the
 variable has no parents, as the case for \(A\) and \(B\) in the previous
 example. And secondly a variable with a non-void set of parents.

 Let \(\bx\) denote the given dataset as observations of a random variable \(X\), and \(\btheta\) the set of parameters modeling the experiment.

 \subsection{No parents}

 Consider a variable \(X \sim Categorical(\btheta)\) with
 \(Dom(X) = \{1, \dots, I\} \) and \(\btheta = (\theta_{1},\dots, \theta_{I})\) so that
 \[
   P(x) = \prod_{i = 1}^{I}\theta_{i}^{\mathbb{I}[x = i]} \text{
     with  } \sum_{i=1}^{I}\theta_{i} = 1.
 \]
 So that the posterior is
 \[
   P(\btheta \mid \bx) = \frac{1}{P(\bx)} P(\btheta) \prod_{n = 1}^{N}\prod_{i =1 }^{I}\theta_{i}^{\mathbb{I}[x_{n} = i]} =  \frac{1}{P(\bx)} P(\btheta) \prod_{i = 1}^{I} \theta_{i}^{\sum_{n} \mathbb{I}[x_{n}=i]}.
 \]
 Assuming a Dirichlet prior \(\btheta \sim Dirichlet(\bm{u})\) with \( \bm{u} = (u_{1}, \dots, u_{I})\) and using the fact that the Dirichlet distribution is the conjugate prior of the Categorical distribution (Appendix~\ref{ap:C-D}), the posterior follows a Dirichlet distribution
 \[
   \btheta \mid \bx \sim Dirichlet(\bm{u} + \bm{c}).
 \]
 Where
 \[
   \bm{c} = \Big( \sum_{n=1}^{N}\mathbb{I}[x_{n} = 1],\dots,  \sum_{n=1}^{N}\mathbb{I}[x_{n} = I]\Big).
 \]
 The predictive marginal is then given by
 \[
   \begin{aligned}
     P(X=i \mid \bx) &= \int_{\theta}P(X=i \mid \theta)P(\theta \mid \bx) =  \int_{\theta}\theta_{i}P(\theta \mid \bx)\\
     &=  \int_{\theta_{i}}\theta_{i}P(\theta_{i} \mid \bx) = \E{}{\theta_{i} \mid \bx}.
   \end{aligned}
 \]
 Where we used that
 \[
   \int_{\theta_{j \neq i}}\theta_{i} P(\theta \mid \bx) = \theta_{i}\prod_{k\neq j}P(\theta_{k} \mid \bx) \int_{\theta_{j}}P(\theta_{j}\mid \bx) = \theta_{i} \prod_{k \neq j}P(\theta_{k} \mid \bx).
 \]
From Proposition~\ref{prop:dirichlet_marginal}, the uni-variate marginal of a Dirichlet distribution is a Beta distribution of parameters
\[
  \theta_{i} \mid \bx \sim \text{Beta}(u_{i} + c_{i}, \sum_{j\neq i} u_{j} + c_{j}).
\]
Using the expectation of a Beta distribution we get that
\[
  P(X = i \mid \bx) = \frac{u_{i} + c_{i}}{\sum_{j}u_{j} + c_{j}}.
\]

\subsection{Parents}
Consider now that \(X\) has a set of parent variables \(pa(X)\), in this case,
we want to compute the marginal given a state of its parents and the data:
\[
  P(X = i \mid pa(X) = \bm{j}, \bx).
\]
We are using the following notation for the parameters
\[
  P(X = i \mid pa(X) = \bm{j}, \theta) = \theta_{i,\bm{j}}, \quad \bm{\theta_{j}} = (\theta_{1,\bm{j}},\dots, \theta_{I,\bm{j}}).
\]
Local independence means that
\[
  P(\bm{\theta}) = \prod_{j}P(\bm{\theta_{j}}).
\]

As we did before, we consider a Dirichlet prior
\[
  \bm{\theta_{j}} \sim Dirichlet(\bm{u_{j}}),
\]

the posterior is then
\[
  \begin{aligned}
    P(\bm{\theta} \mid \bx) &= \frac{P(\bm{\theta})P(\bx \mid \bm{\theta}) }{P(\bx)} = \frac{1}{P(\bx)}\Big(\prod_{\bm{j}}P(\bm{\theta_{j}}) \Big)P(\bx \mid \bm{\theta}) \\
    &= \frac{1}{P(\bx)}\Big(\prod_{\bm{j}}\frac{1}{B(\bm{u_{j}})}\prod_{i}\theta_{i,\bm{j}}^{u_{i,\bm{j}}-1}\Big) P(\bx \mid \bm{\theta})\\
    &= \frac{1}{P(\bx)}\Big(\prod_{\bm{j}}\frac{1}{B(\bm{u_{j}})}\prod_{i}\theta_{i,\bm{j}}^{u_{i,\bm{j}}-1}\Big) \Big(\prod_{n}\prod_{\bm{j}}\prod_{i} \theta_{i,\bm{j}}^{\mathbb{I}[x_{n} = i, pa(x_{n}) = \bm{j}]}\Big)\\
    &= \frac{1}{P(\bx)}\prod_{\bm{j}}\frac{1}{B(\bm{u_{j}})}\prod_{i}\theta_{i,\bm{j}}^{u_{i,\bm{j}}-1 + \#(X = i,pa(X)=\bm{j})}.
  \end{aligned}
\]
Naming \(\bm{v_{j}} = \bm{u_{j}} + \#(X = i, pa(X) = \bm{j})\) and using the same argument as we did in the \emph{no parents} case with the normalization constants, the posterior is
\[
  (\bm{\theta} \mid \bx) \sim \prod_{j}\text{Dirichlet}(\bm{v_{j}}).
\]
Denoting \(v_{i,j}\) the components of \(\bm{v_{j}}\), the marginal is
\[
  P(X=i, pa(X) = \bm{j}, \bx) = \frac{v_{i,j}}{\sum_{i}v_{i,j}}.
\]
Notice all the above has been done using a fixed variable \(X\), so that all the parameters depend on that variable.
