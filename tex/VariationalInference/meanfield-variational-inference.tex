\section{The mean-field variational family}

The \emph{mean-field variational family} \(\mathcal{Q}\) is defined as the family of distributions where the variables are mutually independent, i.e, any \(Q \in \mathcal{Q}\) verifies
\[
  Q(\bz) = \prod_{m=1}^{M}Q_{m}(z_{m}),
\]

where \(\bZ = \{Z_{1},\dots,Z_{M}\}\) is the considered set of variables. The mean-field family is commonly used to model the family of distributions over the latent variables in our optimization problem. Notice that each \emph{factor} \(Q_{m}\) can be different and this family does not depend on the observed data.

The mean-field family can capture any marginal of the latent variables but not correlation between them, as it assumes they are independent. For example, consider a two dimensional Gaussian distribution where a high percentage of the density is inside the blue ellipse shown in Figure~\ref{fig:mean_field}. Any mean-field approximation would factorize as a product of two Gaussian distributions, condensing its density in a circle as shown in purple.

\begin{figure}[h!]
\centering
\begin{tikzpicture}
  \pgfplotsset{ticks=none}
    \begin{axis}[
        axis lines = middle,
        xmin=-11, xmax=11, ymin=-11, ymax=11,
        axis equal,
        xlabel = $z_{1}$,
        ylabel = {$z_{2}$},
        yticklabels={,,},
        ]
        \filldraw[rotate around={45:(110,110)}, color=blue!50, fill=blue!20, very thick, opacity=0.7] (110,110) ellipse (70 and 20);
        % \addlegendentry{Exact Posterior}
        \filldraw[color=purple!50, fill=purple!20, very thick, opacity=0.7]  (110,110) circle (25);
        % \addlegendentry{Mean-field Approximation}
    \end{axis}
  \end{tikzpicture}
  \caption{Mean-field family distribution (purple) approximating a Gaussian distribution (blue).}\label{fig:mean_field}
\end{figure}

Notice the parametric form of each factor \(Q_{m}\) is not specified and the appropriate configuration depends on the variable. For example, a continuous variable might have a Gaussian factor and a categorical variable have a categorical factor.

\section{CAVI algorithm }

In this section, we describe a widely used algorithm to solve the optimization problem we discussed in the previous section using the mean-field family. It is \emph{coordinate ascent variational inference} or \emph{CAVI} (also known as \emph{Variational Bayes}) and its procedure is to iteratively optimize each factor of the mean-field family distribution, while fixing the others. With this, the ELBO reaches a local optimum.

The \emph{coordinate ascent} approach is similar to the \emph{gradient ascent} in that both attempt to reach a local optima following a iterative procedure and maximizing each step. They differ in that the former updates each \emph{coordinate} (variable in this context) whereas the latter updates all of them using the direction of the gradient.

CAVI might be seen as a generalization of the EM algorithm, where the algorithm is extended to a full Bayesian approach, considering the parameters as latent variables. This will be reviewed in Section~\ref{sec:CAVI_EM}.

 \begin{algorithm}[t]
  \SetAlgoLined\KwData{A distribution \(P(\bx , \bm{z})\) with a dataset \(\bx\).}
  \KwResult{A distribution of the mean-field family \(Q(\bm{z}) = \prod_{m=1}^{M}Q_{m}(z_{m})\)}
  Initialize \(Q(\bm{z})\)\;
  \While{Convergence stop criteria}{
    \For{\(m \in 1,\dots,M\)}{
      \(\bm{z}_{\backslash m} = (z_{1},\dots,z_{m-1},z_{m+1},\dots,z_{M})\)\;
      Set \(Q_{m}(z_{m}) \propto \exp{\E{Q_{\backslash m}}{\log{P(z_{m} \mid \bm{z}_{\backslash m}, \bx)}}}\)\;
    }
    Compute \(ELBO(Q)\)\tcp*{Used for convergence criteria.}
  }
  \KwRet{\(Q\)}\;
  \caption{Coordinate Ascent Variational Inference}\label{alg:cavi}
\end{algorithm}


Let \(\bx\) be the given observations of the observed variables. CAVI iterates fixing all hidden variable but one at a time and maximizing its contribution to the ELBO. Consider the \(m^{th}\) variable \(Z_{m}\), denoting by \(\backslash m\) full set of indexes without the \(m^{th}\), then \(\bm{Z}_{\backslash m}\) is the full set of variables without the focused one. Let the factors \(Q_{n}, n\neq m\) be fixed.

The contribution of \(Z_{m}\) to the ELBO is (summarizing other factors in the constant term):
\[
  \begin{aligned}
    ELBO(Q) &= \E{Q}{\log{P(\bx, \bz)}} - \E{Q}{\log{Q(\bz)}}\\
    &\stackrel{1}{=} \E{Q_{m}}{\E{Q_{\backslash m}}{\log{P(\bx, \bm{z})}}} - \E{Q_{m}}{\log{Q_{m}(z_{m})}} + \text{ const. }\\
    &\stackrel{2}{=} \E{Q_{m}}{\E{Q_{\backslash m}}{\log{P(z_{m} \mid \bm{z}_{\backslash m}, \bx)} + \log{P(\bm{z}_{\backslash m}, \bx)}}} - \E{Q_{m}}{\log{Q_{m}(z_{m})}} + \text{ const. }\\
    &\stackrel{3}{=}  \E{Q_{m}}{\E{Q_{\backslash m}}{\log{P(z_{m} \mid \bm{z}_{\backslash m}, \bx)}}} - \E{Q_{m}}{\log{Q_{m}(z_{m})}} + \text{ const. }\\
    &\stackrel{4}{=} - \KL{Q_{m}(z_{m})}{  \exp{\E{Q_{\backslash m}}{\log{P(z_{m} \mid \bm{z}_{\backslash m}, \bx)}}} } + \text{ const.}.
  \end{aligned}
\]

\begin{enumerate}
  \item The expectations in the ELBO formula are separated. The logarithm factorizes as \(\log Q(z) = \sum_{m=1}^M \log Q_m(z_m)\).  The constant term comes from \( \E{Q_{\backslash m}}{\log Q_{\backslash m}(z_{\backslash m})} \).
  \item \( P \) is separated as \( P(\bz, \bx) = P(z_m \mid \bz_{\backslash m}, \bx)P(\bz_{\backslash m}, \bx) \implies \log P(\bz, \bx) = \log P(z_m \mid \bz_{\backslash m}, \bx) + \log P(\bz_{\backslash m}, \bx)\).
  \item \( \E{Q_{m}}{\E{Q_{\backslash m}}{\log{P(\bm{z}_{\backslash m}, \bx)}}} = \E{Q_{\backslash m}}{\log{P(\bm{z}_{\backslash m}, \bx)}}\) is constant.
  \item Applied Kullback-Leibler definition.
\end{enumerate}

Maximizing the ELBO is equivalent to minimize the given Kullback-Leibler divergence and this divergence is zero when \(Q_{m}^{new}\) is:
\[
  Q_{m}^{new}(z_{m}) \propto \exp{\E{Q_{\backslash m}}{\log{P(z_{m} \mid \bm{z}_{\backslash m}, \bx)}}}.
\]

Notice that the proportionality restriction is enough to fully determine the distribution as its integral is normalized. Equivalently, the distribution is proportional to
\begin{equation}\label{eq:cavi_update}
    Q_{m}^{new}(z_{m}) \propto \exp{\E{Q_{\backslash m}}{\log{P(z_{m}, \bm{z}_{\backslash m}, \bx)}}}.
\end{equation}

As the ELBO is generally a non-convex function and the CAVI algorithm converges to a local optimum, the initialization values of the algorithm play an important role on its performance. The convergence criteria is usually a threshold for the ELBO or a fixed amount of iterations.

\section{CAVI as an EM generalization}\label{sec:CAVI_EM}

As CAVI considers parameters as hidden variables, we need to specify a variational distribution for them. We are choosing the distribution that summarizes the information in the optimal point, let \(\btheta_{opt}\) be the optimal value of \(\btheta\):
\[
  Q(\btheta) = \delta(\btheta - \btheta_{opt}).
\]
The variational distribution factorizes as
\[
  Q(\bz, \btheta) = Q(\bz)Q(\btheta).
\]
The lower bound takes the form
\[
  \begin{aligned}
    \log{P(\bx \mid \btheta)} &\geq \E{Q(\bz, \btheta)}{\log{P(\bx, \bz, \theta)}} - \E{Q(\bz, \btheta)}{\log{Q(\bz, \btheta)}} \\
    &= \E{Q(\bz)}{\E{Q(\btheta)}{\log{P(\bx, \bz, \btheta)}}} - \E{Q(\bz)}{\log{Q(\bz)}} - \E{Q(\btheta)}{\log{Q(\btheta)}} \\
    &= \E{Q(\bz)}{\log{P(\bx, \bz, \btheta_{opt})}} - \E{Q(\bz)}{\log{Q(\bz)}} + \text{ const. }
  \end{aligned}
\]
The CAVI update can be seen as an iterative two step process. Firstly, given a fixed \(Q(\bz)\), as the distribution class of \(Q(\btheta)\) is fixed, optimizing it is equivalent to find the optimal parameter \(\btheta_{opt}\). 
\[
  \begin{aligned}
    \btheta_{opt} &= \argmax_{\btheta} \Big( \E{Q(\bz)}{\log{P(\bx,\bz,\btheta)}} \Big)\\
    &=  \argmax_{\btheta} \Big( \E{Q(\bz)}{\log{P(\bx, \bz \mid \btheta)P(\btheta)}} \Big) \\
    &= \argmax_{\btheta} \Big( \E{Q(\bz)}{\log{P(\bx, \bz \mid \btheta)}} + \log{P(\btheta)} \Big). \\
  \end{aligned}
\]

If we take a flat prior, this term is equivalent to the M-step. Secondly, given a fixed parameter \(\btheta\), the update is
\[
  Q(\bz) \propto \exp \E{Q(\btheta)}{\log P(\bz \mid \bx, \btheta)} =  P(\bz \mid \bx, \btheta_{opt}),
\]
which is the E-step of the EM algorithm.
