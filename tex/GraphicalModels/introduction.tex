
\emph{Probabilistic graphical models} are diagrammatic representations of probability distributions that represent the dependence/independence relation between the considered variables. Their use presents the following advantages (\cite{bishop2006pattern}):
\begin{enumerate}\setlength{\itemsep}{0.2cm}
  \item They allow a simple visualization of probabilistic models.
  \item Insights into the model's properties can be obtained form the graph.
  \item Complex computations, required in inference task, are simplified using the graph structure.
\end{enumerate}

In probabilistic graphical models, each node represents a random variable and links represents relations between them. The graph encodes how the joint probability distribution of the considered variables factorizes. The different classes of graphical models differ in how they represent these relations and how the distribution is factorized.

We begin discussing \emph{Bayesian networks}, also known as \emph{directed graphical models}, where links between variables are indicated by a directed arrow.  The other major class of graphical models are \emph{Markov random fields} or \emph{undirected graphical models} in which links have no directional meaning. The former are useful to express casual relations between the variables whereas the latter express soft constraints between the variables.

Consider a set of variables \(\bX = (X_{1},\dots, X_{N})\), the possible ways these variables can interact is extremely large, for binary variables, the joint distribution table would take \(O(2^{N})\) space, which is unpractical when the amount of variables scales. When dealing with this such large distributions it is a common practice to factorize the joint probability in a graphical model, reducing the needed resources to deal with the inference problem.
