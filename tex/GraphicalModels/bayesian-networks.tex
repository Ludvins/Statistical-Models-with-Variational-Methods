
Given a set of variables \(\bX = (X_{1},\dots,X_{N})\), \emph{Bayesian networks} might be defined either as a probability distribution of a certain form or a DAG whose nodes represent these variables and links an independence constraint. Both ideas are present in the following definition.

\begin{definition}
  A \emph{belief network or Bayesian network} is a pair \((G,P)\) formed by a DAG \(G\) and  joint probability distribution \(P\) such that, there is a correspondence between variables and nodes such as:
  \[
    P(x_{1},\dots,x_{N}) = \prod_{n=1}^{N}P(x_{n}\mid pa(x_{n})).
  \]
\end{definition}

\begin{remark}
  A Bayesian network might be given as a distribution from which the DAG can be constructed or a DAG which represents the distribution. For example in Figure~\ref{fig:bn_example}, given the DAG one could easily define the joint distribution and conversely.
\end{remark}

\begin{figure}[h!]
  \centering
  \begin{tikzpicture}[
    node distance=1.5cm and 1.5cm,
    mynode/.style={draw,circle,text width=0.5cm,align=center}
    ]

    \node[mynode] (1) {\(X_1\)};
    \node[mynode,right=of 1] (2) {\(X_2\)};
    \node[mynode,right=of 2] (3) {\(X_3\)};
    \node[mynode,right=of 3] (4) {\(X_4\)};

    \path (4) edge[-latex][bend right] (1)
    (3) edge[-latex] (2)
    (4) edge[-latex][bend right] (2)
    ;

    \end{tikzpicture}
    \captionof{figure}{Bayesian Network factorizing \(P(x_1, x_2, x_3, x_4) = P(x_1 \mid x_4)P(x_2\mid x_3, x_4)P(x_3)P(x_4)\)}\label{fig:bn_example}
\end{figure}

Any probability distribution can be written as a Bayesian Network, even though
it may end up been a fully-connected ``cascade''\footnote{This term comes from the visual structure of the graph.} DAG, which means that each variable \( X_n \) is a parent of any \( X_m \) with \( m > n \). This is because any distribution satisfies:
\[
   P(x_1, \dots, x_{N}) = P(x_1) \prod_{n=2}^{N}P(x_{n} \mid x_{1},\dots, x_{n-1})
 \]

Bayesian Networks are good for encoding \emph{conditional independence} over the
variables, but are not for encoding dependence. For example, with the following
network
\[
P(x,y) = P(y\mid x)P(x).
\]
represented as \(x \to y\) in a DAG, it may appear to encode dependence between both variables but the conditional \(P(y\mid x)\) could happen to equal \(P(y)\), giving \(P(x,y) = P(x)P(y)\).

How could we check if two variables are conditionally independent given a
Bayesian network? For example in Figure~\ref{fig:relations}, \(X_1 \bigCI
X_2 \mid X_4\) as
\[
\begin{aligned}
P(x_2 | x_4) &= \frac{1}{P(x_4)}\int_{x_1,x_3}P(x_1, x_2, x_3, x_4)
= \frac{1}{P(x_4)}\int_{x_1,x_3}P(x_1|x_4)P(x_2|x_3,x_4)P(x_3)P(x_4)\\
                 &= \int_{x_3}P(x_2|x_3, x_4)P(x_3),
\end{aligned}
\]
\[
\begin{aligned}
P(x_1, x_2 | x_4) &= \frac{1}{P(x_4)}\int_{x_3}P(x_1, x_2, x_3, x_4)
= \frac{1}{P(x_4)}\int_{x_3}P(x_1|x_4)P(x_2|x_3,x_4)P(x_3)P(x_4)\\
                 &= P(x_1|x_4)\int_{x_3}P(x_2|x_3, x_4)P(x_3) = P(x_1|x_4)P(x_2|x_4).
\end{aligned}
\]

\begin{proposition}\label{prop:bn_neig_indep}
  Let \(X\) and \(Y\) be two different nodes of a Bayesian network such as \(Y \not \in ne(X)\), then
  \[
    X \bigCI Y \mid ne(X).
  \]
  Which means that all information about \(X\) is in its neighbors.
\end{proposition}

\section{D-separation and D-connection}

Now we are going to define two central concepts to determine conditional
independence in any Bayesian network, these are \emph{d-connection} and \emph{d-separation}.

\begin{definition}
Let \(G\) be a DAG where \(\bm{X}, \bm{Y} \text{ and } \bm{Z}\)
are disjoint sets of vertices. We say that \(\bm{X} \text{ and
} \bm{Y}\) are \emph{d-connected} by \(\bm{Z}\) if and only if there
exists an undirected path \(U\) from any vertex in \(\bm{X}\) to any
vertex in \(\bm{Y}\) such that:
\begin{itemize}
\item For any collider \(C\), itself or any it's descendants is in \(\bm{Z}\).
\item No non-collider on \(U\) is on \(\bm{Z}\).
\end{itemize}
\end{definition}

\begin{definition}
Let \(G\) be a DAG where \(\bm{X}, \bm{Y} \text{ and } \bm{Z}\)
are disjoint sets of vertices. \(\bm{X}\) and \(\bm{Y}\)
are \emph{d-separated} by \(\bm{Z}\) if and only if they are not
d-connected by \(\bm{Z}\) in \(G\)
\end{definition}

\begin{figure}[t]
\centering
\begin{tikzpicture}[
  node distance=1cm and 0.5cm,
  mynode/.style={draw,circle,text width=0.5cm,align=center}
]

\node[mynode] (a) {a};
\node[mynode, below right=of a] (d) {d};
\node[mynode,above right=of d] (b) {b};
\node[mynode, below right=of b] (e) {e};
\node[mynode,above right=of e] (c) {c};

\path (a) edge[-latex] (d)
(b) edge[-latex] (d)
(c) edge[-latex] (e)
(b) edge[-latex] (e)
;

\end{tikzpicture}
\caption{D-separation example}\label{fig:d-sep}
\end{figure}

For example, in Figure~\ref{fig:d-sep}, \(d\) d-separates \(a\) and \(c\) (\(e\)
is a collider in the path that is not in \(\{d\}\)),
and \(\{d,e\}\) d-connect them.

\begin{theorem}[\cite{pearl_and_detcher}]\label{th:d-separation}
Let \(G\) be a DAG where \(\bm{X}, \bm{Y} \text{ and } \bm{Z}\)
are disjoint sets of vertices. If  \(\bm{X}\) and \(\bm{Y}\)
are d-separated by \(\bm{Z}\), then they are independent conditional
on \(\bm{Z}\) in all probability distributions that \(G\) may represent.
\end{theorem}

The Bayes Ball algorithm (\cite{bayes_ball}) provides a linear time complexity
algorithm that computes conditional independent using this theorem.


In cases where the Bayesian networks contains i.i.d nodes that are
essentially the same but repeated a number of times, the \emph{plate notation} is commonly used to represent this nodes in a compacted manner (Figure~\ref{fig:plate_notation}).

\begin{figure}[t]
\centering
\begin{tikzpicture}[
  node distance=1cm and 0.5cm,
  mynode/.style={draw,circle,text width=0.5cm,align=center}
]

\node[mynode] (a) {A};
\node[mynode,below=of a] (d) {\(B_3\)};
\node[mynode,left=of d] (c) {\(B_2\)};
\node[mynode,left=of c] (b) {\(B_1\)};
\node[mynode,right=of d] (e) {\(\dots\)};
\node[mynode,right=of e] (f) {\(B_n\)};

\node[mynode,right=2cm of f] (g) {\(B_i\)};
\node[mynode, above=of g] (h) {A};
\plate{} {(g)} {\(n\)}; %


\path (a) edge[-latex] (b)
(a) edge[-latex] (c)
(a) edge[-latex] (d)
(a) edge[-latex] (e)
(a) edge[-latex] (f)
(h) edge[-latex] (g)
;

\end{tikzpicture}
\caption{Plate notation example. Standard notation on the left and plate on the right.}\label{fig:plate_notation}
\end{figure}


\begin{exampleth}
In this example we are modeling three discrete random variables: Sprinkler (\(S\)),
Rain (\(R\)) and Grass wet (\(G\)).

The joint probability function is:
\[
P(s,r,g) = P(s|r)P(g|s,r)P(r)
\]

The following DAG illustrates the Bayesian network among with the probability
tables we are using.

\begin{figure}[ht]
\begin{tikzpicture}[
  node distance=0.6cm and 0cm,
  mynode/.style={draw,ellipse,text width=1.7cm,align=center}
]
\node[mynode] (sp) {Sprinkler};
\node[mynode,below right=of sp] (gw) {Grass wet};
\node[mynode,above right=of gw] (ra) {Rain};
\path (ra) edge[-latex] (sp)
(sp) edge[-latex] (gw)
(gw) edge[latex-] (ra);
\node[left=0.5cm of sp]
{
\begin{tabular}{cm{1cm}m{1cm}}
\toprule
& \multicolumn{2}{c}{Sprinkler} \\
Rain & \multicolumn{1}{c}{T} & \multicolumn{1}{c}{F} \\
\cmidrule(r){1-1}\cmidrule(l){2-3}
F & 0.4 & 0.6 \\
T & 0.01 & 0.99 \\
\bottomrule
\end{tabular}
};
\node[right=0.5cm of ra]
{
\begin{tabular}{m{1cm}m{1cm}}
\toprule
\multicolumn{2}{c}{Rain} \\
\multicolumn{1}{c}{T} & \multicolumn{1}{c}{F} \\
\cmidrule{1-2}
0.2 & 0.8 \\
\bottomrule
\end{tabular}
};
\node[below=0.5cm of gw]
{
\begin{tabular}{ccm{1cm}m{1cm}}
\toprule
& & \multicolumn{2}{c}{Grass wet} \\
\multicolumn{2}{l}{Sprinkler Rain} & \multicolumn{1}{c}{T} & \multicolumn{1}{c}{F} \\
\cmidrule(r){1-2}\cmidrule(l){3-4}
F & F & 0.0 & 1.0 \\
F & T & 0.8 & 0.2 \\
T & F & 0.9 & 0.1 \\
T & T & 0.99 & 0.01 \\
\bottomrule
\end{tabular}
};
\end{tikzpicture}
\end{figure}

This model can answer questions about the presence of a cause given the presence
of an effect. For example, What is the probability that it has being raining
given the grass is wet?

\[
P(R = T | G = T) = \frac{P(G = T, R = T)}{P(G=T)} = \frac{\sum_{s}P(G=T, R=T,
s)}{\sum_{r,s} P(G=T, r, s)}
\]

Using the expression of the joint probability among with the tables we can
compute every term. For example:
\[
\begin{aligned}
P(G=T, R=T, S=T) &= P(S=T|R=T)P(G=T|R=T,S=T)P(R=T) \\
&= 0.01 * 0.99 * 0.2 = 0.00198
\end{aligned}
\]
\end{exampleth}

\section{Structure Learning}

The Bayesian network structure is not always given and has to be learned from the data, to achieve this, there are some issues that need to be taken into account.
\begin{itemize}\setlength{\itemsep}{0.15cm}
  \item The number of Bayesian networks is exponential over the number of variables so a brute force algorithm would not be viable.
  \item Testing dependencies requires a large amount of data, therefore, a threshold must be set to measure when a dependence is significant.
  \item The presence of hidden variables might not be learned from the data.
\end{itemize}


\begin{algorithm}[t]
  \SetAlgoLined\KwData{Complete undirected graph \(G\), with vertices \(V\)}
  \KwResult{\(G\) with removed links}
  \(i = 0\)\;
  \While{all nodes have \(\leq i\) neighbors}{
    \For{\(X \in V\)}{
      \For{\(Y \in ne(x)\)}{
        \If{\(\exists S \subset ne(X)\backslash Y\) such that \(\#S = i\) and
          \(X \bigCI Y \mid S\)}{
          Remove \(X-Y\) from \(G\)\;
          \(S_{XY} = S\)\;
        }
      }
    }
    \(i = i+1\)\;
  }
  \caption{PC Algorithm for skeleton learning}\label{alg:pc}
\end{algorithm}

\subsection{PC Algorithm}

The \emph{PC algorithm} (\cite{spirtes2000causation} Chapter 5.4.2) learns the skeleton structure given a complete graph \(G=(V,E)\) (constructed from the considered set of variables) and orients these edges using variable independence in the empirical distribution.

The skeleton learning part (Algorithm~\ref{alg:pc}) iterates over subsets of neighborhoods, from smaller to bigger ones. It chooses a linked pair of variables \((X,Y) \in E\) and
a subset \(S_{XY} \subset ne(X)\), verifying \(Y \notin S_{XY}\). If \(X \bigCI Y \mid S\), then the link is removed and \(S_{XY}\) is stored.

The main idea behind the algorithm is that a set of independencies is faithful to a graph if (using Proposition~\ref{prop:bn_neig_indep}):
\[
  (X,Y) \not \in E \iff \exists S \subset ne(X) \ : \ X \bigCI Y \mid S.
\]

This procedure results in the skeleton of the Bayesian Network, no more edges will be removed or added. The directed graph may be constructed following two rules:
\begin{enumerate}
  \item For any undirected link \(X - Y - Z\), if \(Y \notin S_{XZ}\) then set
    \(X \to Y \leftarrow Z\) (we are creating a collider for that path).
  \item The rest of links may oriented arbitrarily not
creating cycles or colliders.
\end{enumerate}
The reasoning behind is the d-separation Theorem~\ref{th:d-separation},  if \(Y \notin S_{XZ}\) then \(X \bigCD Z \mid Y\) so \(Y\) must d-connect them, to get this we set it as a collider. On the other hand, if \(Y \in S_{XZ}\) and
\(X \bigCI Z \mid S_{XZ}\) then we want \(S_{XZ}\) to d-separate them, that is,
using any configuration that doesn't create a collider in \(S_{XZ}\).

\subsection{Independence Learning}

The PC algorithms assumes there exists a procedure of testing conditional independence of variables, that is, given three variables \(X, Y\) and \( Z \),  measure \(X \bigCI Y \mid Z\). One approach is to measure the empirical \emph{conditional mutual information} of the variables.

\begin{definition}
  Given two random variables \(X, Y\), we define their \emph{mutual information} as the Kullback-Leibler divergence of their joint distribution and the product of their marginals,
  \[
    MI(X;Y) = \KL{P_{X,Y}}{P_{X}P_{Y}}.
  \]
\end{definition}

\begin{definition}
  Given three random variables \(X, Y, Z\) we define the \emph{conditional mutual information} of \(X\) and \(Y\) over \(Z\) as
  \[
    MI(X;Y\mid Z) = \E{Z}{\KL{P_{X,Y \mid Z}}{P_{X\mid Z} P_{Y \mid Z}}}.
  \]
\end{definition}
Where \(MI(X;Y \mid Z) \geq 0\) and \(MI(X;Y \mid Z) = 0 \iff P_{X,Y \mid Z} = P_{X\mid Z} P_{Y \mid Z} \iff X\bigCI Y \mid Z\). We can estimate this using the empirical distributions, however, this \emph{empirical} mutual information will be typically greater than \(0\) even when \(X\bigCI Y \mid Z\), therefore a threshold must be established.

A Bayesian approach would consist on comparing the model likelihood under independence and dependence hypothesis:
\[
  P_{indep}(x,y,z \mid \btheta) = P(x\mid z, \btheta)P(y \mid z, \btheta)P(z \mid \btheta)P(\btheta),
\]
\[
P_{dep}(x,y,z \mid \btheta) = P(x,y,z \mid \btheta)P(\btheta).
\]
Which means checking which assumptions is most probable to have generated the data.
