
\emph{Inferential statistical} infer the underlying properties of a dataset or population, using different techniques as deriving estimates and testing hypotheses. This analysis are made under the assumption that the observed data is sampled from a larger population (\cite{upton2014dictionary}).

Compared to \emph{inferential statistics}, \emph{descriptive analysis} uniquely concerns about properties of the observed data and does not assume that this data comes from a larger population.

In the field of \emph{machine learning}, rather than inference, the procedure for deducing properties of the model is commonly referred to as \emph{training or learning}, in contrast, \emph{inference} refers to using a model for prediction.

A \emph{statistical model} is a set of assumptions concerning how the observer data was generated (\cite{cox2006principles}). There are different levels of modeling assumptions, which differ on whether the process that generates the data, is fully, partially or minimally described by a family of probability distributions. These distributions involve a finite amount of unknown parameters. This study's approach is \emph{fully parametric}, which assumes this generation is fully described by those parameters.

Different schools or paradigms of statistical inference have become established (\cite{bandyopadhyay2011philosophy}). These are not mutually exclusive, meaning that methods that work well on a given paradigm might have interpretations on other paradigms. In this chapter we are reviewing two of them: the \emph{Bayesian paradigm} and the \emph{likelihoodist paradigm}.

As purely Bayesian or likelihoodist methods are beyond the scope of this study, we are just reviewing the needed definitions to later understand related \emph{variational methods}, a few example are analyzed in each section.
