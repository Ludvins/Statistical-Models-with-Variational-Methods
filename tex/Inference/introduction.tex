
\emph{Inferential statistical analysis} infers properties of a population or dataset, using different techniques as testing hypotheses and deriving estimates. This analysis assumes that the observed data set is sampled from a larger population (\cite{upton2014dictionary}).

\emph{Descriptive statistics} is solely concerned with properties of the observed data, and opposed to \emph{inferential statistics} does not rest on the assumption that the data comes from a larger population.

In \emph{machine learning} the procedure for deducing properties of the model is typically referred to as \emph{training or learning} rather than inference, in contrast, using a model for prediction is referred to as \emph{inference}.

A \emph{statistical model} is a set of assumptions concerning the generation of the observed data and similar data (\cite{cox2006principles}). There are different levels of modeling assumptions, which differ on whether the process that generates the data, is fully, partially or minimally described by a family of probability distributions involving a finite amount of unknown parameters. This study's approach is \emph{fully parametric}, which assumes this generation is fully described by those parameters.

Different schools or paradigms of statistical inference have become established (\cite{bandyopadhyay2011philosophy}). These paradigms are not mutually exclusive, and methods that work well under one paradigm often have attractive interpretations under other paradigms. In this chapter we are reviewing two of them: the \emph{Bayesian paradigm} and the \emph{likelihoodist paradigm}.

As purely Bayesian or likelihoodist methods are beyond the scope of this study, we are just reviewing the needed definitions to later understand related \emph{variational methods}, a few example are analyzed in each section.
