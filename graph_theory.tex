
\begin{definition}
A \emph{graph} \(G = (V,E)\) is a set of vertices or nodes \(V\) and edges \(E\subset
V\times V\) between them.
If \(V\) is a set of ordered pairs then the graph is called a \emph{directed
  graph}, otherwise if \(V\) is a set of unordered pairs it is called an \emph{undirected graph}.
\end{definition}

\begin{figure*}[h]
\centering
\begin{tikzpicture}[
  node distance=1cm and 0.5cm,
  mynode/.style={draw,circle,text width=0.5cm,align=center}
]

\node[mynode] (a) {A};
\node[mynode,below right=of a] (b) {B};
\node[mynode,above right=of b] (c) {C};

\node[mynode, right=of c] (d) {A};
\node[mynode,below right=of d] (e) {B};
\node[mynode,above right=of e] (f) {C};

\path (c) edge[-latex] (a)
(a) edge[-latex] (b)
(b) edge[latex-] (c);

\draw (d) -- (e) -- (f) -- (d);

\end{tikzpicture}
\caption{Example of directed and undirected graph, respectively.}
\label{fig:graphs}
\end{figure*}

\begin{definition}
In a directed graph \(G = (V, E)\), a \emph{directed path} \(A \to B\) is a sequence of vertices \({A = A_0,
  A_1,\dots,A_{n-1}, A_n = B}\) where \((A_i, A_{i+1}) \in E \ \forall i \in
0,\dots ,n-1\).

If \(G\) is a undirected graph, \(A \to B\) is an \emph{undirected path} if \(\{A_i, A_{i+1}\} \in E \ \forall i \in
0,\dots, n-1\)
\end{definition}

\begin{definition}
Let \(A,B\) be two vertices of a directed graph \(G\). If \(A \to B\) is a
directed path and \(B \not \to A\) (meaning there isn't a directed path from
\(B\) to \(A\)), then \(A\) is called an \emph{ancestor} of \(B\) and \(B\) is called a \emph{descendant} of \(A\).
\end{definition}

For example, in the figure \ref{fig:graphs}, \(C\) is an ancestor of \(B\).

\begin{definition}
A \emph{directed acyclic graph (DAG)} is a directed graph such that no directed path between any two nodes revisits a vertex.
\end{definition}


\begin{figure}[h]
\centering
\begin{tikzpicture}[
  node distance=1cm and 0.5cm,
  mynode/.style={draw,circle,text width=0.5cm,align=center}
]

\node[mynode] (a) {A};
\node[mynode,below right=of a] (b) {B};
\node[mynode,above right=of b] (c) {C};

\path (c) edge[-latex] (a)
(a) edge[-latex] (b)
(b) edge[-latex] (c);

\end{tikzpicture}
\captionof{figure}{Example of graph which isn't a DAG.}
\label{fig:not_dag}
\end{figure}

As we can see in the figure \ref{fig:not_dag}, \(A \to B \to C \to A \to B\) is a
path from \(A\) to \(B\) that revisits \(A\).

Now where are going to define some relations between nodes in a DAG.

\begin{definition}
The \emph{parents} of a node \(A\) is the set of nodes \(B\) such that there is a
directed edge from \(B\) to \(A\). The same applies for the \emph{children} of a node.

The \emph{Markov blanket} of a node is composed by the node itself, its children, its parents and the parents
of its children.
\end{definition}


\begin{figure}[h]
\centering
\begin{tikzpicture}[
  node distance=1cm and 0.5cm,
  mynode/.style={draw,circle,text width=0.5cm,align=center}
]

\node[mynode] (a) {A};
\node[mynode,below right=of a] (b) {B};
\node[mynode,above right=of b] (c) {C};
\node[mynode,below right=of b] (d) {D};
\node[mynode,below left=of b] (e) {E};
\node[mynode,above right=of d] (f) {F};
\node[mynode, above right=of f] (h) {H};

\path (c) edge[-latex] (a)
(a) edge[-latex] (b)
(b) edge[latex-] (c)
(b) edge[-latex] (e)
(c) edge[-latex] (f)
(b) edge[-latex] (d)
(f) edge[-latex] (d)
(h) edge[-latex] (f)
;

\end{tikzpicture}
\captionof{figure}{Directed acyclic graph}
\label{fig:relations}
\end{figure}

\begin{definition}
In a graph, the \emph{neighbors} of a node are those directly connected
to it.
\end{definition}

We can use figure \ref{fig:relations} to reflect on these definitions. The parents
of \(B\) are \(pa(B) = \{A,C\}\) and its children are \(ch(B) = \{E,D\}\). Taking this into account, its neighbors
are \(ne(B) = \{A,C,E,D\}\) and its Markov blanket is \(\{A,B,C,D,E,F\}\).

